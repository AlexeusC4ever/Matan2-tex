\documentclass[a4paper, fleqn]{article}
\usepackage{header}

\title{Семинарский лист 5}
\author{
    Александр Богданов   \\ \href{https://t.me/SphericalPotatoInVacuum}{Telegram} \and
    Алиса Вернигор       \\ \href{https://t.me/allisyonok}{Telegram} \and
    Анастасия Григорьева \\ \href{https://t.me/weifoll}{Telegram} \and
    Василий Шныпко       \\ \href{https://t.me/yourvash}{Telegram} \and
    Данил Казанцев       \\ \href{https://t.me/vserosbuybuy}{Telegram} \and
    Денис Козлов         \\ \href{https://t.me/DKozl50}{Telegram} \and
    Елизавета Орешонок   \\ \href{https://t.me/eaoresh}{Telegram} \and
    Иван Пешехонов       \\ \href{https://t.me/JohanDDC}{Telegram} \and
    Иван Добросовестнов  \\ \href{https://t.me/ivankot13}{Telegram} \and
    Настя Городилова     \\ \href{https://t.me/nastygorodi}{Telegram} \and
    Никита Насонков      \\ \href{https://t.me/nnv_nick}{Telegram} \and
    Сергей Лоптев        \\ \href{https://t.me/beast_sl}{Telegram}
}

\date{Версия от {\ddmmyyyydate\today} \currenttime}

\begin{document}
    \maketitle
    
    \section*{Найдите множества абсолютной и условной сходимости функционального ряда.}
    % \subsection*{Задача 1}
    
    % \subsection*{Задача 2}
    
    % \subsection*{Задача 3}
    
    % \subsection*{Задача 4}
    
    % \subsection*{Задача 5}
    
    % \subsection*{Задача 6}
    
    \section*{Пользуясь необходимым условием равномерной сходимости, покажите, что ряд сходится на множестве $D$
        неравномерно.}
    % \subsection*{Задача 7}
    
    % \subsection*{Задача 8}
    
    % \subsection*{Задача 9}
    
    % \subsection*{Задача 10}
    
    \section*{Пользуясь локализацией особенности, покажите, что ряд сходится на множестве $D$ неравномерно.}
    % \subsection*{Задача 11}
    
    % \subsection*{Задача 12}
    
    % \subsection*{Задача 13}
    
    % \subsection*{Задача 14}
    
    \section*{Пользуясь критерием Коши, покажите, что ряд сходится на множестве $D$ неравномерно.}
    % \subsection*{Задача 15}
    
    % \subsection*{Задача 16}
    
    % \subsection*{Задача 17}
    
    % \subsection*{Задача 18}
    
    \section*{Пользуясь признаком Вейрштрасса, покажите, что ряд сходится на множестве $D$ равномерно.}
    % \subsection*{Задача 19}
    
    % \subsection*{Задача 20}
    
    % \subsection*{Задача 21}
    
    % \subsection*{Задача 22}
    
    \section*{Пользуясь признаком Лейбница, покажите, что знакочередующийся ряд сходится на множестве $D$
        равномерно.}
    % \subsection*{Задача 23}
    
    % \subsection*{Задача 24}
    
    \section*{Пользуясь признаком Дирихле или Абеля, покажите, что ряд сходится на множестве $D$ равномерно.}
    % \subsection*{Задача 25}
    
    % \subsection*{Задача 26}
    
    % \subsection*{Задача 27}
    
    % \subsection*{Задача 28}
    
\end{document}
