\documentclass[a4paper,fleqn]{article}
\usepackage{header}

\title{Семинарский лист 1337}
\author{
    Александр Богданов \\ \href{https://t.me/SphericalPotatoInVacuum}{Telegram} \and
    Алиса Вернигор     \\ \href{https://t.me/allisyonok}{Telegram} \and
    Василий Шныпко     \\ \href{https://t.me/yourvash}{Telegram} \and
    Денис Козлов       \\ \href{https://t.me/DKozl50}{Telegram} \and
    Иван Пешехонов     \\ \href{https://t.me/JohanDDC}{Telegram}
}

\date{Версия от {\ddmmyyyydate\today} \currenttime}

\begin{document}
    \maketitle
    \section*{Применяя признак Вейрштрасса, покажите, что ряд сходится абсолютно.}
    %Task 1 

    %Task 2

    %Task 3
    
    \section*{Применяя признак Лейбница, покажите, что ряд сходится.}
    %Task 4

    %Task 5

    %Task 6
    
    \section*{Применяя группировку членов постоянного знака, покажите, что ряд расходится.}
    %Task 7

    %Task 8

    %Task 9

    %Task 10

    %Task 11
    
    \section*{Применяя признак Дирихле или Абеля, покажите, что ряд сходится.}
    %Task 12

    %Task 13

    %Task 14

    %Task 15

    %Task 16

    %Task 17
    
    \section*{Исследуйте ряд на сходимость и абсолютную сходимость, используя асимптотику общего члена.}
    %Task 18

    %Task 19

    %Task 20

    %Task 21

    %Task 22

    %Task 23
    
    \section*{Вычислите произведение рядов.}
    %Task 24

    %Task 25

\end{document}
