\documentclass[a4paper,fleqn]{article}
\usepackage{header}

\title{Семинарский лист 1337}
\author{
    Александр Богданов \\ \href{https://t.me/SphericalPotatoInVacuum}{Telegram} \and
    Алиса Вернигор     \\ \href{https://t.me/allisyonok}{Telegram} \and
    Василий Шныпко     \\ \href{https://t.me/yourvash}{Telegram} \and
    Денис Козлов       \\ \href{https://t.me/DKozl50}{Telegram} \and
    Иван Пешехонов     \\ \href{https://t.me/JohanDDC}{Telegram}
}

\date{Версия от {\ddmmyyyydate\today} \currenttime}

\begin{document}
    \maketitle
    \section*{Применяя признак Вейрштрасса, покажите, что ряд сходится абсолютно.}
    %Task 1 

    %Task 2

    %Task 3
    
    \section*{Применяя признак Лейбница, покажите, что ряд сходится.}
    %Task 4

    %Task 5

    %Task 6
    
    \section*{Применяя группировку членов постоянного знака, покажите, что ряд расходится.}
    %Task 7

    %Task 8

    %Task 9

    %Task 10

    %Task 11
    
    \section*{Применяя признак Дирихле или Абеля, покажите, что ряд сходится.}
    %Task 12
    \subsection*{12}
    $\displaystyle \sum\limits_{n = 1}^{\infty} \frac{cos \; \sqrt{2}n}{2n - 5}$ \\
    Пусть $\displaystyle a_n = cos \; \sqrt{2}n$, тогда $\left| \sum\limits_{n = 1}^N a_n \right| \leq 1$, то есть частичная сумма $\displaystyle a_n$ ограничена. \\
    Пусть $\displaystyle b_n = \frac{1}{2n - 5}$, тогда, очевидно, $b_n \searrow 0$. \\
    Заметим, что $\displaystyle \sum\limits_{n = 1}^{\infty} \frac{cos \; \sqrt{2}n}{2n - 5} = \sum\limits_{n = 1}^{\infty} a_n \; \cdot \; b_n =>$ ряд сходится по признаку Дирихле. \\ 

    %Task 13
    \subsection*{13}
    $\displaystyle \sum\limits_{n = 2}^{\infty} \frac{sin(4n)}{ln \; n - ln \; ln \; n}$ \\
    $\displaystyle a_n = sin(4n) => \sum\limits_{n = 2}^{\infty} a_n - \text{ограничена}$. \\
    $\displaystyle b_n = \frac{1}{ln \; n - ln \; ln \; n}$ \\
    Покажем, что $\displaystyle b_n \searrow 0$: \\
    Пусть $\displaystyle f(x) = \frac{1}{ln \; x - ln \; ln \; x}$. Найдём её производную и покажем, что она всегда меньше нуля. Это будет означать, что функция, а значит, и $\displaystyle b_n$ монотонно убывает: \\
    \begin{equation*} f'(x) = -\frac{1 \; \cdot \; (ln \; x - ln \; ln \; x)'}{(ln \; x - ln \; ln \; x)^2} = -\frac{\frac{1}{x} - \frac{1}{x\; ln \; x}}{(ln \; x - ln \; ln \; x)^2} \end{equation*}
    При $\displaystyle x \rightarrow \infty \; \; \frac{1}{x} - \frac{1}{x\; ln \; x} > 0 => f'(x) < 0 => b_n$ монотонно убывает. Также заметим, что при $\displaystyle n \rightarrow \infty \; \; ln \; n$ растёт быстрее, чем $ln\; ln\; n => (ln \; n - ln \; ln \; n) \rightarrow \infty => b_n = \frac{1}{ln \; n - ln \; ln \; n} \rightarrow 0$. Следовательно $\displaystyle b_n \searrow 0 =>$ ряд сходится по признаку Дирихле. $\displaystyle \blacksquare$ \\

    %Task 14

    %Task 15
    \subsection*{15}
    $\displaystyle \frac{(-1)^n \; \cdot \; cos \; 3n}{\sqrt{n^2 + 2}}$ \\
    Пусть $\displaystyle a_n = (-1)^n \cdot cos\; 3n$. Докажем, что частичная сумма этого ряда ограничена. Для этого посчитаем $\displaystyle S_N^{+}$ и $\displaystyle S_N^{-}$ и докажем, что они ограничены. Для удобства рассмотрим такие $\displaystyle n$, что $\displaystyle n = 2 \cdot p, \; p \in \mathbb{N}$. \\
    Тогда $\displaystyle S_N^{+} = \sum\limits_{p=1}^{N} cos(6p)$ ограничена (доказано на семинаре). $\displaystyle S_N^{-} = \sum\limits_{p = 1}^{N} cos(3 + 6p) = \sum\limits_{p = 1}^{N} (cos \; 3 \; cos \; 6p - sin \; 3 \; sin \; 6p) = cos\; 3 \; \sum\limits_{p = 1}^{N} cos\; 6p - sin\; 3 \; \sum\limits_{p = 1}^{N} sin\; 6p \; - $ и уменьшаемое, и вычитаемое ограничены, значит и $S_N^{-}$ ограничена. \\
    Таким образом $\displaystyle S_N = S_N^{+} - S_N^{-}$ ограничена. \\
    Теперь пусть $\displaystyle b_n = \frac{1}{\sqrt{n^2 + 2}}$. Докажем, что $\displaystyle b_n \searrow 0$. Очевидно, что $b_n \rightarrow 0$. Для доказательства монотонного убывания сравним $\displaystyle b_n$ и $\displaystyle b_{n + 1}$: \\
    \begin{flalign*}
        & \frac{1}{\sqrt{n^2 + 2}}\; \vee \; \frac{1}{\sqrt{(n+1)^2 + 2}} & \\
        & \frac{1}{n^2 + 2}\; \vee \; \frac{1}{(n+1)^2 + 2} & \\
        & (n + 1)^2 + 2 \; \vee \; n^2 + 2 & \\
        & n^2 + 2n + 3 \; \vee \; n^2 + 2 & \\
        & 2n + 1 > 0 \text{, начиная с какого-то} \; n_0. &
    \end{flalign*}
    Следовательно, $\displaystyle b_n \searrow 0$. Значит, наш ряд сходится по признаку Дирихле.

    %Task 16
    \subsection*{16}
    $\displaystyle \sum\limits_{n=1}^{\infty} \frac{(3n - 2)\; \cdot \; sin(n)}{n^2 - 3n + 1}$ \\
    $\displaystyle a_n = sin(n) => \sum\limits_{n = 1}^{\infty} a_n$ $\displaystyle -$ ограничена. \\
    $\displaystyle b_n = \frac{3n - 2}{n^2 - 3n + 1}$ \\ \\
    Докажем, что $\displaystyle b_n \searrow 0$. Очевидно, что $\displaystyle b_n \rightarrow 0$. Для дальнейшего доказательства сравним $\displaystyle b_n$ и $\displaystyle b_{n + 1}$: \\
    \begin{flalign*}
        & \frac{3n - 2}{n^2 - 3n + 1} \; \vee \; \frac{3n + 1}{(n + 1)^2 - 3n - 2} & \\
        & (3n - 2) ((n + 1)^2 - 3n - 2) \; \vee \; (3n + 1)(n^2 - 3n + 1) & \\
        & (3n - 2) (n^2 - n - 1) \; \vee \; (3n + 1)(n^2 - 3n + 1) & \\
        & 3n^3 - 3n^2 - 3n - 2n^2 + 2n + 2 \; \vee \; 3n^3 - 9n^2 + 3n + n^2 -3n + 1 & \\
        & 3n^2 - n + 1 > 0 \; \text{начиная с какого-то} \; n_0 => \; \text{ряд сходится по признаку Дирихле.} \; \blacksquare&
    \end{flalign*} \\

    %Task 17
    
    \section*{Исследуйте ряд на сходимость и абсолютную сходимость, используя асимптотику общего члена.}
    %Task 18

    %Task 19

    %Task 20

    %Task 21
    \subsection*{21}
    $\displaystyle \sum\limits_{n=1}^{\infty} \frac{cos \; n}{\sqrt{n} + cos \; n}$ \\
    \begin{equation*} a_n = \frac{cos \; n}{\sqrt{n} + cos \; n} = \frac{cos \; n}{\sqrt{n}} \cdot \frac{1}{1+\frac{cos\; n}{\sqrt{n}}} = \frac{cos\; n}{\sqrt{n}} \cdot (1 - \frac{cos\; n}{\sqrt{n}} + \frac{\frac{cos^2n}{n}}{1 + \frac{cos\; n}{\sqrt{n}}}) = \frac{cos\; n}{\sqrt{n}} - \frac{cos^2n}{n} + \frac{\frac{cos^3n}{n^{1.5}}}{1 + \frac{cos\; n}{\sqrt{n}}} = O(\frac{cos \; n}{\sqrt{n}})\end{equation*}
    $\displaystyle \frac{cos \; n}{\sqrt{n}}$ сходится по признаку Дирихле $\displaystyle => \sum\limits_{n=1}^{\infty} a_n$ сходится условно. \\
    Рассмотрим теперь абсолютную сходимость. $\displaystyle |a_n| = \left|\frac{cos \; n}{\sqrt{n}}\right| = \frac{|cos \; n|}{\sqrt{n}}$. С семинара известно, что $\displaystyle \sum\limits_{n = 1}^{\infty} \frac{|cos\; n|}{n}$ расходится, следовательно, так как $\displaystyle \frac{|cos\; n|}{n} \leq \frac{|cos\; n|}{\sqrt{n}}$, то по признаку сравнения $\sum\limits_{n = 1}^{\infty} \frac{|cos\; n|}{\sqrt{n}}$ расходится $\displaystyle =>$ ряд расходится абсолютно. \\

    %Task 22

    %Task 23
    
    \section*{Вычислите произведение рядов.}
    %Task 24

    %Task 25

\end{document}
