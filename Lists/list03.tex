\documentclass[a4paper,fleqn]{article}
\usepackage{header}

\title{Семинарский лист 1337}
\author{
    Александр Богданов \\ \href{https://t.me/SphericalPotatoInVacuum}{Telegram} \and
    Алиса Вернигор     \\ \href{https://t.me/allisyonok}{Telegram} \and
    Василий Шныпко     \\ \href{https://t.me/yourvash}{Telegram} \and
    Денис Козлов       \\ \href{https://t.me/DKozl50}{Telegram} \and
    Иван Пешехонов     \\ \href{https://t.me/JohanDDC}{Telegram}
}

\date{Версия от {\ddmmyyyydate\today} \currenttime}

\begin{document}
    \maketitle
    \section*{Применяя признак Вейрштрасса, покажите, что ряд сходится абсолютно.}
    %Task 1 

    %Task 2

    %Task 3
    
    \section*{Применяя признак Лейбница, покажите, что ряд сходится.}
    %Task 4
    \subsection*{4}
    \[ \sum_{n=1}^{\infty} \dfrac{(-1)^n (2n-1)}{n^2 + 3n + 5} \text{ --- знакочередующийся ряд}, \;\;\;
		 a_n = \dfrac{(-1)^n (2n-1)}{n^2 + 3n + 5}, \;\; |a_n|  = \dfrac{2n-1}{n^2 + 3n + 5} \]
		\[ \text{Проверим, что $|a_n|$ монотонно убывает. Рассмотрим $f(x) = \dfrac{2x-1}{x^2 + 3x + 5}$:} \]
		\[ f'(x) = \dfrac{2(x^2 + 3x + 5) - (2x-1)(2x+3)}{(x^2 + 3x + 5)^2} = 
		\dfrac{2x^2 + 6x + 10 - 4x^2 - 4x + 3}{(x^2 + 3x + 5)^2} = 
		-\dfrac{2x^2 - 2x - 3}{(x^2 + 3x + 5)^2} = \]
		\[ = -\dfrac{2\left(x - \dfrac12\right)^2 - \dfrac72}{(x^2 + 3x + 5)^2} < 0 \text{ при } x \ge 2  \;\Rightarrow \]
		\[ \Rightarrow \; |a_n| \searrow \text{ начиная с } n = 2 \]\\[-20 pt]
		\[ \lim_{n\to\infty} |a_n| = \lim_{n\to\infty} \dfrac{2n-1}{n^2 + 3n + 5} = \lim_{n\to\infty}\dfrac{2-\dfrac1n}{n + 3 + \dfrac5n} = 0 \]\\[-10 pt]
		\[ \left\{\begin{array}{l} 
		\text{ряд знакочередующийся},\\[5 pt]
		|a_n| > |a_{n+1}| \; \forall n \ge 2,\\[5 pt]
		\lim_{n\to\infty} |a_n| = 0
		\end{array}\right.
		\Rightarrow \; \sum_{n=1}^{\infty} \dfrac{(-1)^n (2n-1)}{n^2 + 3n + 5} \; \text{ сходится условно по признаку Лейбница.} \]\\[-5 pt]
		\[ \sum_{n=1}^{\infty} |a_n| = \sum_{n=1}^{\infty} \dfrac{2n-1}{n^2 + 3n + 5} \sim \sum_{n=1}^{\infty} \dfrac{1}{n} \text{ --- расходится по признаку сравнения } \Rightarrow \text{ абсолютной сходимости нет.} \]\\

    %Task 5
    \subsection*{5}
    \[ \sum_{n=1}^{\infty} \dfrac{(-1)^n \ln^2 n}{\sqrt{2n + 3}} \text{ --- знакочередующийся ряд}, \;\;\;
		 a_n = \dfrac{(-1)^n \ln^2 n}{\sqrt{2n + 3}}, \;\; |a_n|  = \dfrac{\ln^2 n}{\sqrt{2n + 3}} \]
		\[ \text{Проверим, что $|a_n|$ монотонно убывает. Рассмотрим $f(x) = \dfrac{\ln^2 x}{\sqrt{2x + 3}}$:} \]
		\[ f'(x) = \dfrac{\dfrac{2\ln x}x \sqrt{2x + 3} - \ln^2 x \dfrac{2}{2\sqrt{2x + 3}}}{2x + 3} = 
		\dfrac{\ln x (4x + 6 - x \ln x)}{x (2x + 2)^{3/2}} = 
		-\dfrac{\ln x }{(2x + 2)^{3/2}} \left( \ln x - 4 - \dfrac6x \right) < 0 \text{ при } x \ge 50  \;\Rightarrow \]
		\[ \Rightarrow \; |a_n| \searrow \text{ начиная с } n = 50 \]\\[-20 pt]
		\[ \lim_{n\to\infty} |a_n| = \lim_{n\to\infty} \dfrac{\ln^2 n}{\sqrt{2n + 3}} =  0, \; \text{ т.к. } \ln^2 n = o(\sqrt{n}) \]\\[-10 pt]
		\[ \left\{\begin{array}{l} 
		\text{ряд знакочередующийся},\\[5 pt]
		|a_n| > |a_{n+1}| \; \forall n \ge 50,\\[5 pt]
		\lim_{n\to\infty} |a_n| = 0
		\end{array}\right.
		\Rightarrow \; \sum_{n=1}^{\infty} \dfrac{(-1)^n \ln^2 n}{\sqrt{2n + 3}} \; \text{ сходится условно по признаку Лейбница.} \]\\[-5 pt]
		\[ \sum_{n=1}^{\infty} |a_n| = \sum_{n=1}^{\infty} \dfrac{\ln^2 n}{\sqrt{2n + 3}} > \sum_{n=1}^{\infty} \dfrac{1}{n} \text{ --- расходится по признаку сравнения } \Rightarrow \text{ абсолютной сходимости нет.} \]\\

    %Task 6
    
    \section*{Применяя группировку членов постоянного знака, покажите, что ряд расходится.}
    %Task 7
    \subsection*{7}
    \[ \sum_{n=1}^{\infty} \dfrac{(-1)^{[\ln n]}}{2n + 1} \text{ --- знакопеременный ряд} \]
		\[ \text{При $2k \le [\ln n] < 2k + 1 \; \Leftrightarrow \; [e^{2k}] \le n < [e^{2k+1}]$
		$n$-е слагаемое положительно.\\[5 pt]} \]
		\[ \text{Перестановка:} \]
		\[ \sum_{k=1}^{\infty} \sum_{n=[e^{2k}]}^{[e^{2k+1}]-1} \dfrac{(-1)^{[\ln n]}}{2n + 1} =
		\sum_{k=1}^{\infty} \sum_{n=[e^{2k}]}^{[e^{2k+1}]-1} \dfrac{1}{2n + 1} \]
		\[ \text{Оценим сумму группы 
		$A_k = \displaystyle \sum_{n=[e^{2k}]}^{[e^{2k+1}]-1} \dfrac{1}{2n + 1}$ снизу:} \]
		\[ \sum_{n=[e^{2k}]}^{[e^{2k+1}]-1} \dfrac{1}{2n + 1} \ge 
		\dfrac1{2[e^{2k+1}] - 1}\left( [e^{2k+1}]-1 - [e^{2k}] \right) \ge 
		\dfrac{2[e^{2k}] - [e^{2k}]-1}{2[e^{2k+1}] - 1} \ge 
		\dfrac{ [e^{2k}]-1}{2[e^{2k+1}] - 1} \ge
		\dfrac{ [e^{2k + 1}]}{2[e^{2k+1}]} = \dfrac12 \ne 0 \]\\[-20 pt]
		\[ \lim_{k\to\infty} |A_k| = \dfrac12 \ne 0 \; \Rightarrow \; 
		\sum_{k=1}^{\infty} A_k \text{ расходится, т.к. не выполнено необходимое условие сходимости} \; \Leftrightarrow \]
		\[ \Leftrightarrow \; \sum_{n=1}^{\infty} \dfrac{(-1)^{[\ln n]}}{2n + 1} \; \text{ тоже расходится} \]

    %Task 8

    %Task 9
    \subsection*{9}
    \[ \sum_{n=1}^{\infty} \dfrac{(-1)^{[\sqrt[3]{n}]}}{\sqrt[3]{n^2 + 3}} \text{ --- знакопеременный ряд} \]
		\[ \text{При $2k \le [\sqrt[3]{n}] < 2k + 1 \; \Leftrightarrow \; 8k^3 \le n < (2k+1)^3$
		$n$-е слагаемое положительно.\\[5 pt]} \]
		\[ \text{Перестановка:} \]
		\[ \sum_{k=1}^{\infty} \sum_{n=8k^3}^{8k^3+12k^2+6k} \dfrac{(-1)^{[\sqrt[3]{n}]}}{\sqrt[3]{n^2 + 3}} =
		\sum_{k=1}^{\infty} \sum_{n=8k^3}^{8k^3+12k^2+6k} \dfrac{1}{\sqrt[3]{n^2 + 3}} \]
		\[ \text{Оценим сумму группы 
		$A_k = \displaystyle  \sum_{n=8k^3}^{8k^3+12k^2+6k} \dfrac{1}{\sqrt[3]{n^2 + 3}}$ снизу:} \]
		\[  \sum_{n=8k^3}^{8k^3+12k^2+6k} \dfrac{1}{\sqrt[3]{n^2 + 3}} \ge 
		\dfrac{1}{\sqrt[3]{((2k+1)^3 - 1)^2 + 3}}(12k^2 + 6k) = 
		\dfrac{6k(2k+1)}{\sqrt[3]{(2k+1)^6 - 2(2k + 1)^3 + 4}} = \]
		\[ = \dfrac{6k}{\sqrt[3]{(2k+1)^3 - 2 + \dfrac4{(2k + 1)^3}}} \ge
		\dfrac{6k}{2k+1} \; (\text{при } k \ge 1) \ge 3 \ne 0 \]\\[-20 pt]
		\[ \lim_{k\to\infty} |A_k| = 3 \ne 0 \; \Rightarrow \; 
		\sum_{k=1}^{\infty} A_k \text{ расходится, т.к. не выполнено необходимое условие сходимости} \; \Leftrightarrow \]
		\[ \Leftrightarrow \; \sum_{n=1}^{\infty} \dfrac{(-1)^{[\sqrt[3]{n}]}}{\sqrt[3]{n^2 + 3}} \; \text{ тоже расходится} \]

    %Task 10

    %Task 11
    
    \section*{Применяя признак Дирихле или Абеля, покажите, что ряд сходится.}
    %Task 12

    %Task 13

    %Task 14

    %Task 15

    %Task 16

    %Task 17
    
    \section*{Исследуйте ряд на сходимость и абсолютную сходимость, используя асимптотику общего члена.}
    %Task 18

    %Task 19

    %Task 20

    %Task 21

    %Task 22

    %Task 23
    
    \section*{Вычислите произведение рядов.}
    %Task 24

    %Task 25

\end{document}