\documentclass[a4paper, fleqn]{article}
\usepackage{header}

\title{Семинарский лист 1337}
\author{
    Денис Козлов       \\ \href{https://t.me/DKozl50}{Telegram}
}

\date{Версия от {\ddmmyyyydate\today} \currenttime}

\begin{document}
    \maketitle
    
    \section*{Через секшн* пишем условия групп задач}
    
    \subsection*{Задача 1}
    Номер задачи пишется через сабсекшн*
    
    % \subsection*{Задача 2}
    
    \subsection*{Задача 3}
    \paragraph*{Решение 1} 
    Параграфами* отделяем разные варианты решений, если актуально
    
    \paragraph*{Решение 2}
    Вот так выглядит второе решение.
    
    % \subsection*{Задача 4}
    
    % \subsection*{Задача 5}
    
    \subsection*{Задача 6}
    Мы пишем слова ''решение'' и ''задача'', потому что так легче ориентироваться в файле.
    
    \subsection*{Задача 7}
    Некоторые задачи отсутствуют — значит их еще не затехали. Решено не писать названия нерешенных задач
    
    \section*{О математике внутри решений}
    % \subsection*{Задача 8}
    
    \subsection*{Задача 9}
    В документе действует опция fleqn. Это значит, что математика хочет находиться по левому краю.
    Если мы хотим писать длинную математику, стоит использовать align*, обязательно ставя в начале строк ''\&''.
    Внутри элайна* математика дисплейного стиля, можно делать разрывы строк.
    \begin{align*}
        & 1 + 2 + 3 + 4 + 5 \\
        & \sum_{i=1}^5 i \\
        & \text{тескт писать можно через \textbackslash text} \\
        & \begin{pmatrix}
            1 & 2 \\
            4 & 3
        \end{pmatrix} \;\;\;\;\;\; \text{ внутри элайна можно вставлять и другие математические окружения}
    \end{align*}
    Вот элайн и закончился всем привет

    % \subsection*{Задача 10}
    
    \subsection*{Задача 11}
    Если в решении много текста, но мало математики, можно пользоваться математикой через 
    $\sum_{n=1}^{0!} \frac{4 + 5}{9} = 1$
    доллар. Кажется дисплейстайл активируется автоматом. 

    \subsection*{Задача 12}
    Если хочется написать одну большую и длинную строку математики, то все-таки давайте стараться делать это в
    элайне*, а не через два доллара или \textbackslash[ \textbackslash] 

    \subsection*{Задача 13}
    Хорошие советы по коду предлагаю читать в ридми хсе-тех, в особенности используйте \textbackslash left 
    \textbackslash right.
    
    \subsection*{Задача 14}
    И, пожалуйста, старайтесь ставить пробелы, а не табы
    
    % \subsection*{Задача 15}
    
    % \subsection*{Задача 16}
    
    % \subsection*{Задача 17}
    
    % \subsection*{Задача 18}
    
    % \subsection*{Задача 19}
    
    % \subsection*{Задача 20}
    
    % \subsection*{Задача 21}
    
    % \subsection*{Задача 22}
    
    % \subsection*{Задача 23}
    
    % \subsection*{Задача 24}
    
    % \subsection*{Задача 25}
    
    % \subsection*{Задача 26}
    
    % \subsection*{Задача 27}
    
    % \subsection*{Задача 28}
    
    % \subsection*{Задача 29}
    

\end{document}
