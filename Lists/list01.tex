\section{Листок №1. Частичная сумма для ряда и необходимое условие сходимости.}

\subsection{Вычислите частичную сумму ряда и исследуйте ее предел.}

\textbf{1.} $\sum_{n=1}^{\infty} \frac{1}{(3n-1)(3n+2)} = \sum_{n=1}^{\infty} \frac{A}{3n-1} + \frac{B}{3n+2} = \sum_{n=1}^{\infty} \frac{3Bn-B+3An+2A}{(3n-1)(3n+2)} = \left[ \begin{cases} 3B + 3A = 0\\ 2A - B = 1 \end{cases} \implies \begin{cases} A = \frac{1}{3} \\ B = \frac{-1}{3} \end{cases}  \right] = \frac{1}{3} \sum_{n=1}^{\infty} \frac{1}{3n-1} - \frac{1}{3n+2} = \frac{1}{3} (\frac{1}{2} - \frac{1}{5} + \frac{1}{5} - \frac{1}{8} + \frac{1}{8} - \dots + \frac{1}{3n-1} - \frac{1}{3n+2}) = \frac{1}{3} (\frac{1}{2} - \frac{1}{3n+2}) = \lim_{n \to \infty} \sum =  \frac{1}{6}$

\textbf{2.} $\sum_{n=1}^{\infty} \frac{n}{(4n^2-1)^2} = \sum_{n=1}^{\infty} \frac{A}{(2n-1)^2} + \frac{B}{(2n+1)^2} =$

$\begin{cases} 4A+4B = 0\\ 4A - 4B = 1\\ A + B = 0 \end{cases} \implies \begin{cases} A = \frac{1}{8}\\ B = -\frac{1}{8} \end{cases}$

$= \frac{1}{8} \sum_{n=1}^{\infty} \frac{1}{(2n-1)^2} - \frac{1}{(2n+1)^2} = \frac{1}{8} \left(1 - \frac{1}{9} + \frac{1}{9} - \frac{1}{25} + \frac{1}{25} - \dots + \frac{1}{(2n-1)^2} - \frac{1}{(2n+1)^2}\right) = \frac{1}{8} \left(1 - \frac{1}{(2n+1)^2}\right) = \lim_{n \to \infty} \sum = \frac{1}{8}$

\textbf{3.}  $\sum_{n=1}^{\infty} \frac{1}{n(n+1)(n+2)} = \sum_{n=1}^{\infty} \frac{A}{n} + \frac{B}{n+1} + \frac{C}{n+2} = \frac{An^3+3An+2A+Bn^2+2B+Cn^2+C}{n(n+1)(n+2)} = $

$$\begin{cases} A + B + C = 0\\ 3A = 0\\ 2A+2B+C=1 \end{cases} \implies \begin{cases} A = 0\\ B = 1\\ C=-1 \end{cases}$$

$= \sum_{n=1}^{\infty} \frac{1}{n+1} - \frac{1}{n+2} = \frac{1}{2} - \frac{1}{3} + \frac{1}{3} - \frac{1}{4} + \frac{1}{4} - \dots + \frac{1}{n+1} - \frac{1}{n+2} = \frac{1}{2} - \frac{1}{n + 2} = \lim_{n \to \infty} \sum = \frac{1}{2}$

\textbf{4.} $\sum_{n=1}^{\infty} \frac{2n+1}{n(n+1)(n+3)} = \frac{1}{6} \sum_{n=1}^{\infty} \frac{2}{n} + \frac{3}{n+1} - \frac{5}{n + 3}$

Заметим, что $\frac{2}{k} + \frac{3}{k} - \frac{5}{k} = 0$. Т.е. 3 члена суммы с одинаковыми знаменателями уничтожатся. Так как первый общий знаменатель 4, а последний $n$, получаем $\frac{1}{6} \sum_{n=1}^{\infty} \frac{2}{n} + \frac{3}{n+1} - \frac{5}{n + 3} = \frac{1}{6} \cdot \left(\left(\sum_{n=1}^{3} \frac{2}{n}\right) + \left(\sum_{n=1}^{2} \frac{3}{n+1}\right) + \frac{3}{n+1} - \left(\sum_{n=n-2}^{n} \frac{5}{n+3}\right)\right) = \frac{37}{36} - \frac{12n^2+45n+37}{6(n+1)(n+2)(n+3)} = \lim_{n \to \infty} \sum = \frac{37}{36}$

\textbf{5.} $\sum_{n=1}^{\infty} \frac{1}{\sqrt{n} + \sqrt{n + 1}} = \sum_{n=1}^{\infty} \frac{\sqrt{n} - \sqrt{n+1}}{n - (n + 1)} = \sum_{n=1}^{\infty} \sqrt{n+1} - \sqrt{n} = \sqrt{2} - 1 + \sqrt{3} - \sqrt{2} + \sqrt{4} - \sqrt{3} + \dots - \sqrt{n+1} + \sqrt{n} = \sqrt{n+1} - 1 = \lim_{n \to \infty} \sum = \infty$

\textbf{6.} $\sum_{n=1}^{\infty}(\sqrt[3]{n+2}-2\cdot\sqrt[3]{n+1}+\sqrt[3]{n}):$ \\
$S_N=\sum_{n=1}^{\infty}(\sqrt[3]{n+2}-2\cdot\sqrt[3]{n+1}+\sqrt[3]{n})=\sum_{n=1}^{N}\sqrt[3]{n+2}-2\sum_{n=1}^{N}\sqrt[3]{n+1}+\sum_{n=1}^{N}\sqrt[3]{n}=\sum_{n=3}^{N+2}\sqrt[3]{n}-2\sum_{n=2}^{N+1}\sqrt[3]{n}+\sum_{n=1}^{N}\sqrt[3]{n}=(\sqrt[3]{N+1}+\sqrt[3]{N+2})-2(\sqrt[3]{2}+\sqrt[3]{N+1})+(\sqrt[3]{1}+\sqrt[3]{2})=1-\sqrt[3]{2}+\sqrt[3]{N+2}-\sqrt[3]{N+1}=1-\sqrt[3]{2}+\sqrt[3]{N}\Big( (1+\frac{2}{N})^{1/3} - (1+\frac{1}{N})^{1/3} \Big) = \\ = \Big[ (1+x)^{a}=1+ax+O(x) \Big] = 1-\sqrt[3]{2}+\sqrt[3]{N}\Bigg( \Big(1+\frac{1}{3}\cdot \frac{2}{N} + O(\frac{1}{N})\Big) - \Big(1+\frac{1}{3}\cdot \frac{1}{N} + O(\frac{1}{N})\Big)  \Bigg)=1-\sqrt[3]{2}+\sqrt[3]{N}\Big( \frac{1}{3N} + O(\frac{1}{N}) \Big) \xrightarrow [\text{N $\to$ $\infty$}]{} 1 - \sqrt[3]{2}$

\textbf{7.} $\sum_{n=1}^{\infty} \frac{n}{(n+1)!} = \sum_{n=1}^{\infty} \frac{n+1-1}{(n+1)!} = \sum_{n=1}^{\infty} \frac{1}{n!} - \frac{1}{(n+1)!} = 1 - \frac{1}{(n+1)!} = \lim_{n \to \infty} \sum = 1$

\textbf{8.} $\sum_{n=1}^{\infty} \frac{1}{n!(n+2)} = \sum_{n=1}^{\infty} \frac{n + 1 + 1 - 1}{(n+2)!} = \sum_{n=1}^{\infty} \frac{1}{(n+1)!} - \frac{1}{(n+2)!} = \frac{1}{2!} - \frac{1}{3!} + \frac{1}{3!} - \dots + \frac{1}{(n+1)!} - \frac{1}{(n+2)!} = \frac{1}{2} - \frac{1}{(n+2)!} = \lim_{n \to \infty} \sum = \frac{1}{2}$

\textbf{9.} $\sum_{n=1}^{\infty}(n+1)x^n: \\ \sum_{n=1}^{N}(n+1)x^n=\sum_{n=1}^{N}n\cdot x^n + \sum_{n=1}^{N}x^n= x\sum_{n=1}^{N}n\cdot x^{n-1} + \sum_{n=1}^{N}x^n= \Big[ x \neq 1 \Big]=x \Big( x \cdot \frac{x^N - 1}{x - 1} \Big)' + \Big( x \cdot \frac{x^N - 1}{x - 1} \Big)=x \Big( \frac{x^{N+1} - x}{x - 1} \Big)' + \Big( \frac{x^{N+1} - x}{x - 1} \Big)=x\cdot \frac{((N+1)x^N-1)\cdot(x-1)-(x^{N+1}-x)}{(x-1)^2}+\Big( \frac{x^{N+1} - x}{x - 1} \Big) \\$ Получаем, что для:
\begin{enumerate}
	\item  $|x| \geq 1 \Longrightarrow a_n(x)=(n+1)x^n \longrightarrow \infty \implies $ ряд расходится;
	\item $|x| < 1 \Longrightarrow x^{N} \longrightarrow 0,\ \ \ N\cdot x^{N} \implies 0$ ряд сходится.
\end{enumerate}

\textbf{10.} $\sum_{n=1}^{\infty} (2n-1)x^n = 2\sum_{n=1}^{\infty}nx^n - \sum_{n=1}^{\infty} x^n = 2x\cdot \frac{((N+1)x^N-1)\cdot(x-1)-(x^{N+1}-x)}{(x-1)^2} - \Big( \frac{x^{N+1} - x}{x - 1} \Big)$

\begin{enumerate}
	\item  $|x| \geq 1 \Longrightarrow a_n(x)=(2n-1)x^n \longrightarrow \infty \implies $ ряд расходится;
	\item $|x| < 1 \Longrightarrow x^{N} \longrightarrow 0,\ \ \ 2N\cdot x^{N} \implies 0$ ряд сходится.
\end{enumerate}

\textbf{11.} $\sum_{n=1}^{\infty} \sin nx = \sin x + \sin 2x + \dots + \sin nx = \dfrac{\sin x \cdot \sin \frac{x}{2} + \sin 2x \cdot \sin \frac{x}{2} + \dots + \sin nx \cdot \sin \frac{x}{2}}{\sin \frac{x}{2}} =$

Воспользуемся формулой произведения синусов:


$ = \frac{\frac{1}{2} \left(\cos \frac{x}{2} - \cos \frac{3x}{2} \right) + \frac{1}{2} \left(\cos \frac{3x}{2} - \cos \frac{5x}{2} \right) + \dots + \frac{1}{2} \left(\cos (n-\frac{1}{2})x - \cos (n + \frac{1}{2})x \right)}{\sin \frac{x}{2}} = \frac{\cos \frac{x}{2} - \cos (n + \frac{1}{2})x}{2\sin \frac{x}{2}}$

$\lim_{n \to \infty} S_n$ не существует при $x \neq \pi k, k \in \ZZ$ т.к. $\nexists \lim_{n \to \infty} \cos (n + \frac{1}{2})x$, но если $x = \pi k, k \in \ZZ$, то $sin nx = 0$ и $\sum_{n=1}^{\infty} \sin nx = 0 + \dots + 0 = 0$

\textbf{12.} $\sum_{n=1}^{\infty} \cos^2 nx = \frac{1}{2} \sum_{n=1}^{\infty} (1 + \cos{(2nx)}) = \frac{n(n+1)}{4} + \frac{1}{2} \sum_{n=1}^{\infty} \cos{(2nx)}$

Дальше можно решать как задачу 11 или так:

$e^{ix} = \cos x + i\sin x, \cos x = \Re(e^{ix}) \implies \cos 2x + \cos 4x + \dots + \cos 2Nx = \Re(1 + e^{2ix} + e^{4ix} + \dots + e^{2Nix}) = \Re\left(\frac{e^{2(N+1)ix} - 1}{e^{2ix} - 1}\right)$

\subsection{Докажите, что ряд расходится}

\textbf{15.} $\sum_{n=1}^{\infty} \frac{n-1}{3n+2}$

Проверим необходимое условие сходимости т.е., что $\lim_{n \to \infty} a_n \to 0$. $\lim_{n \to \infty} \frac{n-1}{3n+2} = \lim_{n \to \infty} \dfrac{1-\frac{1}{n}}{3+\frac{2}{n}} = \frac{1}{3} \implies$ ряд расходится.

\textbf{16.} $\sum_{n=1}^{\infty} \frac{1}{\sqrt[n]{n}}$

$\lim_{n \to \infty} a_n = \frac{1}{\sqrt[n]{n}} = \frac{1}{1} = 1$ ряд расходится. $\sqrt[n]{n} = 1$ очевидный факт с 1 курса, но его легко можно доказать: $ \sqrt[n]{n} = e^{\frac{\ln n}{n}}$. $\lim_{n \to \infty} \frac{\ln n}{n}=$ по Лопиталю $= \frac{1}{n} = 0 \implies \lim_{n \to \infty} e^{\frac{\ln n}{n}} = e^0 = 1$

\textbf{17.} $\sum_{n=1}^{\infty}$

$\lim_{n \to \inf} a_n = \frac{\sin{(\frac{n+1}{n^2+2})} \to 0}{\frac{1}{n} \to 0} =$[применяем Лопиталя]$= \frac{n^2(n^2+2n-2) \cos{(\frac{n+1}{n^2+2})}}{(n^2+2)^2} = 1$ т.к. и в числителе и в знаменателе наибольшая степень $n^4$ с коэффициентами 1.

\textbf{18.} $\sum_{n=1}^{\infty} n (\sqrt[n]{3} - 1)$

$a_n = n \cdot (3^{\frac{1}{n}} - 1)$, вспомним оценку $e^x = 1 + x + o(x)$, тогда $3^{\frac{1}{n}} = e^{\frac{\ln 3}{n}} = 1 + \frac{\ln 3}{n} + o(\frac{1}{n}) \implies a_n = n\cdot (1 + \frac{\ln 3}{n} + o(\frac{1}{n}) - 1) = \ln 3 + o(1) \to \ln 3$

\textbf{19.} $\sum_{n=1}^{\infty} \left(1 - \frac{1}{n}\right)^n$

Докажем одно из свойств замечательного предела $\lim_{x \to \infty} \left(1 + \frac{k}{x}\right)^x = [ u = \frac{x}{k}, u\to \infty ] =\lim_{u \to \infty} \left(1 + \frac{1}{u}\right)^{u\cdot k} = e^k$

В нашем примере $k = -1$, поэтому $\lim_{n \to \infty} a_n \to \frac{1}{e}$

\textbf{20.} $\sum_{n=1}^{\infty} e^{-n}\cdot \left(1 + \frac{1}{n}\right)^{n^2}$

Хочется воспользоваться вторым замечательным пределом, но это ловушка, делаем так: $$\lim_{n \to \infty} a_n = e^{-n}\cdot \left(1 + \frac{1}{n}\right)^{n^2} = \lim_{n \to \infty} e^{n^2\ln{\left(1 + \frac{1}{n}\right)} - n} = \lim_{n \to \infty} e^{n^2\cdot \left(\frac{1}{n} - \frac{1}{2n^2} + o\left(\frac{1}{n^2}\right)\right) - n} =\lim_{n \to \infty} e^{n - \frac{1}{2} + o(1) - n} = \frac{1}{\sqrt{e}}$$

\subsection{При каких значениях x для ряда выполнено необходимое условие сходимости}

\textbf{21.1} $\sum_{n=1}^{\infty} \frac{nx}{1+n^2x^2}$

Рассмотрим случаи:
\begin{enumerate}
	\item $x = 0 \implies \forall n:\; a_n = 0$
	\item $x \neq 0$. Поделим на $nx:\; \lim_{n \to \infty} \dfrac{1}{\frac{1}{nx}+nx} = \lim_{n \to \infty} \dfrac{1}{nx} = 0$
\end{enumerate}

Значит выполнено для $\forall x$

\textbf{21.2} $\sum_{n=1}^{\infty} \Big( \frac{x^n}{n} - \frac{x^{n+1}}{n+1} \Big)$

$a_n(x)=x^{n}\cdot \Big( \frac{1}{n} - \frac{x}{n+1} \Big)=\frac{x^n}{n} \Big( 1-x \cdot \frac{n}{n+1} \Big)$

\begin{enumerate}
	\item $|x|>1: \frac{x^n}{n} \longrightarrow \infty, (1-x\cdot \frac{n}{n+1}) \longrightarrow 1-x \neq 0$
	\item $|x|=1: $ \begin{enumerate}
						\item $x=-1: a_n(-1)=\frac{(-1)^n}{n}\cdot \Big(1+\frac{n}{n+1}\Big) \longrightarrow 0$
						\item $x=1: a_n(1)=\frac{1}{n} \cdot \Big(1- \frac{n}{n+1}\Big) \longrightarrow 0$
					\end{enumerate}
	\item $|x|<1: a_n(x)=\frac{x^n}{n} ]\cdot \Big( 1- x \cdot \frac{n}{n+1} \Big) \longrightarrow 0 \Big( \frac{n}{n+1} \to 1; x^n \to 0; n \to \infty \Big)$
\end{enumerate}

\underline{Ответ:} $x \in [-1;1]$

\textbf{22.} $\sum_{n=1}^{\infty} \Big( \frac{n^x}{x^n} \Big):$

$a_n(x)=\frac{x^n}{x^n}$

\begin{enumerate}
	\item
	$
	\begin{cases}
		n^x - \text{степенная функция;} \\
		x^n - \text{показательная функция;}		
	\end{cases}
	$ 
	$\Longrightarrow$
	 $a_n \xrightarrow[n \to \infty]{} 0$ при $|x|>1$
	 \item $|x|=1:$
	 	\begin{enumerate}
		 	\item $x=1: \frac{n}{1} \longrightarrow \infty;$
		 	\item $x=-1: \frac{n^{-1}}{(-1)^n}=\frac{(-1)^n}{n}\longrightarrow 0$
	 	\end{enumerate}
	 \item $|x|<1:$
	       \begin{enumerate}
	       		\item при $x>0: n^x \xrightarrow[n \to \infty]{} \infty$
	       		\item при $x<0: n^x \xrightarrow[n \to \infty]{} 0$
	       \end{enumerate}
	       $x^n \xrightarrow[n \to \infty]{} 0$
\end{enumerate}
\underline{Ответ:} 

$x \in (0;1): a_n \longrightarrow \infty;$

$x \in (-1;0): a_n=\frac{n^x}{x^n}=
\Big[
	n^x \to 0;
	x^n \to 0	
\Big]=\frac{n^{-|x|}}{(-1)^n\cdot |x|^n} \longrightarrow \infty$

 (показательная ф-ия стремится к нулю быстрее степенной)



\textbf{23.} $\sum_{n=1}^{\infty} \frac{nx^n}{2^n+3^n}$

$\lim_{n \to \infty} a_n = \frac{nx^n}{2^n+3^n} \approx \left(\frac{x}{3}\right)^n \cdot n$

Показательная функция быстрее линейной, поэтому сходимость зависит от неё:
\begin{enumerate}
	\item $\left|\frac{x}{3}\right| \geq 1$, то $a_n \to \infty$
	\item $\left|\frac{x}{3}\right| < 1$, то $a_n \to 0$
\end{enumerate}

Получаем ответ, при $x \in (-3; 3)$

\textbf{24.} $\sum_{n=1}^{\infty} \frac{x^n}{n!}$

$a_n(x) = \frac{x^n}{n!} \to 0$ т.к. показательная функция медленее факториала, покажем это:

$\frac{x^n}{n!} \sim \frac{e^{n \ln x}\cdot e^n}{\sqrt{2\pi n} \cdot n^n} \leq \frac{e^{n\cdot (\ln x + 1)}}{e^{n\ln n}} \sim e^{n(\ln x - \ln n)} \to e^{-\infty} = 0$

Значит сходится для $\forall x$

\textbf{25.} $\sum_{n=1}^{\infty} {\frac{x^{n^2}}{n!}}:$

Формула Стирлинга: $n!=\sqrt{2 \pi n} \cdot \Big( \frac{n}{e} \Big)^n \cdot (1+O(1)) $

необходимое условие сходимости: $a_n=\frac{x^{n^2}}{n!} \longrightarrow 0 \Longleftrightarrow |a_n| \longrightarrow 0: |a_n|=\Big| \frac{x^{n^2}}{n!} \Big|=\frac{|x|^{n^2}}{n!}$

$|a_n|=\frac{|x|^{n^2}}{\sqrt{2 \pi n}\cdot \big( \frac{n}{e} \big)^n}\cdot \frac{\sqrt{2 \pi n}\cdot \Big( \frac{n}{e} \Big)^n}{n!}=\Big[ \frac{\sqrt{2 \pi n}\cdot \Big( \frac{n}{e} \Big)^n}{n!} \xrightarrow [n \to \infty]{} 1 \Big] \sim \frac{|x|^{n^2}}{\sqrt{2 \pi n}\cdot \big( \frac{n}{e} \big)^n}=\frac{e ^ {n^2 \ln|x|-n\ln|\frac{n}{e}|-\frac{1}{2}\ln n}}{\sqrt{2 \pi}} $ 
 
\begin{enumerate}
	\item $|x| \leq 1: |a_n| \longrightarrow \frac{e^{-\infty}}{\sqrt{2\pi}}=0$
	\item $|x| > 1: |a_n| \longrightarrow \frac{e^{+\infty}}{\sqrt{2\pi}}=+\infty$
\end{enumerate}
