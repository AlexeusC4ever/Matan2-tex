\documentclass[a4paper]{article}
\usepackage{header}

\begin{document}
\section{Семинарский лист 2}

% Task 2
\begin{problem} \ \\[3pt]
$\displaystyle \sum_{n = 1}^{\infty}\frac{1}{n\sqrt[n]{n}}, \
    a_n = \frac{1}{n\sqrt[n]{n}}$ \\[3pt]
$\displaystyle \lim_{n \to \infty}{\sqrt[n]{n}} = 1:
    \lim_{n \to \infty}{e^{\frac{1}{n} \ln{n}}} = e^0 = 1$
$\displaystyle \implies a_n \sim \frac{1}{n}$\\[3pt]
$\displaystyle \sum_{n = 1}^{\infty}\frac{1}{n}$ сходится
$\displaystyle \implies \sum_{n = 1}^{\infty}\frac{1}{n\sqrt[n]{n}}$
по признаку сравнения.
\end{problem}

% Task 15
\begin{problem} \ \\[3pt]
$\displaystyle \sum_{n = 2}^{\infty}
    \frac{2 \cdot 5 \cdot 8 \cdot \ldots \cdot (3n - 4)}{3^n \cdot n!}$ \\[3pt]
$\displaystyle \frac{a_{n + 1}}{a_n} =
\frac{(3(n + 1) - 4) \cdot 3^n \cdot n!}{3^{n + 1} \cdot (n + 1)!} =
\frac{3(n + 1) - 4}{3(n + 1)} = \frac{3n - 1}{3n + 3} =
\frac{1 - \frac{\frac{1}{3}}{n}}{1 + \frac{1}{n}} =
\left(1 - \frac{1}{3n}\right)\left(1 - \frac{1}{n} + O\left(\frac{1}{n^2}\right)\right)=\\[3pt]
=1 - \frac{1}{n} + O\left(\frac{1}{n^2}\right)
- \frac{\frac{1}{3}}{n} + \frac{\frac{1}{3}}{n^2} - O\left(\frac{1}{3n^3}\right) =
1 - \frac{\frac{4}{3}}{n} + O\left(\frac{1}{n^2}\right) \implies
\begin{cases}
    p = \frac{4}{3} \\
    \delta = 1
\end{cases} \implies
$ ряд сходится по признаку Гаусса.
\end{problem}

% Task 17
\begin{problem} \ \\[3pt]
$\displaystyle \sum_{n = 2}^{\infty}\frac{1}{n\ln{n}},
    \ f(x) = \frac{1}{x \ln{x}}$ \\[3pt]
$\displaystyle f'(x) = -\frac{\ln{x} + 1}{x^2 \ln^2{x}} = 0 \iff
    \begin{cases}
        x \neq 0 \\
        x \neq 1 \\
        x = \frac{1}{e}
    \end{cases} \implies f(x)$ монотонно убывает при $\displaystyle x > 1$ \\[3pt]
$\displaystyle f(n + t) \leq a_n \leq f((n - 1) + t)\\[3pt]$
$\displaystyle \int_{n}^{n + 1}{f(x)dx} \leq a_n \leq \int_{n - 1}^n{f(x)dx}$ \\[3pt]
$\displaystyle \int_{2}^{N + 1}{f(x)dx} \leq \sum_{n = 2}^{N}a_n \leq \int_{1}^N{f(x)dx}$ \\[3pt]
$\displaystyle \int{\frac{1}{x\ln{x}}}dx =
    \left[\begin{aligned}t &= \ln{x}\\e^t &= x \\ dx &= e^tdt\end{aligned}\right] =
    \int{\frac{e^t}{t\cdot e^t}}dt = \int{\frac{1}{t}}dt = \ln{t} + C = \ln{\ln{x}} + C$\\[3pt]
$\displaystyle \int_{2}^{N + 1}f(x)dx = \ln{\ln(N + 1)} - \ln{\ln{2}}$ \\[3pt]
$\displaystyle \int_{1}^{N}f(x)dx = \ln{\ln{N}} - 0 = \ln{\ln{N}}$ \\[3pt]
\textbf{Ответ:} $\displaystyle \ln{\ln(N + 1)} - \ln{\ln{2}} \leq \sum_2^{N} \frac{1}{n \ln{n}}\leq \ln{\ln{N}}$
\end{problem}

\problem[19]
$\displaystyle \\
    S_n = \sum_{n=1}^N \frac{1}{n^2} \text{--- сходящийся ряд} \\
    S_N \to S, N \to \infty \\
    \text{Доказать: } S_N = S - \frac{1}{N} + o(\frac{1}{N}), N \to \infty
$

Теорема Штольца: Пусть $x_n$ и $y_n$ обе сходится к нулю, причём $0 < y_n < y_{n-1} ~ \forall n$
т.е. $\lim_{n \to \infty} \frac{x_n}{y_n} = \left[\frac{0}{0}\right]$. Тогда, если
$\exists \lim_{n \to \infty} \frac{x_n - x{n-1}}{y_n - y_{n-1}} = A$, то
$\exists \lim_{n \to \infty} \frac{x_n}{y_n} = A$.

Обозначим $\displaystyle
    x_n = S - S_n \to 0,  y_n = \frac{1}{n} \to 0,
    \lim_{n \to \infty} \frac{x_n}{y_n} = \left[\frac{0}{0}\right]
$

Расмотрим $\displaystyle
    \frac{x_n - x{n-1}}{y_n - y_{n-1}} = \frac{S - S_n - (S - S_{n - 1})}{\frac{1}{n} - \frac{1}{n-1}} =
    \frac{S_{n-1} - S_n}{\frac{n - 1 - n}{n(n - 1)}} = \frac{S_n -  S_{n-1}}{\frac{1}{n(n - 1)}}  =
    \frac{1/n^2}{\frac{1}{n^2 - n}} = \frac{n^2 - n}{n^2} = \frac{1 - \frac{1}{n}}{1}
    \xrightarrow{n \to \infty} 1
$

По т. Штольца $\displaystyle \lim_{n \to \infty} \frac{x_n}{y_n} = 1$, т.е.
$\displaystyle \frac{x_n}{y_n} = 1 + o(1), x_n = y_n + o(y_n)$

$\displaystyle
    x_n = S - S_n = \frac{1}{n} + \text{o}\left(\frac{1}{n}\right) \implies S_n = S - \frac{1}{n} +
    \text{o}\left(\frac{1}{n}\right)
$, ч.т.д.


% Task 22
\begin{problem} \ \\[3pt]
$\displaystyle \sum_{n = 1}^{N} \frac{1}{n 2^n} =
    S - \frac{1}{N 2^N} + o\left(\frac{1}{N 2^N}\right)$ \\[3pt]
$\displaystyle q_n = S - \sum_{n = 1}^{N} \frac{1}{n 2^n}, \;\;\;
    \displaystyle p_n = \frac{1}{N 2^N}$ \\[3pt]
По теореме Штольца: $\displaystyle \lim_{n \to \infty} \frac{q_n}{p_n} =
\lim_{n \to \infty}\frac{q_{n} - q_{n - 1}}{p_{n} - p_{n - 1}} =
\lim_{n \to \infty} \frac{S - S_{n} - S + S_{n - 1}}
{\frac{1}{N 2^N} - \frac{1}{(N - 1) 2^{N - 1}}} =
\frac{-\frac{1}{N 2^N} \cdot N(N - 1) 2^{2N-1}}{2^{N-1}(N-1-2N)} = 1,\\[3 pt]
\frac{q_n}{p_n} = 1 + o(1) \; \Rightarrow \; q_n = p_n + o(p_n)
\Leftrightarrow S - \sum_{n = 1}^{N} \frac{1}{n 2^n} =
\frac{1}{N 2^N} + o\left(\frac{1}{N 2^N}\right)
\Leftrightarrow \sum_{n = 1}^{N} \frac{1}{n 2^n} =
S - \frac{1}{N 2^N} \pm o\left(\frac{1}{N 2^N}\right)$,\\[3pt]
что и требовалось доказать.
\end{problem}

\problem[23]
$\displaystyle \\
    S = \sum_{n=1}^\infty \sin \frac{1}{n^2}$ - представить $S$ в виде суммы ряда с
общим членом $a_n  = \text{O}\left(\frac{1}{n^3}\right)$.

$\displaystyle
    b_n = \frac{1}{n(n + 1)} \approx \frac{1}{n^2} \\
    sin \frac{1}{n^2} = u_n \approx \frac{1}{n^2}  \\
    u_n - b_n = a_n \approx ? \\
    \sum_{n=1}^\infty b_n = \sum_{n=1}^\infty \frac{1}{n(n+1)} = \sum_{n=1}^\infty \left(\frac{1}{n} -
    \frac{1}{n+1}\right) = 1 - \frac{1}{2} + \frac{1}{2} - \frac{1}{3} + \dots = 1 \\
    \sum_{n=1}^\infty a_n = \sum_{n=1}^\infty u_n - \sum_{n=1}^\infty = S - 1 \implies S = 1 +
    \sum_{n=1}^\infty a_n
$

$\displaystyle
    a_n = u_n - b_n = \sin \frac{1}{n^2} - \frac{1}{n(n + 1)} = \frac{1}{n^2} - \frac{1}{6}\frac{1}{n^6}
    - \frac{1}{n^2}\frac{1}{1 + \frac{1}{n}} \approx \frac{1}{n^2} - \frac{1}{6n^6} - \frac{1}{n^2}
    \left(1 - \frac{1}{n}\right) = -\frac{1}{6n^6} + \frac{1}{n^3} \approx \frac{1}{n^3}
$

$\displaystyle
    \implies S = 1 + \sum_{n=1}^\infty\left(\sin \frac{1}{n^2} - \frac{1}{n(n + 1)}\right)
$


\end{document}
