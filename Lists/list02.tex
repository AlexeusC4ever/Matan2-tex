%\documentclass[a4paper]{article}
\%usepackage{header}

\begin{document}
\section{Семинарский лист 2}
    % Task 2
    \begin{problem} \ \\[3pt]
        $\displaystyle \sum_{n = 1}^{\infty}\frac{1}{n\sqrt[n]{n}}, \
         a_n = \frac{1}{n\sqrt[n]{n}}$ \\[3pt]
        $\displaystyle \lim_{n \to \infty}{\sqrt[n]{n}} = 1:
         \lim_{n \to \infty}{e^{\frac{1}{n} \ln{n}}} = e^0 = 1$
        $\displaystyle \implies a_n \sim \frac{1}{n}$\\[3pt]
        $\displaystyle \sum_{n = 1}^{\infty}\frac{1}{n}$ сходится 
        $\displaystyle \implies \sum_{n = 1}^{\infty}\frac{1}{n\sqrt[n]{n}}$ 
        по признаку сравнения.
    \end{problem}

    % Task 15
    \begin{problem} \ \\[3pt]
        $\displaystyle \sum_{n = 2}^{\infty}
        \frac{2 \cdot 5 \cdot 8 \cdot \ldots \cdot (3n - 4)}{3^n \cdot n!}$ \\[3pt]
        $\displaystyle \frac{a_{n + 1}}{a_n} =
         \frac{(3(n + 1) - 4) \cdot 3^n \cdot n!}{3^{n + 1} \cdot (n + 1)!} = 
         \frac{3(n + 1) - 4}{3(n + 1)} = \frac{3n - 1}{3n + 3} =
         \frac{1 - \frac{\frac{1}{3}}{n}}{1 + \frac{1}{n}} =
         \left(1 - \frac{1}{3n}\right)\left(1 - \frac{1}{n} + O\left(\frac{1}{n^2}\right)\right)=\\[3pt]
         =1 - \frac{1}{n} + O\left(\frac{1}{n^2}\right) 
         - \frac{\frac{1}{3}}{n} + \frac{\frac{1}{3}}{n^2} - O\left(\frac{1}{3n^3}\right) =
         1 - \frac{\frac{4}{3}}{n} + O\left(\frac{1}{n^2}\right) \implies
         \begin{cases}
            p = \frac{4}{3}\\
            \delta = 1     
         \end{cases} \implies 
         $ ряд сходится по признаку Гаусса.
    \end{problem}
    
    % Task 17
    \begin{problem} \ \\[3pt]
        $\displaystyle \sum_{n = 2}^{\infty}\frac{1}{n\ln{n}},
         \ f(x) = \frac{1}{x \ln{x}}$ \\[3pt]
        $\displaystyle f'(x) = -\frac{\ln{x} + 1}{x^2 \ln^2{x}} = 0 \iff
        \begin{cases}
            x \neq 0 \\
            x \neq 1 \\
            x = \frac{1}{e}
        \end{cases} \implies f(x)$ монотонно убывает при $\displaystyle x > 1$ \\[3pt]
        $\displaystyle f(n + t) \leq a_n \leq f((n - 1) + t)\\[3pt]$
        $\displaystyle \int_{n}^{n + 1}{f(x)dx} \leq a_n \leq \int_{n - 1}^n{f(x)dx}$ \\[3pt]
        $\displaystyle \int_{2}^{N + 1}{f(x)dx} \leq \sum_{n = 2}^{N}a_n \leq \int_{1}^N{f(x)dx}$ \\[3pt]
        $\displaystyle \int{\frac{1}{x\ln{x}}}dx =
         \left[\begin{aligned}t &= \ln{x}\\e^t &= x \\ dx &= e^tdt\end{aligned}\right] =
         \int{\frac{e^t}{t\cdot e^t}}dt = \int{\frac{1}{t}}dt = \ln{t} + C = \ln{\ln{x}} + C$\\[3pt]
         $\displaystyle \int_{2}^{N + 1}f(x)dx = \ln{\ln(N + 1)} - \ln{\ln{2}}$ \\[3pt]
         $\displaystyle \int_{1}^{N}f(x)dx = \ln{\ln{N}} - 0 = \ln{\ln{N}}$ \\[3pt]
         \textbf{Ответ:} $\displaystyle \ln{\ln(N + 1)} - \ln{\ln{2}} \leq \sum_2^{N} \frac{1}{n \ln{n}}\leq \ln{\ln{N}}$
    \end{problem}


\end{document}
