\documentclass[a4paper, fleqn]{article}
\usepackage{header}

\title{Коллоквиум 1}
\author{
        % ОТКОММЕНТИРУЙ СЕБЯ
    % Александр Богданов   \\ \href{https://t.me/SphericalPotatoInVacuum}{Telegram} \and
    % Алиса Вернигор       \\ \href{https://t.me/allisyonok}{Telegram} \and
    % Анастасия Григорьева \\ \href{https://t.me/weifoll}{Telegram} \and
    % Василий Шныпко       \\ \href{https://t.me/yourvash}{Telegram} \and
    % Данил Казанцев       \\ \href{https://t.me/vserosbuybuy}{Telegram} \and
    Денис Козлов         \\ \href{https://t.me/DKozl50}{Telegram} \and
    % Елизавета Орешонок   \\ \href{https://t.me/eaoresh}{Telegram} \and
    % Иван Пешехонов       \\ \href{https://t.me/JohanDDC}{Telegram} \and
    % Иван Добросовестнов  \\ \href{https://t.me/ivankot13}{Telegram} \and
    % Настя Городилова     \\ \href{https://t.me/nastygorodi}{Telegram} \and
    Никита Насонков      \\ \href{https://t.me/nnv_nick}{Telegram} \and
    % Сергей Лоптев        \\ \href{https://t.me/beast_sl}{Telegram}
}

\date{Версия от {\ddmmyyyydate\today} \currenttime}

\begin{document}
    \maketitle
    % Решение вопроса пишем после комментария 
    % комментарии трогать пожалуйста не надо, будет круто
    
    % давайте для постоянства формулировку в сабсекшн кидать
    % доказательства советуют оборачивать в \begin{proof} \end{proof}
    
    % вопрос 1
    \subsection*{1. Дайте определения: числовой ряд, частичная сумма ряда, сумма ряда, сходящийся ряд, расходящийся ряд. Рассмотрим ряд с общим членом $a_n$. Докажите, что если ряд сходится, то $a_n \rightarrow 0$.}
    Пусть $a_n$ $-$ последовательность, т.е. $\mathbb{N} \rightarrow \mathbb{R}$. 
    Формальная бесконечная сумма $a_1 + a_2 + a_3 + ... = \sum\limits_{n = 1}^{\infty} a_n$ называется рядом. \\
    $S_N = \sum\limits_{n = 1}^N a_n$ $-$ частичная сумма. \\
    Суммой ряда называется $S = \lim\limits_{N\rightarrow \infty} S_N$. \\
    Если $\exists S \in \mathbb{R}$, то ряд называют сходящимся. \\
    Если $\exists S = \infty$ или $\nexists S$, то ряд называют расходящимся. \\ \\
    \textbf{Необходимое условие сходимости: } Если ряд сходится, то $a_n \rightarrow 0$. \\
    \textit{Доказательство: } $a_n = S_n - S_{n - 1} \rightarrow 0, $ т.к. $S_n \rightarrow S$ и $S_{n - 1} \rightarrow S$. $\blacksquare$ \\

    % вопрос 2

    % вопрос 3
        
    % вопрос 4
        
    % вопрос 5
        
    % вопрос 6
        
    % вопрос 7
        
    % вопрос 8
        
    % вопрос 9
        
    % вопрос 10
        
    % вопрос 11
    \subsection*{11. Пусть положительный ряд $\sum a_n$ сходится и $r_n$ $-$ его остаток. Докажите, что ряд $\sum(\sqrt{r_n} - \sqrt{r_{n + 1}})$ также сходится, причём медленнее, чем ряд $\sum a_{n + 1}$.}
    Сначала докажем сходимость ряда $\sum\limits_{n = 0}^N (\sqrt{r_n} - \sqrt{r_{n + 1}})$:\\
    $\sum\limits_{n = 0}^N (\sqrt{r_n} - \sqrt{r_{n + 1}}) = \sqrt{r_0} - \sqrt{r_1} + \sqrt{r_1} - \sqrt{r_2} + ... + \sqrt{r_N} - \sqrt{r_{N+1}} = \sqrt{r_0} - \sqrt{r_{N+1}} = \sqrt{S} - \sqrt{r_{N+1}} \rightarrow \sqrt{S}$ (т.к. $\sqrt{r_{N+1}} \rightarrow 0$) \\
    Теперь покажем, что он сходится медленнее, чем $a_{n+1}$: \\
    $\frac{\sqrt{r_n} - \sqrt{r_{n + 1}}}{a_{n+1}} = \frac{\sqrt{r_n} - \sqrt{r_{n + 1}}}{r_n - r_{n+1}} = \frac{1}{\sqrt{r_n} + \sqrt{r_{n + 1}}} \rightarrow \infty$, т.к. $\sqrt{r_n} \rightarrow 0$ и $\sqrt{r_{n + 1}} \rightarrow 0$. $\blacksquare$ \\

    % вопрос 12
        
    % вопрос 13
        
    % вопрос 14
        
    % вопрос 15
        
    % вопрос 16
        
    % вопрос 17
        
    % вопрос 18
        
    % вопрос 19
        
    % вопрос 20
        
    % вопрос 21
    \subsection*{21. Выведите двустороннюю оценку частичной суммы ряда через неопределённый интеграл. Сформулируйте и докажите интегральный признак Коши-Маклорена.}
    \textbf{Интегральный признак Коши-Маклорена:} Если функция $f(x)$ принимает неотрицательные значения на всей области определения и монотонно убывает, а также $\forall n \in \mathbb{N} : f(n) = a_n$, то $\sum a_n$ и $\int^{\infty} f(x)dx$ сходятся или расходятся одновременно. \\
    \textit{Доказательство:} Рассмотрим убывающую при $x \geq n_0 - 1$ функцию $f(x)$ и ряд $\sum\limits_{n=n_0}^{\infty} a_n$, где $a_n = f(n)$. Заметим, что \\
    \begin{equation*} f(n + t) \leq a_n \leq f(n - 1 + t), \; t \in [0; 1] \end{equation*}
    Проинтегрируем каждый член неравенства определённым интегралом от $0$ до $1$ по $dt$: \\
    \begin{flalign*} 
        & \int_0^1 f(n + t) dt \leq \int_0^1 a_n dt \leq \int^1_0 f(n - 1 + t) dt & \\
        & \int^{n + 1}_n f(x)dx \leq a_n \leq \int^n_{n-1} f(x)dx &
    \end{flalign*}
    Просуммируем эти неравенства при всех $n$: \\
    \begin{equation*}
        \int_{n_0}^{N+1}f(x)dx \leq \sum\limits_{n = n_0}^N a_n \leq \int^N_{n_0 - 1} f(x)dx
    \end{equation*}
    Тогда $\sum a_n$ ведёт себя как несобственный интеграл $\int^{\infty} f(x)dx$. $\blacksquare$ \\
    Двусторонняя оценка для частичной суммы ряда через определённый интеграл была выведена в процессе.

    % вопрос 22
        
    % вопрос 23
        
    % вопрос 24
        
    % вопрос 25
        
    % вопрос 26
        
    % вопрос 27
        
    % вопрос 28
        
    % вопрос 29
        
    % вопрос 30
        
    % вопрос 31
    \subsection*{31. Сформулируйте признак Лейбница для знакочередующегося ряда. Приведите пример применения признака Лейбница.}
    \textbf{Признак Лейбница:} Если ряд имеет вид $\sum_{n=1}^{\infty}(-1)^n \cdot u_n$ и $u_n$ монотонно убывает к $0$ (обозначение: $u_n \searrow 0$), то ряд сходится. \\
    \textit{Пример: } \\
    $\sum_{n=1}^{\infty} \dfrac{(-1)^{n}}{n^p}$, $p > 0$ \\
    $\dfrac{1}{n^p} \searrow 0 \implies $ ряд сходится (при $\forall p > 0$) \\
        
    % вопрос 32
        
    % вопрос 33
        
    % вопрос 34
        
    % вопрос 35
        
    % вопрос 36
        
    % вопрос 37
        
    % вопрос 38
        
    % вопрос 39
        
    % вопрос 40
        
    % вопрос 41
    \subsection*{41. Что такое произведение рядов в форме Коши? Приведите пример вычисления такого произведения.}
    \textbf{Произведение рядов в форме Коши:} Если $\left(\sum_{k=1}^{\infty} a_k\right) \cdot \left(\sum_{m=1}^{\infty} b_m \right) = \sum_{n=2}^{\infty} c_n$, то $c_i = \sum\limits_{j = 1}^{i - 1} a_j \cdot b_{i - j}, \; i \geq 2$. \\
    \textit{Пример:} $\left(\sum\limits_{k=1}^{\infty} k + 1 \right) \cdot \left( \sum\limits_{m = 1}^{\infty} m^2 \right) = \sum\limits_{n = 2}^{\infty} c_n$. Для примера посчитаем несколько первых членов $c_n$: \\
    $c_2 = a_1 \cdot b_1 = 2 \cdot 1 = 2$\\
    $c_3 = a_2 \cdot b_1 + a_1 \cdot b_2 = 3 \cdot 1 + 2 \cdot 4 = 11$ \\
    $c_4 = a_3 \cdot b_1 + a_2 \cdot b_2 + a_1 \cdot b_3 = 4 \cdot 1 + 3 \cdot 4 + 2 \cdot 9 = 34$ \\
    $\dots$ \\

    % вопрос 42
        
    % вопрос 43
        
    % вопрос 44
        
    % вопрос 45
        
    % вопрос 46
        
    % вопрос 47
        
    % вопрос 48
        
    % вопрос 49
        
    % вопрос 50
        
    % вопрос 51
    \subsection*{51. Сформулируйте определения равномерной сходимости функциональной последовательности: в терминах нормы и на языке $\varepsilon - \delta$.}
    \begin{enumerate}
        \item $f_n \overset{D}{\rightrightarrows} f \iff ||f_n - f|| \rightarrow 0$. \\
      \item $\sum f_n(x) \rightrightarrows S(x) \iff \forall \epsilon > 0, \exists N(\epsilon): \forall n \geqslant N(\epsilon), |S_n(x) - S(x)| < \epsilon$. \\
    \end{enumerate}
        
    % вопрос 52
        
    % вопрос 53
        
    % вопрос 54
        
    % вопрос 55
        
    % вопрос 56
        
    % вопрос 57
        
    % вопрос 58
        
    % вопрос 59
        
    % вопрос 60
        
    % вопрос 61
    \subsection*{61. Докажите, что предел равномерно сходящейся последовательности непрерывных функций является непрерывной функцией.}
    \textit{Доказательство:} Пусть функция $s(x)$ $-$ предел некоторой последовательности непрерывных функций $s_n(x)$. Тогда непрерывность функции $s(x)$, которую нам нужно доказать, по определению будет заключаться в том, что в любой точке $x_0$ для любого $\varepsilon > 0$ можно найти такое $\delta$, что из $|h| < \delta$ следует, что $|s(x_0 + h) - s(x_0)| < \varepsilon$. \\
    Для любых $x_0, h, n$ имеем \\
    \begin{flalign*}
        &|s(x_0 + h) - s(x_0)| = |s(x_0 + h) - s_n(x_0 + h) + s_n(x_0 + h) - s_n(x_0) + s_n(x_0) - s(x_0)|\leq &\\
        &\leq |s(x_0 + h) - s_n(x_0 + h)| + |s_n(x_0 + h) - s_n(x_0)| + |s_n(x_0) - s(x_0)|&
    \end{flalign*}
    По определению равномерной сходимости мы можем взять такое $n$, что для любого $x_0$ будет выполняться неравенство $|s(x_0) - s_n(x_0)| < \frac{\varepsilon}{3}$. Значит справедливы неравенства \\
    \begin{equation*}
        |s(x_0 + h) - s_n(x_0 + h)| < \frac{\varepsilon}{3}
    \end{equation*}
    \begin{equation*}
        |s(x_0) - s_n(x_0)| < \frac{\varepsilon}{3}
    \end{equation*}
    Итак, пусть мы зафиксировали некоторое $n$, тогда, поскольку функция $s_n(x)$ монотонна по условию, найдётся такое $\delta$, что для любого $|h| < \delta$ выполняется неравенство $|s_n(x_0 + h) - s_n(x_0)| < \frac{\varepsilon}{3}$. Таким образом \\
    \begin{equation*}
        |s(x_0 + h) - s(x_0)| \leq |s(x_0 + h) - s_n(x_0 + h)| + |s_n(x_0 + h) - s_n(x_0)| + |s_n(x_0) - s(x_0)| < \frac{\varepsilon}{3} + \frac{\varepsilon}{3} + \frac{\varepsilon}{3} = \varepsilon \; \; \blacksquare
    \end{equation*}

    % вопрос 62
        
    % вопрос 63
        
    % вопрос 64
        
    % вопрос 65
        
    % вопрос 66
        
    % вопрос 67
        
    % вопрос 68
        
    % вопрос 69
        
    % вопрос 70
        
    % вопрос 71
    \subsection*{71. Сформулируйте критерий Коши равномерной сходимости функционального ряда.}
    Функциональный ряд $\sum_{n=1}^{\infty} a_n(x)$ сходится равномерно на $D \iff$ $\forall \varepsilon > 0$ $\exists N(\varepsilon)$, $\forall n \geq N$, $\forall m$: 
    $$||a_n + a_{n + 1} + \dots + a_{n + m}|| < \varepsilon$$
            
    Т.е. $|a_n(x) + a_{n + 1}(x) + \dots + a_{n + m}(x)| < \varepsilon$ $\forall x \in D$.

    % вопрос 72
        
    % вопрос 73
        
    % вопрос 74
        
    % вопрос 75
        
    % вопрос 76
        
    % вопрос 77
        
    % вопрос 78
        
    % вопрос 79
        
    % вопрос 80
        
    % вопрос 81
    \subsection*{81. Сформулируйте теорему о почленном дифференцировании функционального ряда.}
    $-\infty \leq a < b \leq +\infty$, $D= (a; b)$, $D = [a; b]$ \\
    Пусть $c_n(x)$ дифференцируемы на $D$ и $\sum_{n=1}^{\infty} c'_n(x)$ сходится равномерно на $D$. \\
    Тогда ряд $\sum_{n=1}^{\infty} c_n(x)$ сходится на $D$ (а если $D$ огр, то сходится равномерно), а его сумма будет дифференцируемой функцией на $D$ и $\left(\sum_{n=1}^{\infty} c_n(x)\right)' = \sum_{n=1}^{\infty} c'_n(x)$ \\

    % вопрос 82
        
    % вопрос 83
        
    % вопрос 84
        
    % вопрос 85
        
    % вопрос 86
        
    % вопрос 87
        
    % вопрос 88
        
    % вопрос 89
        
    % вопрос 90
        
    % вопрос 91
        
    % вопрос 92
        
    % вопрос 93
        
    % вопрос 94
        
    % вопрос 95
        
    % вопрос 96
        
    % вопрос 97
        
    % вопрос 98
        
    % вопрос 99
        
    % вопрос 100
        
    % вопрос 101
        
    % вопрос 102

\end{document}
