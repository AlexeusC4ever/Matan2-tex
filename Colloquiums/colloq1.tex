\documentclass[a4paper, fleqn]{article}
\usepackage{header}

\title{Коллоквиум 1}
\author{
        % ОТКОММЕНТИРУЙ СЕБЯ
    % Александр Богданов   \\ \href{https://t.me/SphericalPotatoInVacuum}{Telegram} \and
    Алиса Вернигор       \\ \href{https://t.me/allisyonok}{Telegram} \and
    % Анастасия Григорьева \\ \href{https://t.me/weifoll}{Telegram} \and
    % Василий Шныпко       \\ \href{https://t.me/yourvash}{Telegram} \and
    % Данил Казанцев       \\ \href{https://t.me/vserosbuybuy}{Telegram} \and
    Денис Козлов         \\ \href{https://t.me/DKozl50}{Telegram} \and
    % Елизавета Орешонок   \\ \href{https://t.me/eaoresh}{Telegram} \and
    % Иван Пешехонов       \\ \href{https://t.me/JohanDDC}{Telegram} \and
    % Иван Добросовестнов  \\ \href{https://t.me/ivankot13}{Telegram} \and
    % Настя Городилова     \\ \href{https://t.me/nastygorodi}{Telegram} \and
    % Никита Насонков      \\ \href{https://t.me/nnv_nick}{Telegram} \and
    % Сергей Лоптев        \\ \href{https://t.me/beast_sl}{Telegram}
}

\date{Версия от {\ddmmyyyydate\today} \currenttime}

\begin{document}
    \maketitle
    % Решение вопроса пишем после комментария 
    % комментарии трогать пожалуйста не надо, будет круто
    
    % давайте для постоянства формулировку в сабсекшн кидать
    % доказательства советуют оборачивать в \begin{proof} \end{proof}
    
    % вопрос 1

    % вопрос 2

    % вопрос 3
        
    % вопрос 4
        
    % вопрос 5
        
    % вопрос 6
        
    % вопрос 7
        
    % вопрос 8
        
    % вопрос 9
    \subsection*{9. Пусть $ \Sigma a_n, \Sigma a_n'$ -- сходящиеся положительные
    ряды. Говорят, что ряд $  \Sigma a_n'$ сходится быстрее ряда 
    $ \Sigma a_n$, если $  a_n' = o(a_n)$. 
    Докажите, что в этом случае также $  r_n' = o(r_n)$, где
    $ r_n$, $  r_n'$ -- остатки соответствующих рядов.}
    \begin{proof} \ \\ 
        $\displaystyle a_n' = o(a_n)$, то есть, $\displaystyle \frac{a_n'}{a_n} \to 0$ \\
        $\displaystyle r_n = \sum_{k = n + 1}^{\infty} \Rightarrow r_{n - 1} = \sum_{k = n}^{\infty}$ \\ 
        Так как ряды положительные и сходятся,
         $\displaystyle r_n, r_n' \to 0$, $\displaystyle r_n \downarrow \Rightarrow$ можем применить
         теорему Штольца \\ 
         $\displaystyle \lim_{n \to \infty}{\frac{r_n'}{r_n}} = 
         \lim_{n \to \infty}{\frac{r_n' - r_{n - 1}'}{r_n - r_{n - 1}}} = 
         \lim_{n \to \infty}{\frac{a_n'}{a_n}} = 0 \Rightarrow r_n' = o(r_n)$
    \end{proof}
        
    % вопрос 10
        
    % вопрос 11
        
    % вопрос 12
        
    % вопрос 13
        
    % вопрос 14
        
    % вопрос 15
        
    % вопрос 16
        
    % вопрос 17
        
    % вопрос 18
        
    % вопрос 19
    \subsection*{19. Сформулируйте признак Гаусса для положительного ряда. Приведите
    пример применения признака Гаусса.}
    \proposition[Признак Гаусса]{ \ \\
        Пусть $\displaystyle \exists \delta > 0, p: \frac{a_{n + 1}}{a_n} = 
        1 - \frac{p}{n} + O\left(\frac{1}{n^{1+\delta}}\right)$ \\
        Тогда: \\
        если $\displaystyle p > 1 \Rightarrow \sum a_n$ -- сходится
        если $\displaystyle p \leqslant 1 \Rightarrow \sum a_n$ -- расходится
    }
    \example{  $\displaystyle \sum_{n = 2}^{\infty}
    \frac{2 \cdot 5 \cdot 8 \cdot \ldots \cdot (3n - 4)}{3^n \cdot n!}$ \\
    $\displaystyle \frac{a_{n + 1}}{a_n} =
     \frac{(3(n + 1) - 4) \cdot 3^n \cdot n!}{3^{n + 1} \cdot (n + 1)!} = 
     \frac{3(n + 1) - 4}{3(n + 1)} = \frac{3n - 1}{3n + 3} =
     \frac{1 - \frac{\frac{1}{3}}{n}}{1 + \frac{1}{n}} =
     \left(1 - \frac{1}{3n}\right)\left(1 - \frac{1}{n} + O\left(\frac{1}{n^2}\right)\right)=\\
     =1 - \frac{1}{n} + O\left(\frac{1}{n^2}\right) 
     - \frac{\frac{1}{3}}{n} + \frac{\frac{1}{3}}{n^2} - O\left(\frac{1}{3n^3}\right) =
     1 - \frac{\frac{4}{3}}{n} + O\left(\frac{1}{n^2}\right) \implies
     \begin{cases}
        p = \frac{4}{3}\\
        \delta = 1     
     \end{cases} \implies 
     $ ряд сходится по признаку Гаусса.}
     \example{ $\displaystyle
     S = \sum\limits_{n=1}^{\infty}\left(\frac{(2n - 1)!!}{(2n)!!}\right)^2$\\[3pt]
     Применим признак Гаусса:
     $\displaystyle
     \frac{a_{n+1}}{a_n} = \left(\frac{(2n + 1)!!}{(2n + 2)!!}\right)^2\cdot\left(\frac{(2n)!!}{(2n - 1)!!}\right)^2 =
     \left(\frac{(2n + 1)!!}{(2n + 2)!!}\cdot\frac{(2n)!!}{(2n - 1)!!}\right)^2 = \left(\frac{2n + 1}{2n+2}\right)^2 =
     \left(\frac{2 + \frac{1}{n}}{2+\frac{2}{n}}\right)^2
     =\frac{4 + \frac{4}{n} + O\left(\frac{1}{n^2}\right)}{4 + \frac{8}{n} + O\left(\frac{1}{n^2}\right)}=\\[3pt]
     =\frac{1 + \frac{1}{n} + O\left(\frac{1}{n^2}\right)}{1 + \frac{2}{n} + O\left(\frac{1}{n^2}\right)}
     =\left(1 + \frac{1}{n} + O\left(\frac{1}{n^2}\right)\right)\left(1 - \frac{2}{n} + O\left(\frac{1}{n^2}\right)\right)
     =1 - \frac{1}{n} + O\left(\frac{1}{n^2}\right) \implies\\
     \implies
     \begin{cases}
         \delta = 1 \\
         p = 1 & = 1
     \end{cases}
     \implies
     $ ряд расходится по признаку Гаусса.}
    % вопрос 20
        
    % вопрос 21
        
    % вопрос 22
        
    % вопрос 23
        
    % вопрос 24
        
    % вопрос 25
        
    % вопрос 26
        
    % вопрос 27
        
    % вопрос 28
        
    % вопрос 29
    \subsection*{29. Приведите пример приведения преобразования знакопеременного
    (но не знакочередующегося) ряда к знакочередующемуся.}
    \example{
        $\displaystyle \sum_{n = 1}^{\infty}\frac{(-1)^{[\ln{n}]}}{n}$\\
        $\displaystyle (-1)^k:$ \\
        $\displaystyle k \leqslant \ln{n} < k + 1$ \\
        $\displaystyle e^k \leqslant n < e^{k + 1}$ \\
        $\displaystyle A_k = (-1)^k \sum_{n = [e^k] + 1}^{[e^{k + 1}]}\frac{1}{n}$ \\
        $\displaystyle |A_k| \geqslant \frac{1}{e^{k + 1}}([e^{k + 1}] - ([e^k] + 1)) 
        \geqslant \frac{1}{e^{k + 1}}(2[e^{k}] - [e^k] - 1) =  \frac{[e^k] - 2}{e^{k + 1}}
        > \frac{e^k - 2}{e^{k + 1}} \xrightarrow[k \to \infty]{} \frac{1}{e} \neq 0$\\
        $\displaystyle \Rightarrow \sum A_k$ -- расходится (не выполняется необходимое условие
        сходимости ряда) $\displaystyle \Rightarrow \sum a_n$ -- расходится   
    }
    % вопрос 30
        
    % вопрос 31
        
    % вопрос 32
        
    % вопрос 33
        
    % вопрос 34
        
    % вопрос 35
        
    % вопрос 36
        
    % вопрос 37
        
    % вопрос 38
        
    % вопрос 39
    \subsection*{39. Приведите пример условно сходящегося ряда и перестановки, меняющей его сумму
    (с обоснованием).}
    \example{
        $\displaystyle \sum_{n = 1}^{\infty} \frac{(-1)^n}{n} = -1 + \frac{1}{2} -
         \frac{1}{3} + \frac{1}{4} - \ldots = -\ln{2}$\\
        $\displaystyle S_{2n}^{+} = \frac{1}{2} + \frac{1}{4} + \ldots + \frac{1}{2n} =
         \frac{1}{2}(\ln{n} + \gamma) + o(1)$\\ 
        $\displaystyle S_{2n}^{-} = 1 + \frac{1}{3} + \ldots + \frac{1}{2n - 1} = 
        \ln{2} + \frac{1}{2}(\ln{n} + \gamma) + o(1)$\\
        Пусть берётся $\displaystyle p$ положительных слагаемых, затем $\displaystyle q$ 
        отрицательных и так далее. \\ 
        Тогда после $\displaystyle m$ действий получим: \\
        $\displaystyle S_{2mp}^+ = \frac{1}{2} + \frac{1}{4} + \ldots + \frac{1}{2mp}=
        \frac{1}{2}(\ln{(mp)} + \gamma) + o(1)$ \\
        $\displaystyle S_{2mq - 1}^- = 1 + \frac{1}{3} + \ldots + \frac{1}{2mq - 1}=
        \ln{2} + \frac{1}{2}(\ln{(mq)} + \gamma) + o(1)$ \\
        $\displaystyle S_{2mp}^{+} - S_{2mq}^{-} = -ln{2} + \frac{1}{2}\ln{\left(\frac{p}{q}\right)} +
        o(1)$ \\
        $\displaystyle \Rightarrow$ ряд сходится к числу $\displaystyle -\ln\left(2\sqrt{\frac{q}{p}}\right)$ 
    }
    % вопрос 40
        
    % вопрос 41
        
    % вопрос 42
        
    % вопрос 43
        
    % вопрос 44
        
    % вопрос 45
        
    % вопрос 46
        
    % вопрос 47
        
    % вопрос 48
        
    % вопрос 49
    \subsection*{49. Дайте определения: функциональная последовательность, точка сходимости функциональной последовательности, область (множество) сходимости функциональной последовательности, поточечная сходимость функциональной последовательности на данном множестве.}
    \definition{
        Функциональным рядом (последовательностью) называется такой ряд (последовательность), что его элементами являются не числа, а функции $f_n(x)$.
    }
   
    \definition{
           Пусть $\forall n, n \in \mathbb{N}, f_n: D \rightarrow \mathbb{R}, D \subseteq \mathbb{R}$.
           Говорят, что $a \in D$ - точка сходимости $\{f_n(x)\}$, если последовательность $\{f_n(a)\}$ сходится.
    }
   \definition{
        Множество всех точек сходимости называется множеством сходимости.
    }
   \definition{
        Говорят, что последовательность сходится на $D$ поточечно, если $D$ – множество сходимости.
   }
    % вопрос 50
        
    % вопрос 51
        
    % вопрос 52
        
    % вопрос 53
        
    % вопрос 54
        
    % вопрос 55
        
    % вопрос 56
        
    % вопрос 57
        
    % вопрос 58
        
    % вопрос 59
    \subsection*{59. Пусть $\displaystyle \varphi: G \to D$ -- биекция. Докажите, что
    равномерная сходимость функциональной последовательности $\displaystyle \{f_n\}$
    на множество $\displaystyle D$ равносильна равномерное сходимости на функциональной
    последовательности $\displaystyle \{f_n \circ \varphi\}$ на множестве $G$.
    }
    \begin{proof} \ \\
        $\displaystyle X \in D, f_n(x)$\\
        $ t \in G, \varphi(t) \in D$ \\
        $\displaystyle (f_n \circ \varphi)(t) = f_n(\varphi(t))$ \\
        Знаем, что $\displaystyle f_n \overset{D}{\rightrightarrows} f$ \\
        Хотим доказать: $\displaystyle f_n \circ \varphi \overset{G}{\rightrightarrows}
        f \circ \varphi$ \\[9.5pt]
        $\displaystyle ||f_n \circ \varphi - f \circ \varphi|| =
        \underset{t \in G}{sup} |f_n(\varphi(t)) - f(\varphi(t))| = M_n$ \\ 
        Что означает, что супремум равен $\displaystyle M_n$? Это означает, что:
        \begin{itemize}
            \item[1)] $\displaystyle |f_n(\varphi(t)) - f(\varphi(t))| \leqslant M_n, \forall t$
            \item[2)] $\displaystyle \exists \{t_k\}: |f_n(\varphi(t_k)) - f(\varphi(t_k))|
            \xrightarrow[k \to \infty]{} M_n$ 
        \end{itemize}  
        Что получаем?
        \begin{itemize}
            \item[1)] $\displaystyle \Leftrightarrow \forall x \in D \
            |f_n(x) - f(x)| \leqslant M_n$
            \item[2)] $\displaystyle \Leftrightarrow \exists \{x_k\}: 
            |f_n(x_k) - f(x_k)| \xrightarrow[k \to \infty]{} M_n$, где $\displaystyle x_k =
            \varphi(t_k)$
        \end{itemize}
        $\displaystyle \Rightarrow M_n =
        \underset{x \in D}{sup} |f_n(x) - f(x)|$ \\ 
        $\displaystyle \Rightarrow ||f_n \circ \varphi - f \circ \varphi||_G =
        ||f_n - f||_D$\\
        Получается, что если одна норма равна 0, то и вторая норма будет равна 0. А так как везде
        знаки равносильности, то доказали мы сразу в две стороны.
    \end{proof}
    % вопрос 60
        
    % вопрос 61
        
    % вопрос 62
        
    % вопрос 63
        
    % вопрос 64
        
    % вопрос 65
        
    % вопрос 66
        
    % вопрос 67
        
    % вопрос 68
        
    % вопрос 69
        
    % вопрос 70
        
    % вопрос 71
        
    % вопрос 72
        
    % вопрос 73
        
    % вопрос 74
        
    % вопрос 75
        
    % вопрос 76
        
    % вопрос 77
        
    % вопрос 78
        
    % вопрос 79
        
    % вопрос 80
        
    % вопрос 81
        
    % вопрос 82
        
    % вопрос 83
        
    % вопрос 84
        
    % вопрос 85
        
    % вопрос 86
        
    % вопрос 87
        
    % вопрос 88
        
    % вопрос 89
    
    % вопрос 90
        
    % вопрос 91
        
    % вопрос 92
        
    % вопрос 93
        
    % вопрос 94
        
    % вопрос 95
        
    % вопрос 96
        
    % вопрос 97
        
    % вопрос 98
        
    % вопрос 99
    \subsection*{99. Приведите пример бесконечной дифференцируемой функции, 
    не являющейся аналитической.}
    \example{
        $\displaystyle f(x) = \begin{cases}
            e^{-\frac{1}{x^2}}, x \neq 0 \\ 
            0, x = 0
        \end{cases}$ \\
        Такая функция бесконечно дифференцируема, но все её производные
        в нуле равны 0: \\ $\displaystyle f'(0) = f''(0) = f'''(0) = \ldots = 0$ \\
        Получается, что её ряд Тейлора при $\displaystyle x_0 = 0: 
        0 + 0x + 0x^2 + \ldots = 0$\\ 
        То есть, такая функция не является аналитической.
    }    
    % вопрос 100
        
    % вопрос 101
        
    % вопрос 102

\end{document}
