\documentclass[a4paper, fleqn]{article}

\usepackage{header}

\title{Коллоквиум 1}
\author{
        % ОТКОММЕНТИРУЙ СЕБЯ
    % Александр Богданов   \\ \href{https://t.me/SphericalPotatoInVacuum}{Telegram} \and
    Алиса Вернигор       \\ \href{https://t.me/allisyonok}{Telegram} \and
    % Анастасия Григорьева \\ \href{https://t.me/weifoll}{Telegram} \and
    Василий Шныпко       \\ \href{https://t.me/yourvash}{Telegram} \and
    % Данил Казанцев       \\ \href{https://t.me/vserosbuybuy}{Telegram} \and
    Денис Козлов         \\ \href{https://t.me/DKozl50}{Telegram} \and
    Елизавета Орешонок   \\ \href{https://t.me/eaoresh}{Telegram} \and
    % Иван Пешехонов       \\ \href{https://t.me/JohanDDC}{Telegram} \and
    % Иван Добросовестнов  \\ \href{https://t.me/ivankot13}{Telegram} \and
    % Настя Городилова     \\ \href{https://t.me/nastygorodi}{Telegram} \and
    Никита Насонков      \\ \href{https://t.me/nnv_nick}{Telegram} \and
    Даниэль Хайбулин      \\ \href{https://t.me/kiDaniel}{Telegram} \and
    Сергей Лоптев        \\ \href{https://t.me/beast_sl}{Telegram}
}

\date{Версия от {\ddmmyyyydate\today} \currenttime}

\begin{document}
    \maketitle
    % Решение вопроса пишем после комментария 
    % комментарии трогать пожалуйста не надо, будет круто
    
    % давайте для постоянства формулировку в сабсекшн кидать
    % доказательства советуют оборачивать в \begin{proof} \end{proof}
    
    % вопрос 1
    \subsection*{1. Дайте определения: числовой ряд, частичная сумма ряда, сумма ряда, сходящийся ряд, расходящийся ряд. Рассмотрим ряд с общим членом $a_n$. Докажите, что если ряд сходится, то $a_n \rightarrow 0$.}
    Пусть $a_n$ $-$ последовательность, т.е. $\mathbb{N} \rightarrow \mathbb{R}$. 
    Формальная бесконечная сумма $a_1 + a_2 + a_3 + ... = \sum\limits_{n = 1}^{\infty} a_n$ называется рядом. \\
    $S_N = \sum\limits_{n = 1}^N a_n$ $-$ частичная сумма. \\
    Суммой ряда называется $S = \lim\limits_{N\rightarrow \infty} S_N$. \\
    Если $\exists S \in \mathbb{R}$, то ряд называют сходящимся. \\
    Если $\exists S = \infty$ или $\nexists S$, то ряд называют расходящимся. \\ \\
    \textbf{Необходимое условие сходимости: } Если ряд сходится, то $a_n \rightarrow 0$. \\
    \textit{Доказательство: } $a_n = S_n - S_{n - 1} \rightarrow 0, $ т.к. $S_n \rightarrow S$ и $S_{n - 1} \rightarrow S$. $\blacksquare$ \\

    % вопрос 2
        \subsection*{2. Критерий Коши сходимости числовой последовательности}
        Последовательность $\{x_n\}$ сходится, если $\forall \varepsilon > 0\ \exists N = N(\varepsilon) : \forall n, m \geqslant N \implies |x_n - x_m| < \varepsilon$\\
        Критерий Коши сходимости числового ряда\\
        Ряд $\sum x_n$ сходится тогда, и только тогда, когда $\forall\varepsilon > 0 \exists N = N(\varepsilon) : \forall n, m \geqslant N,\ n > m \implies |\sum\limits_{i = m}^n x_i| < \varepsilon$
\begin{proof}
        Рассмотрим последовательность частных сумм $\{S_n\}$\\
        Ряд сходится тогда, и только тогда, когда сходится $\{S_n\}$\\
        То есть ряд $\sum x_n$ сходится тогда, и только тогда, когда $\forall \varepsilon > 0 \exists N = N(\varepsilon) : \forall n, m \geqslant N,\ n > m \implies |S_n - S_m| < \varepsilon \implies |\sum\limits_{i = m}^n x_i| < \varepsilon$
\end{proof}

    % вопрос 3
    \subsection*{3. Сформулируйте и докажите признак сравнения положительных числовых рядов, основанный на неравенстве $a_n \leqslant b_n$.}

    Пусть $\exists \, n_0 \in \NN : \; \forall n \geqslant n_0 \;\;\; 0 \leqslant a_n \leqslant b_n$.

    \begin{itemize}
        \item Если сходится ряд $\sum b_n$, то сходится и ряд $\sum a_n$.
        \item Если расходится ряд $\sum a_n$, то расходится и ряд $\sum b_n$.
    \end{itemize}

    \begin{proof}
        Для начала удалим из обеих последовательностей первые $n_0$ элементов, чтобы неравенство $0 \leqslant a_n \leqslant b_n$ выполнялось для всех 
    $n \in \NN$. Имеем на это право, так как конечное число элементов последовательности не влияет на ее поведение.
        \\[6 pt]
        Заметим, что последовательности частичных сумм $A_n = \sum \limits_{n = 1}^N a_n$ и $B_n = \sum \limits_{n = 1}^N b_n$ обе монотонны, так как
    ряды положительны. Также ряд $\sum b_n$ сходится, следовательно последовательность $B_n$ ограничена сверху. Тогда ограничена сверху и
    $A_n \leqslant B_n$, а из ее монотонности последовательность частичных сумм ряда $\sum a_n$ сходится $\implies$ ряд $\sum a_n$ сходится.
        \\[6 pt]
        Второе утверждение выполняется как контрапозиция первого.
    \end{proof}
    
    % вопрос 4
    \subsection*{4. Сформулируйте и докажите признак сравнения положительных числовых рядов, основанный на неравенстве $\frac{a_{n+1}}{a_n} \leq \frac{b_{n+1}}{b_n}$.}
    \begin{proposition}[Сравнение отношений]
        Пусть $\frac{a_{n+1}}{a_n} \leq \frac{b_{n+1}}{b_n}$ при $n \geq n_0$. Тогда:
        \begin{flalign*}
            & \sum b_n \text{ сходится } \implies \sum a_n \text{ сходится }\\
            & \sum a_n \text{ расходится } \implies \sum b_n \text{ расходится }
        \end{flalign*}
    \end{proposition}
    \begin{proof} 
        Предполагаем, что $a_n > 0, b_n > 0$.
        \begin{flalign*}
            & a_{n_0 + 1} \leq \frac{a_{n_0}}{b_{n_0}} \cdot b_{n_0 + 1} \\
            & a_{n_0 + 2} \leq \frac{a_{n_0 + 1}}{b_{n_0 + 1}} \cdot b_{n_0 + 2} \leq \frac{a_{n_0}}{b_{n_0}} \cdot b_{n_0 + 2} \\
            & \cdots \\
            & a_{n_0 + k} \leq \frac{a_{n_0}}{b_{n_0}} \cdot b_{n_0 + k} \\
            & \sum_{n=n_0}^N a_n \leq \frac{a_{n_0}}{b_{n_0}} \cdot \sum_{n=n_0}^N b_n
        \end{flalign*}
    \end{proof}
    
    % вопрос 5
    \subsection*{5}
    \textbf{ Сформулировать и доказать признак сравнения числовых рядов, основанный на пределе $\lim\dfrac{a_n}{b_n}$.} \\[5 pt]
    Пусть $\sum a_n$ и $\sum b_n$ --- положительные ряды, и $\exists \lim\limits_{n \to \infty} \dfrac{a_n}{b_n} \in (0;\, +\infty)$. \\[3 pt]
    Тогда ряд $\sum a_n$ сходится $\Leftrightarrow$ $\sum b_n$ сходится. \\
    \begin{proof}
    $c = \lim\limits_{n \to \infty} \dfrac{a_n}{b_n} > 0$ \\[3 pt]
    По определению предела: \\[3 pt]
    $\forall \varepsilon > 0 \; \exists n_0 : c - \varepsilon \le \dfrac{a_n}{b_n} \le c + \varepsilon \;\; \forall n \ge n_0 \; 
    \Rightarrow \; (c - \varepsilon) b_n \le a_n \le (c + \varepsilon) b_n \; \Rightarrow \\[3 pt]
    \Rightarrow \; \sum (c - \varepsilon) b_n  \le \sum a_n \le \sum (c + \varepsilon) b_n \;
    \Leftrightarrow \; C_1 \sum b_n  \le \sum a_n \le C_2 \sum b_n $
    \end{proof}    
    % вопрос 6
        
    % вопрос 7
    \subsection*{7. Сформулируйте и докажите признак Лобачевского-Коши}

    \begin{proposition}{1}
        Пусть $a_n \downarrow$. Рассмотрим ряды:

        \begin{flalign*}
            \sum a_n ~ (1) \text{ и } \sum 2^n \cdot a_{2^n} ~ (2)
        \end{flalign*}

        Тогда ряды (1) и (2) ведут себя одинаково
    \end{proposition}

    \begin{proof}
        \begin{flalign*}
            &2^m \text{ слаг.: } a_1 + \underbrace{a_2}_{\substack{\leq a_1 \\ \geq a_2}} +
            \underbrace{a_3 + a_4}_{\substack{\leq 2a_2 \\ \geq 2a_4}} +
            \underbrace{a_5 + a_6 + a_7 + a_8}_{\substack{\leq 4a_4 \\ \geq 4a_8}} + \dots +
            \underbrace{a_{2^{m-1}+1} + a_{2^{m-1}+2} + \dots + a_{2^{m-1} +
                        2^{m-1}}}_{\substack{\leq 2^{m-1} \cdot a_{2^{m-1}} \\ \geq \frac{1}{2} \cdot 2^{m} \cdot a_{2^{m}}}}
            \\
            &a_1 + \frac{1}{2} \sum_{n = 1}^{m} 2^na_{2^n} \leq \sum_{n=1}^{2^m}a_n \leq a_1 + \sum_{n=0}^{m-1}2^n \cdot a_{2^n}
            \\
            &\text{Левая часть -- сумма нижних оценок, правая -- сумма верхних}
        \end{flalign*}
    \end{proof}
        
    % вопрос 8
        
    % вопрос 9
    \subsection*{9. Пусть $ \Sigma a_n, \Sigma a_n'$ -- сходящиеся положительные
    ряды. Говорят, что ряд $  \Sigma a_n'$ сходится быстрее ряда 
    $ \Sigma a_n$, если $  a_n' = o(a_n)$. 
    Докажите, что в этом случае также $  r_n' = o(r_n)$, где
    $ r_n$, $  r_n'$ -- остатки соответствующих рядов.}
    \begin{proof} \ \\ 
        $\displaystyle a_n' = o(a_n)$, то есть, $\displaystyle \frac{a_n'}{a_n} \to 0$ \\
        $\displaystyle r_n = \sum_{k = n + 1}^{\infty} \Rightarrow r_{n - 1} = \sum_{k = n}^{\infty}$ \\ 
        Так как ряды положительные и сходятся,
         $\displaystyle r_n, r_n' \to 0$, $\displaystyle r_n \downarrow \Rightarrow$ можем применить
         теорему Штольца \\ 
         $\displaystyle \lim_{n \to \infty}{\frac{r_n'}{r_n}} = 
         \lim_{n \to \infty}{\frac{r_n' - r_{n - 1}'}{r_n - r_{n - 1}}} = 
         \lim_{n \to \infty}{\frac{a_n'}{a_n}} = 0 \Rightarrow r_n' = o(r_n)$
    \end{proof}
        
    % вопрос 10
        \subsection*{10. Пусть $\sum a_n$, $\sum a_n'$ -- сходящиеся положительные ряды. Говорят, что ряд $\sum a_n'$ сходится медленнее чем ряд $a_n$, если $a_n'=o(a_n)$. Докажите, что в этом случае также $S_n'=o(S_n)$, где $S_n,  S_n'$ -- частичные суммы соответствующих рядов.}
        Докажем при помощи теоремы Штольца. У нас даны две расходящиеся последовательности, для которых последовательности частичных сумм положительны и строго возрастают. Рассмотрим предел отношений частичных сумм $S_n$ и $S_n'$:
        $\lim \cfrac{S_n'}{S_n}=\lim \cfrac{S_n'-S'_{n-1}}{S_n-S_{n-1}}=\lim \cfrac{a_n'}{a_n}=
        0$, так как $a_n'=o(a_n)$
        
        Показали, что $S'_n=o(S_n)$
    % вопрос 11
    \subsection*{11. Пусть положительный ряд $\sum a_n$ сходится и $r_n$ $-$ его остаток. Докажите, что ряд $\sum(\sqrt{r_n} - \sqrt{r_{n + 1}})$ также сходится, причём медленнее, чем ряд $\sum a_{n + 1}$.}
    Сначала докажем сходимость ряда $\sum\limits_{n = 0}^N (\sqrt{r_n} - \sqrt{r_{n + 1}})$:\\
    $\sum\limits_{n = 0}^N (\sqrt{r_n} - \sqrt{r_{n + 1}}) = \sqrt{r_0} - \sqrt{r_1} + \sqrt{r_1} - \sqrt{r_2} + ... + \sqrt{r_N} - \sqrt{r_{N+1}} = \sqrt{r_0} - \sqrt{r_{N+1}} = \sqrt{S} - \sqrt{r_{N+1}} \rightarrow \sqrt{S}$ (т.к. $\sqrt{r_{N+1}} \rightarrow 0$) \\
    Теперь покажем, что он сходится медленнее, чем $a_{n+1}$: \\
    $\frac{\sqrt{r_n} - \sqrt{r_{n + 1}}}{a_{n+1}} = \frac{\sqrt{r_n} - \sqrt{r_{n + 1}}}{r_n - r_{n+1}} = \frac{1}{\sqrt{r_n} + \sqrt{r_{n + 1}}} \rightarrow \infty$, т.к. $\sqrt{r_n} \rightarrow 0$ и $\sqrt{r_{n + 1}} \rightarrow 0$. $\blacksquare$ \\
    
    % вопрос 12
        \subsection*{12. Пусть положительный ряд $\sum a_n$ расходится и $S_n$ его частичная сумма. Докажите, что ряд $\sum (\sqrt{S_{n+1}} - \sqrt{S_n})$ также расходится, причём медленнее, чем ряд $\sum a_{n+1}$}

        Докажем расходимость:
        \begin{align*}
            \sum_{n=0}^{N} (\sqrt{S_{n+1}} - \sqrt{S_n}) 
            &= \sqrt{S_1} - \sqrt{S_0} + \sqrt{S_2} - \sqrt{S_1} + \dots + \sqrt{S_{N+1}} - \sqrt{S_{N}} \\
            &= \sqrt{S_{N+1}} - \sqrt{S_0} \\
            &= \sqrt{S_{N+1}} \to \sqrt{S}.
        \end{align*}

        Перейдем ко второй части вопроса:
        \begin{align*}
            \dfrac{\sqrt{S_{n+1}} - \sqrt{S_n}}{a_{n+1}} 
            &= \dfrac{\sqrt{S_{n+1}} - \sqrt{S_n}}{S_{n+1} - S_n} \\
            &= \dfrac{1}{\sqrt{S_{n+1}} + \sqrt{S_n}},
        \end{align*}
        где $\sqrt{S_{n+1}} + \sqrt{S_n} \to \infty$. Это значит, что $\dfrac{1}{\sqrt{S_{n+1}} + \sqrt{S_n}}$ стремится к $0$. Тогда ряд $\sum (\sqrt{S_{n+1}} - \sqrt{S_n})$ расходится, причём медленнее, чем ряд $\sum a_{n+1}$.

    % вопрос 13
    \subsection*{13. Сформулируйте (предельный) признак Даламбера для положительного ряда.}

    Пусть $\sum a_n$ --- положительный ряд. Тогда

    \begin{itemize}
        \item $\overline{\lim} \; \dfrac{a_{n+1}}{a_n} < 1 \implies$ ряд $\sum a_n$ сходится;
        \item $\underline{\lim} \; \dfrac{a_{n+1}}{a_n} > 1 \implies$ ряд $\sum a_n$ расходится.
    \end{itemize}
        
    % вопрос 14
    \subsection*{14. Сформулируйте (предельный) радикальный признак Коши для положительного ряда.}
    \begin{proposition}[Радикальный признак Коши.]
        Пусть $a_n \geq 0$. Тогда:
        \begin{equation*}
            \varlimsup \sqrt[n]{a_n} = \begin{dcases}
                < 1 &\implies \text{ ряд } \sum a_n \text{ сход. } \\
                > 1 &\implies \text{ ряд } \sum a_n \text{ расх. }
            \end{dcases}
        \end{equation*}
    \end{proposition}

    % вопрос 15
    \subsection*{15}
        \textbf{ Доказать, что всякий раз, когда признак Даламбера даёт ответ на вопрос о сходимости или расходимости ряда, 
    радикальный признак Коши также даёт (тот же) ответ на этот вопрос.} \\[5 pt]
    Пусть $a_n > 0$. Тогда 
    $\;\;\; \left\{\begin{array}{lllllll}
    \varlimsup \, \dfrac{a_{n+1}}{a_n} & < & 1 & \Rightarrow & \varlimsup \sqrt[n]{a_n} & < & 1,\\[5 pt]
    \varliminf \, \dfrac{a_{n+1}}{a_n} & > & 1 & \Rightarrow & \varlimsup \sqrt[n]{a_n} & > & 1,\\[5 pt]
    \end{array}\right.$. \\
    \begin{proof}
    Для доказательства основного утверждения докажем неравенство: \\[3 pt]
    $\varliminf \, \dfrac{a_{n+1}}{a_n} \le \varliminf \sqrt[n]{a_n} \le \varlimsup \sqrt[n]{a_n} \le \varlimsup \, \dfrac{a_{n+1}}{a_n} \\[3 pt]
    \varliminf \sqrt[n]{a_n} \le \varlimsup \sqrt[n]{a_n} \,$ очевидно, докажем 
    $\, \varlimsup \sqrt[n]{a_n} \le \varlimsup \, \dfrac{a_{n+1}}{a_n}$ \\[3 pt]
    (левое неравенство доказывается аналогично): \\[3 pt]
    Пусть $q = \varlimsup \sqrt[n]{a_n}, \; p = \varlimsup \, \dfrac{a_{n+1}}{a_n}$. \\[3 pt]
    От противного: пусть $p < q$:\\[3 pt]
    $\forall \varepsilon > 0 \; \exists \{ n_k \} : \sqrt[n_k]{a_{nk}} \ge q - \varepsilon \; \Rightarrow \; a_{nk} \ge (q - \varepsilon)^{n_k}$ \\[3 pt]
    $\exists n_0 : \dfrac{a_{n+1}}{a_n} \le p + \varepsilon, \; n \ge n_0 \; \Rightarrow \; a_{n0 + m} \le a_{n0} (p + \varepsilon)^m$ \\[3 pt]
    $(q - \varepsilon)^{n_k} \le a_{nk} \le a_{n0} (p + \varepsilon)^{n_k - n_0} \; \Rightarrow \; $
    $\dfrac{a_{n0}}{(p + \varepsilon)^{n_0}} \ge \left( \dfrac{q - \varepsilon}{p + \varepsilon} \right)^{n_k} \;\, \forall k = 1, 2, \dots;$ \\[3 pt]
    но $\; \dfrac{q - \varepsilon}{p + \varepsilon} > 1 \; $ при малом $\varepsilon$ по предположению $\Rightarrow$ \\[3 pt]
    $\; \Rightarrow \; \left( \dfrac{q - \varepsilon}{p + \varepsilon} \right)^{n_k}$ --- бесконечно большое, тогда как
    $\dfrac{a_{n0}}{(p + \varepsilon)^{n_0}} = C$ --- некоторая константа. \\[3 pt]
    Получили неравенство $C \ge +\infty$ --- противоречие, следовательно, предположение неверно, и неравенство выполняется. \\[3 pt]
    Из $\; \varliminf \, \dfrac{a_{n+1}}{a_n} \le \varliminf \sqrt[n]{a_n} \le \varlimsup \sqrt[n]{a_n} \le \varlimsup \, \dfrac{a_{n+1}}{a_n}\; $ 
        исходное утверждение следует очевидно.
    \end{proof}    
    % вопрос 16
    \subsection*{16}
        \textbf{ Доказать, что если для $\sum a_n$ существует $\lim \dfrac{a_{n+1}}{a_n} = q$, то существует и $\lim \sqrt[n]{a_n} = q$.} \\[5 pt]
        $\exists \lim \dfrac{a_{n+1}}{a_n} = q \; \Rightarrow \; \exists \lim \sqrt[n]{a_n} = q$ \\[3 pt]
        \begin{proof}
        Рассмотрим неравенство (доказанное в п. 15): \\[3 pt]
        $\varliminf \, \dfrac{a_{n+1}}{a_n} \le \varliminf \sqrt[n]{a_n} \le \varlimsup \sqrt[n]{a_n} 
        \le \varlimsup \, \dfrac{a_{n+1}}{a_n}$ \\[3 pt]
        $\exists \lim \dfrac{a_{n+1}}{a_n} = q \; \Leftrightarrow \; 
        \varliminf \, \dfrac{a_{n+1}}{a_n} = \varlimsup \, \dfrac{a_{n+1}}{a_n} = q \; \Rightarrow \\[3 pt]
        \Rightarrow \; q \le \varliminf \sqrt[n]{a_n} \le \varlimsup \sqrt[n]{a_n} \le q \; \Rightarrow 
        \varliminf \sqrt[n]{a_n} = \varlimsup \sqrt[n]{a_n} = q \; \Leftrightarrow \; \exists \lim \sqrt[n]{a_n} = q$, \\[5 pt]
        что и требовалось доказать.
        \end{proof}    
    % вопрос 17
    \subsection*{17. Приведите пример положительного ряда, вопрос о поведении которого не может быть решён с помощью
        признака Даламбера, но может быть решён с помощью радикального признака Коши (с обоснованием)}
    \begin{example}
        \begin{flalign*}
            & 0 < a < 1 < b
            \\
            & 1 + a + ab + a^2b + a^2b^2 + \dots = \sum_{n=1}^\infty a^{\left[\frac{n}{2}\right]} \cdot b^{\left[\frac{n - 1}{2}\right]}
            \\
            & \frac{a_{n+1}}{a_n} = \begin{cases}
                a, \text{ n -- нечёт.} \\
                b, \text{ n -- чёт.}
            \end{cases}
            \implies \overline{\lim}\frac{a_{n+1}}{a_n} = b > 1, \underline{\lim}\frac{a_{n+1}}{a_n} = a < 1
            \\
            & \text{-- признак Даламбера не работает}
            \\
            & \sqrt[n]{a_n} = \begin{cases}
                a^{\frac{1}{2}} \cdot b^{\frac{n-2}{2n}},    & \text{ n -- чёт.}   \\
                a^{\frac{n-1}{2n}} \cdot b^{\frac{n-1}{2n}}, & \text{ n -- нечёт.}
            \end{cases}
            \implies \lim \sqrt[n]{a_n} = \sqrt{ab}
            \\
            & \text{Если } ab \neq 1 \text{, то радикальный признак работает}
        \end{flalign*}
    \end{example}
        
    % вопрос 18
        
    % вопрос 19
    \subsection*{19. Сформулируйте признак Гаусса для положительного ряда. Приведите
    пример применения признака Гаусса.}
    \proposition[Признак Гаусса]{ \ \\
        Пусть $\displaystyle \exists \delta > 0, p: \frac{a_{n + 1}}{a_n} = 
        1 - \frac{p}{n} + O\left(\frac{1}{n^{1+\delta}}\right)$ \\
        Тогда: \\
        если $\displaystyle p > 1 \Rightarrow \sum a_n$ -- сходится
        если $\displaystyle p \leqslant 1 \Rightarrow \sum a_n$ -- расходится
    }
    \example{  $\displaystyle \sum_{n = 2}^{\infty}
    \frac{2 \cdot 5 \cdot 8 \cdot \ldots \cdot (3n - 4)}{3^n \cdot n!}$ \\
    $\displaystyle \frac{a_{n + 1}}{a_n} =
     \frac{(3(n + 1) - 4) \cdot 3^n \cdot n!}{3^{n + 1} \cdot (n + 1)!} = 
     \frac{3(n + 1) - 4}{3(n + 1)} = \frac{3n - 1}{3n + 3} =
     \frac{1 - \frac{\frac{1}{3}}{n}}{1 + \frac{1}{n}} =
     \left(1 - \frac{1}{3n}\right)\left(1 - \frac{1}{n} + O\left(\frac{1}{n^2}\right)\right)=\\
     =1 - \frac{1}{n} + O\left(\frac{1}{n^2}\right) 
     - \frac{\frac{1}{3}}{n} + \frac{\frac{1}{3}}{n^2} - O\left(\frac{1}{3n^3}\right) =
     1 - \frac{\frac{4}{3}}{n} + O\left(\frac{1}{n^2}\right) \implies
     \begin{cases}
        p = \frac{4}{3}\\
        \delta = 1     
     \end{cases} \implies 
     $ ряд сходится по признаку Гаусса.}
     \example{ $\displaystyle
     S = \sum\limits_{n=1}^{\infty}\left(\frac{(2n - 1)!!}{(2n)!!}\right)^2$\\[3pt]
     Применим признак Гаусса:
     $\displaystyle
     \frac{a_{n+1}}{a_n} = \left(\frac{(2n + 1)!!}{(2n + 2)!!}\right)^2\cdot\left(\frac{(2n)!!}{(2n - 1)!!}\right)^2 =
     \left(\frac{(2n + 1)!!}{(2n + 2)!!}\cdot\frac{(2n)!!}{(2n - 1)!!}\right)^2 = \left(\frac{2n + 1}{2n+2}\right)^2 =
     \left(\frac{2 + \frac{1}{n}}{2+\frac{2}{n}}\right)^2
     =\frac{4 + \frac{4}{n} + O\left(\frac{1}{n^2}\right)}{4 + \frac{8}{n} + O\left(\frac{1}{n^2}\right)}=\\[3pt]
     =\frac{1 + \frac{1}{n} + O\left(\frac{1}{n^2}\right)}{1 + \frac{2}{n} + O\left(\frac{1}{n^2}\right)}
     =\left(1 + \frac{1}{n} + O\left(\frac{1}{n^2}\right)\right)\left(1 - \frac{2}{n} + O\left(\frac{1}{n^2}\right)\right)
     =1 - \frac{1}{n} + O\left(\frac{1}{n^2}\right) \implies\\
     \implies
     \begin{cases}
         \delta = 1 \\
         p = 1 & = 1
     \end{cases}
     \implies
     $ ряд расходится по признаку Гаусса.}

    % вопрос 20
    \subsection*{20}
        \textbf{ Привести пример положительного ряда, вопрос о поведении которого не может быть решен с помощью признака Гаусса.} \\[5 pt]
        $\sum\limits_{n=1}^{\infty} \dfrac1{n \ln^p n}$ --- положительный ряд, $a_n = \dfrac1{n \ln^p n}$ \\[3 pt]
        Рассмотрим отношение: \\[3 pt]
        $\dfrac{a_{n + 1}}{a_n} = \dfrac{\dfrac1{(n + 1) \ln^p (n + 1)}}{\dfrac1{n \ln^p n}} = 
        \dfrac{n}{n + 1} \cdot \dfrac{\ln^p n}{\ln^p (n + 1)} = 
        \dfrac{1}{1 + \frac1n} \cdot \dfrac{\ln^p n}{\left( \ln n + \ln \left(1 + \frac1n \right) \right)^p} = \\[3 pt]
        = \left[\text{ По формуле Тейлора для $(1+x)^{-1}$ и $\ln (1+x) \sim x$ }\right] \; 
        \left(1 - \frac1n + \frac1{n^2} + o\left(\frac1{n^2}\right)\right) \cdot \left(1 + \frac{\frac1n + o\left(\frac1n\right)}{\ln n}\right)^{-p} = \\[3 pt]
        \left[\text{ Перешли к менее строгому приближению и  снова разложили $(1+x)^{-p}$ }\right] \\[3 pt] 
        = \left(1 - \frac1n + o\left(\frac1{n \ln n}\right)\right) \cdot \left(1 - \frac{p}{n \ln n} + o\left(\frac1{n \ln n}\right)\right)
        = 1 - \dfrac1n - \dfrac{p}{n \ln n} + o\left(\frac1{n \ln n}\right)$ \\[3 pt]
        Для использования признака Гаусса должны получить приближение 
        $1 - \frac{q}n + O\left( \frac1{n^{1+\delta}} \right), \; \delta > 0$, \\[3 pt]
        но $ - \dfrac{p}{n \ln n} + o\left(\frac1{n \ln n}\right) \ne O\left( \frac1{n^{1+\delta}} \right)$, т.к.
        $\dfrac1{\ln n} > \dfrac1{n^{\delta}}$ при $n \to \infty \;\; \forall \delta > 0$
        
    % вопрос 21
    \subsection*{21. Выведите двустороннюю оценку частичной суммы ряда через неопределённый интеграл. Сформулируйте и докажите интегральный признак Коши-Маклорена.}
    \textbf{Интегральный признак Коши-Маклорена:} Если функция $f(x)$ принимает неотрицательные значения на всей области определения и монотонно убывает, а также $\forall n \in \mathbb{N} : f(n) = a_n$, то $\sum a_n$ и $\int^{\infty} f(x)dx$ сходятся или расходятся одновременно. \\
    \textit{Доказательство:} Рассмотрим убывающую при $x \geq n_0 - 1$ функцию $f(x)$ и ряд $\sum\limits_{n=n_0}^{\infty} a_n$, где $a_n = f(n)$. Заметим, что \\
    \begin{equation*} f(n + t) \leq a_n \leq f(n - 1 + t), \; t \in [0; 1] \end{equation*}
    Проинтегрируем каждый член неравенства определённым интегралом от $0$ до $1$ по $dt$: \\
    \begin{flalign*} 
        & \int_0^1 f(n + t) dt \leq \int_0^1 a_n dt \leq \int^1_0 f(n - 1 + t) dt & \\
        & \int^{n + 1}_n f(x)dx \leq a_n \leq \int^n_{n-1} f(x)dx &
    \end{flalign*}
    Просуммируем эти неравенства при всех $n$: \\
    \begin{equation*}
        \int_{n_0}^{N+1}f(x)dx \leq \sum\limits_{n = n_0}^N a_n \leq \int^N_{n_0 - 1} f(x)dx
    \end{equation*}
    Тогда $\sum a_n$ ведёт себя как несобственный интеграл $\int^{\infty} f(x)dx$. $\blacksquare$ \\
    Двусторонняя оценка для частичной суммы ряда через определённый интеграл была выведена в процессе.

        
    % вопрос 22
        \subsection*{22. Что такое улучшение сходимости положительного ряда? Покажите на примере как можно улучшить сходимость ряда.}

        Пусть у нас есть некоторый ряд $\sum a_n$ и он сходится медленно. В таких случаях для расчёта суммы ряда с необходимой точностью потребуется взять больше членов, что неудобно. Мы можем преобразовать наш ряд для улучшения сходимости, т.е. получить некоторый ряд $\sum a_n'$, который будет сходиться быстрее, чем исходный $\sum a_n$.
        Пусть у нас есть ряд $S = \sum_{n = 1}^{\infty} \dfrac{1}{n^2 + 2} \approx \sum_{n = 1}^{\infty} \dfrac{1}{n^2}$. Воспользуемся  методом Куммера. Для улучшения сходимости будем брать ряды вида $\sum_{n = 1}^{\infty} \dfrac{1}{n(n+1)} = 1, \sum_{n = 1}^{\infty} \dfrac{1}{n(n+1)(n+2)} = \dfrac{1}{4}, \dots$. 
            
        В данном случае нам подойдёт первый ряд в этом списке, поскольку $\dfrac{1}{n^2} \sim \dfrac{1}{n(n+1)}$.
        \begin{flalign*}    
        &\sum_{n=1}^{\infty} \left(\frac{1}{n^2+2} - \frac{1}{n(n+1)}\right) = S - 1 \implies S = 1 + \sum_{n=1}^{\infty} \left(\frac{1}{n^2 + 2} - \frac{1}{n(n+1)}\right)\\
        &\frac{1}{n^2+2} - \frac{1}{n(n+1)} = \frac{1}{n^2} \cdot \left(\frac{1}{1 + \frac{2}{n^2}} - \frac{1}{1 + \frac{1}{n}}\right) = \frac{1}{n^2} \cdot \left(1 - \frac{2}{n^2} + o\left(\frac{1}{n^2}\right) - \left(1 - \frac{1}{n} + \frac{1}{n^2} + o\left(\frac{1}{n^2}\right)\right)\right) = \frac{1}{n^3} + o\left(\frac{1}{n^3}\right) 
        \end{flalign*}
    
    % вопрос 23
    \subsection*{23. Дайте определения: знакопеременный ряд, знакочередующийся ряд, абсолютно сходящийся ряд, 
    условно сходящийся ряд, положительная и отрицательная части ряда.}

    \begin{itemize}
        \item Ряд $\sum a_n$ называется знакопеременным, если на знаки его элементов $a_n$ не наложены ограничения. 
    Фактически любой ряд --- знакопеременный.
        \item Ряд $\sum a_n$ называется знакочередующимся, если $a_i \cdot a_{i+1} < 0 \;\; \forall i \in \NN$.

        \item Ряд $\sum a_n$ сходится абсолютно, если сходятся ряды $\sum a_n$ и $\sum |a_n|$.
        \item Ряд $\sum a_n$ сходится условно, если сходится ряд $\sum a_n$ и расходится ряд $\sum |a_n|$.

        \item Введем последовательности $a_n^+ = 
        \begin{cases}
        a_n, & a_n > 0 \\
        0, & a_n \leqslant 0 \\
        \end{cases}\;$ и $\;a_n^- =
        \begin{cases}
        |a_n|, & a_n < 0 \\
        0, & a_n \geqslant 0 \\
        \end{cases} \implies \\[4 pt] \implies$ ряды $\sum a_n^+$ и $\sum a_n^-$ --- положительная и отрицательная части ряда $\sum a_n$ соответственно.
    \end{itemize}

    % вопрос 24
    \subsection*{24. Докажите, что ряд сходится абсолютно ровно в том случае, когда сходятся его положительная и отрицательная части.}
    \begin{proposition}
        Ряд $\sum a_n$ сходится абсолютно $\iff \sum a_n^+, \sum a_n^- < \infty$ сходятся. 
    \end{proposition}
    \begin{proof}
        Если $\sum \left|a_n\right| < \infty$, то $S_N^+,\ S_N^-$ ограничены $\implies$ сходятся.

        Если $S_N^+ \to S^+,\ S_N^- \to S^-$, то $\sum_{n=1}^N a_n \to S^+ - S^-,\ \sum_{n=1}^N \left|a_n\right| \to S^+ + S^-$.
    \end{proof}
        
    % вопрос 25
    \subsection*{25}
        \textbf{ Доказать, что если ряд сходится условно, то его положительная и отрицательная части расходятся.} \\[5 pt]
    $\sum a_n$ --- ряд, $S_{+} = \sum a_n^{+}$ и $S_{-} = \sum a_n^{-}$
     --- положительная и отрицательная части суммы соответственно. \\[3 pt]
     $\left\{\begin{array}{lll} 
     \sum a_n &=& C,\\[5 pt]
     \sum |a_n| &=& \pm \infty
     \end{array}\right. \Rightarrow \; S_{+}$ и $S_{-}$ расходятся.
    \begin{proof}
    По определению: \\[3 pt]
    $\sum a_n = S_{+} - S_{-}, \;\; \sum |a_n| = S_{+} + S_{-}$. \\[3 pt]
    От противного: пусть\\[-20 pt]
    \begin{enumerate}
    \item $S_{+}, \; S_{-}$ конечны. Тогда $\;\sum |a_n| = S_{+} + S_{-} = C_1 + C_2 = const$ --- сходится, противоречие.
    \item $S_{+} \, $ конечна, $S_{-}$ расходится (симметричный случай аналогично). \\[0 pt]
    Тогда $\; \sum a_n = S_{+} - S_{-} = C_1 - \underbrace{S_{-}}_{\text{беск. большое}} = -\infty$ --- расходится, противоречие.
    \end{enumerate}
    \end{proof}    
    % вопрос 26
        
    % вопрос 27
    \subsection*{27. Что такое группировка членов ряда? Докажите, что любой ряд, полученный из сходящегося
        группировкой его членов, сходится и имеет ту же сумму.}

    \begin{definition}
        Говорят, что ряд $\sum A_k$ получен из ряда $\sum a_n$ группировкой членов, если
        $\exists n_1, n_2, \dots \colon 1 \leq n_1 < n_2 < \dots$ такие, что
        \begin{flalign*}
            & A_1 = a_1 + a_2 + \dots + a_{n_1}
            \\
            & A_2 = a_{n_1 + 1} + a_{n_1 + 2} + \dots + a_{n_2}
        \end{flalign*}
    \end{definition}

    \begin{proposition}
        Если яд $\sum a_n$ сходится, то ряд $\sum A_k$ тоже сходится, причём к той же сумме.
    \end{proposition}

    \begin{proof}
        Последовательность частичных сумм $S_k' = A_1 + \dots + A_k$ ряда $\sum A_k$
        явл. подпоследовательностью последовательности частичных сумм $S_n = a_1 + \dots + a_n$ ряда $\sum a_n$
    \end{proof}

    % вопрос 28
        
    % вопрос 29
     \subsection*{29. Приведите пример приведения преобразования знакопеременного
    (но не знакочередующегося) ряда к знакочередующемуся.}
    \example{
        $\displaystyle \sum_{n = 1}^{\infty}\frac{(-1)^{[\ln{n}]}}{n}$\\
        $\displaystyle (-1)^k:$ \\
        $\displaystyle k \leqslant \ln{n} < k + 1$ \\
        $\displaystyle e^k \leqslant n < e^{k + 1}$ \\
        $\displaystyle A_k = (-1)^k \sum_{n = [e^k] + 1}^{[e^{k + 1}]}\frac{1}{n}$ \\
        $\displaystyle |A_k| \geqslant \frac{1}{e^{k + 1}}([e^{k + 1}] - ([e^k] + 1)) 
        \geqslant \frac{1}{e^{k + 1}}(2[e^{k}] - [e^k] - 1) =  \frac{[e^k] - 2}{e^{k + 1}}
        > \frac{e^k - 2}{e^{k + 1}} \xrightarrow[k \to \infty]{} \frac{1}{e} \neq 0$\\
        $\displaystyle \Rightarrow \sum A_k$ -- расходится (не выполняется необходимое условие
        сходимости ряда) $\displaystyle \Rightarrow \sum a_n$ -- расходится   
    }
    % вопрос 30
            
    % вопрос 31
    \subsection*{31. Сформулируйте признак Лейбница для знакочередующегося ряда. Приведите пример применения признака Лейбница.}
    \textbf{Признак Лейбница:} Если ряд имеет вид $\sum_{n=1}^{\infty}(-1)^n \cdot u_n$ и $u_n$ монотонно убывает к $0$ (обозначение: $u_n \searrow 0$), то ряд сходится. \\
    \textit{Пример: } \\
    $\sum_{n=1}^{\infty} \dfrac{(-1)^{n}}{n^p}$, $p > 0$ \\
    $\dfrac{1}{n^p} \searrow 0 \implies $ ряд сходится (при $\forall p > 0$) \\
        
    % вопрос 32
        \subsection*{32. Покажите на примере, что к знакопеременным рядам неприменим предельный признак сравнения}

        Рассмотрим 2 ряда: $\sum_{n=1}^{\infty} \dfrac{(-1)^{n}}{\sqrt{n} - (-1)^{n}}$ и $\sum_{n=1}^{\infty} \dfrac{(-1)^{n}}{\sqrt{n}}$. Второй ряд сходится по признаку Лейбница.

        $\dfrac{(-1)^{n}}{\sqrt{n} - (-1)^{n}} \approx \dfrac{(-1)^{n}}{\sqrt{n}}$

        $\dfrac{(-1)^{n}}{\sqrt{n} - (-1)^{n}} - \dfrac{(-1)^{n}}{\sqrt{n}} = \dfrac{1}{\sqrt{n}(\sqrt{n} - (-1)^{n})} \approx \dfrac{1}{n}$ -- расходится

        $\sum_{n=1}^{N} \dfrac{(-1)^{n}}{\sqrt{n} - (-1)^{n}} = \sum_{n=1}^{N} \dfrac{(-1)^{n}}{\sqrt{n}} + \sum_{n=1}^{N} \dfrac{1}{\sqrt{n}(\sqrt{n} - (-1)^{n})}$ -- расходится как сумма сходящегося и расходящегося ряда.
    
    % вопрос 33
    \subsection*{33. Покажите, что для любых числовых последовательностей $\{a_n\}, \, \{B_n\}$ справедлива формула суммирования по частям (преобразование Абеля):}
    \begin{flalign*}
    \sum_{n = m + 1}^N \!\! a_n \, (B_n - B_{n - 1}) \:=\: (a_N B_N - a_m B_m) \;-\!\! \sum_{n = m + 1}^N \!\! (a_n - a_{n - 1}) \, B_{n - 1}
    \end{flalign*}

    \begin{proof}
        Заметим, что $a_n \, (B_n - B_{n - 1}) = (a_n B_n - a_{n-1} B_{n-1}) - (a_n - a_{n-1}) B_{n-1}$. Просуммируем левую часть от $m + 1$ до $N$:
        \begin{flalign*}
        & \sum_{n = m + 1}^N \!\! a_n \, (B_n - B_{n - 1}) = \sum_{n = m + 1}^N \!\! (a_n B_n - a_{n-1} B_{n-1}) \;-\!\! \sum_{n = m + 1}^N \!\! (a_n - a_{n-1}) \, B_{n-1} = \\
        & (a_N B_N - a_m B_m) \;-\!\! \sum_{n = m + 1}^N \!\! (a_n - a_{n-1}) \, B_{n-1}
        \end{flalign*}
    \end{proof}
        
    % вопрос 34
    \subsection*{34. Сформулируйте признак Дирихле. Приведите пример его применения.}
    \begin{proposition}[Признак Дирихле.]
        Если $a_n \searrow 0$ и $\left| \sum_{n=1}^N b_n \right| = \left| B_N \right| \leq C$~--- ограничена, то ряд $\sum a_n b_n$ сходится. 
    \end{proposition}
    \begin{example}
        \begin{flalign*}    
            &\sum_{n=1}^\infty \frac{\sin nx}{n^p},\ x \neq \pi k,\ p > 0 \\
            &a_n = \frac{1}{n^p},\ b_n = \sin nx \\
            &B_n = \sin x + \sin 2x + \ldots + \sin nx = \frac{\cos \frac{x}{2} - \cos\left(\left(n + \frac{1}{2}\right)x\right)}{2 \sin \frac{x}{2}}; \hspace{1cm} \left| B_n \right| \leq \frac{1}{\left| \sin \frac{x}{2} \right|} \\
            &\implies \text{ ряд сходится по признаку Дирихле. }
        \end{flalign*}
    \end{example}
        
    % вопрос 35
    \subsection*{35}
        \textbf{Сформулировать признак Абеля. Вывести утверждение признака Абеля из признака Дирихле.} \\[5 pt]
    \textbf{Признак Абеля. } Если $\{ a_n \}$ монотонна и ограничена $|a_n| \le C$, а $\sum b_n$ сходится,
    ряд $\; \sum a_n \cdot b_n \;$ также сходится.\\[5 pt]
    Пусть некоторая последовательность $a_n \cdot b_n$ удовлетворяет признаку Абеля. \\[3 pt]
    У монотонной ограниченной последовательности существует конечный предел: $\lim a_n = A$. \\[3 pt]
    Представим исходную последовательность в виде суммы: \\[3 pt]
    $a_n \cdot b_n = A \cdot b_n + (a_n - A) b_n \; \Rightarrow \; 
    \sum a_n \cdot b_n = \underbrace{\sum A \cdot b_n}_{\text{сходится}} + \sum (a_n - A) b_n$ \\[3 pt]
    $a_n \to A \; \Rightarrow \; (a_n - A) \to 0$, причем, т.к. $\{ a_n \}$ монотонная, $\{ (a_n - A) \}$ монотонно стремится к 0. \\[3 pt]
    Т.к. ряд $\;\sum b_n$ сходится, последовательность его частичных сумм также сходится. \\[3 pt]
    $\left\{\begin{array}{lll} 
    \{ (a_n - A) \} &\downarrow& 0,\\[10 pt]
    \left\{ \sum\limits_{n=1}^N b_n \right\} &\le& B
    \end{array}\right. \Rightarrow \; \sum (a_n - A) b_n$ сходится по признаку Дирихле.\\[3 pt]
    $\sum a_n \cdot b_n = \underbrace{\sum A \cdot b_n}_{\text{сходится}} +\underbrace{\sum (a_n - A) b_n}_{\text{сходится}}$ --- сходится.    
    % вопрос 36
    
    \subsection*{36. Что такое перестановка членов ряда? Приведите пример.}
    
    Пусть $f: \NN \to \NN$ биекция.
    
    Говорят, что ряд $\sum b_n$ получен из ряда $\sum a_n$ перестановкой членов, если $\exists$ биекция $f: \; b_n = a_{f(n)}.$ 
    
    \underline{Пример.}
    
    $\displaystyle \sum_{n = 1}^{\infty} a_n = \sum_{n = 1}^{\infty} \frac{(-1)^n}{n} = -1 + \frac{1}{2} - \frac{1}{3} + \frac{1}{4} - \dots = -\ln 2.$
    
    Пусть $\sum b_n$ получен так: сложим сначала $p$ положительных слагаемых из $\sum a_n$, потом $q$ отрицательных, затем снова $p$ положительных и так далее ($p, q \in \NN$, берем слагаемые по возрастанию их индексов).
    
    % вопрос 37
    \subsection*{37. Сформулируйте свойство абсолютно сходящегося ряда, связанное с перестановкой членов.}
    \begin{proposition}
        Сумма абс. сходящегося ряда не меняется при любой перестановке его членов
    \end{proposition}
        
    % вопрос 38
    \subsection*{38. Сформулируйте свойство условно сходящегося ряда, связанное с перестановкой членов (теорема Римана).}
    \begin{theorem}[Свойство условно сходящегося ряда (теорема Римана)]
        Каков бы ни был условно сходящийся ряд $\sum a_n$ и $S \in \left[-\infty;\ +\infty\right]$, найдётся такая перестановка $f: \NN \to \NN$, что $\sum a_{f(n)} = S$.
    \end{theorem}

        
    % вопрос 39
    \subsection*{39. Приведите пример условно сходящегося ряда и перестановки, меняющей его сумму
    (с обоснованием).}
    \example{
        $\displaystyle \sum_{n = 1}^{\infty} \frac{(-1)^n}{n} = -1 + \frac{1}{2} -
         \frac{1}{3} + \frac{1}{4} - \ldots = -\ln{2}$\\
        $\displaystyle S_{2n}^{+} = \frac{1}{2} + \frac{1}{4} + \ldots + \frac{1}{2n} =
         \frac{1}{2}(\ln{n} + \gamma) + o(1)$\\ 
        $\displaystyle S_{2n}^{-} = 1 + \frac{1}{3} + \ldots + \frac{1}{2n - 1} = 
        \ln{2} + \frac{1}{2}(\ln{n} + \gamma) + o(1)$\\
        Пусть берётся $\displaystyle p$ положительных слагаемых, затем $\displaystyle q$ 
        отрицательных и так далее. \\ 
        Тогда после $\displaystyle m$ действий получим: \\
        $\displaystyle S_{2mp}^+ = \frac{1}{2} + \frac{1}{4} + \ldots + \frac{1}{2mp}=
        \frac{1}{2}(\ln{(mp)} + \gamma) + o(1)$ \\
        $\displaystyle S_{2mq - 1}^- = 1 + \frac{1}{3} + \ldots + \frac{1}{2mq - 1}=
        \ln{2} + \frac{1}{2}(\ln{(mq)} + \gamma) + o(1)$ \\
        $\displaystyle S_{2mp}^{+} - S_{2mq}^{-} = -ln{2} + \frac{1}{2}\ln{\left(\frac{p}{q}\right)} +
        o(1)$ \\
        $\displaystyle \Rightarrow$ ряд сходится к числу $\displaystyle -\ln\left(2\sqrt{\frac{q}{p}}\right)$ 
    }
    % вопрос 40
    
    % вопрос 41
    \subsection*{41. Что такое произведение рядов в форме Коши? Приведите пример вычисления такого произведения.}
    \textbf{Произведение рядов в форме Коши:} Если $\left(\sum_{k=1}^{\infty} a_k\right) \cdot \left(\sum_{m=1}^{\infty} b_m \right) = \sum_{n=2}^{\infty} c_n$, то $c_i = \sum\limits_{j = 1}^{i - 1} a_j \cdot b_{i - j}, \; i \geq 2$. \\
    \textit{Пример:} $\left(\sum\limits_{k=1}^{\infty} k + 1 \right) \cdot \left( \sum\limits_{m = 1}^{\infty} m^2 \right) = \sum\limits_{n = 2}^{\infty} c_n$. Для примера посчитаем несколько первых членов $c_n$: \\
    $c_2 = a_1 \cdot b_1 = 2 \cdot 1 = 2$\\
    $c_3 = a_2 \cdot b_1 + a_1 \cdot b_2 = 3 \cdot 1 + 2 \cdot 4 = 11$ \\
    $c_4 = a_3 \cdot b_1 + a_2 \cdot b_2 + a_1 \cdot b_3 = 4 \cdot 1 + 3 \cdot 4 + 2 \cdot 9 = 34$ \\
    $\dots$ \\

        
    % вопрос 42
        \subsection*{42. Дайте определения: бесконечное произведение, частичное произведение, сходящееся бесконечное произведение, расходящееся бесконечное произведение.}
 
        $\prod_{n=1}^{N} a_n = a_1 \cdot a_2 \cdot \dots \cdot a_N$ -- частичное произведение.

        Бесконечным произведением называют формальную запись $\prod_{n=1}^{\infty} a_n$

        Значением бесконечного произведения является предел частичного произведения:

        $\prod_{n=1}^{\infty} a_n = \lim_{N \to \infty} \prod_{n=1}^{N} a_n$

        Если предел существует и он конечен -- то бесконечное произведение сходится, иначе расходится.
    
    % вопрос 43
    \subsection*{43. Сформулируйте и докажите необходимое условие сходимости бесконечного произведения.}

    Если бесконечное произведение $\prod a_n$ сходится, то $a_n \xrightarrow{n \to \infty} 1$.

    \begin{proof}
        Пусть $P_N = \prod \limits_{n = 1}^N a_n$ --- частичное произведение. Тогда $a_n = \dfrac{P_n}{P_{n - 1}} \xrightarrow{n \to \infty} 1$, 
    так как $\lim \limits_{n \to \infty} P_n = \lim \limits_{n \to \infty} P_{n - 1} = \prod \limits_{n = 1}^{\infty} a_n$.
    \end{proof}
        
    % вопрос 44
    \subsection*{44. Пусть последовательности $\left\{ a_n\right\},\ \left\{A_n\right\},\ A_n \neq 0$ таковы, что $a_n = \frac{A_n}{A_{n-1}} \cdot c_n$ и бесконечное произведение $\prod c_n$ сходится. Докажите, что существует число $C \neq 0$ такое, что $\prod_{n=1}^N a_n = A_N \left(C + o(1)\right)$.}
    \begin{proof}
        \begin{flalign*}
            &a_n = \frac{A_n}{A_{n-1}} \cdot c_n, \hspace{1cm} \prod c_n \text{ сходится, то есть } \prod_{n=1}^N c_n \to P \neq 0 \\
            &\prod_{n=1}^N a_n= \frac{\cancel{A_1}}{A_0} \cdot c_1 \cdot \frac{\cancel{A_2}}{\cancel{A_1}} \cdot c_2 \cdot \ldots \cdot \frac{A_N}{\cancel{A_{N-1}}} \cdot c_N = A_N \cdot \underbrace{\frac{1}{A_0} \cdot \prod_{n=1}^N c_n}_{\to \frac{P}{A_0} \neq 0} \\
            & \implies \prod_{n=1}^N a_n = A_N \cdot \left(C + o(1)\right), \hspace{1cm} C = \frac{P}{A_0} \neq 0
        \end{flalign*}
    \end{proof}
        
    % вопрос 45
    \subsection*{45}
        \textbf{Как определяется соответствующий бесконечному произведению ряд? Сформулировать и доказать утверждение об их взаимосвязи.} \\[5 pt]
    Пусть $\prod\limits_{n = 1}^{\infty} a_n$ --- бесконечное произведение. \\[3 pt]
    Тогда ряд $\sum\limits_{n = 1}^{\infty} \ln a_n$ называется соответствующим этому бесконечному произведению. \\[3 pt]
    Так как $a_n = e ^{\ln a_n}$, верно равенство $\; \prod\limits_{n = 1}^{\infty} a_n = \prod\limits_{n = 1}^{\infty} e^{\ln a_n} = $ 
    {\large $e^{\;\sum\limits_{n = 1}^{\infty} \ln a_n}$ } (по свойству степени)    
    % вопрос 46
        
    \subsection*{46. В каком случае бесконечное произведение называется сходящимся абсолютно? Сформулируйте и докажите критерий абсолютной сходимости бесконечного произведения.}
    
    $\displaystyle \prod_{n = 1}^{\infty} a_n$ наз-ся абсолютно сходящимся, если абсолютно сх-ся соответствующий ряд из логарифмов $\displaystyle \sum_{n = 1}^{\infty} \ln a_n$.
    
    Критерий абс. сх-ти:
    
    \fbox {$\displaystyle \prod_{n = 1}^{\infty} a_n$ сход. абс. $\iff \sum_{n = 1}^{\infty} (a_n - 1)$ сход. абс.}
    
    \begin{proof}
    
    Пусть $a_n = 1 + \alpha_n; \; \alpha_n \to 0. \; \circledast$
    
    Тогда $\ln a_n = \ln (1 + \alpha_n) = \alpha_n + \overline{o} (\alpha_n ) =  \alpha_n(1 + \overline{o} (1)) \implies
    |\ln a_n| = |\alpha_n| \cdot (1 + \overline{o} (1)),$ то есть $|\ln a_n| \sim |\alpha_n|.$
    
    \textit{Возможно, тут стоит упомянуть, что необходимое условие сходимости $\displaystyle \sum_{n = 1}^{\infty} |\ln a_n|$ это $|\ln a_n| \to 0 \iff a_n \to 1.$ Поэтому, если $\displaystyle \prod_{n = 1}^{\infty} a_n$ сход. абс., то $\circledast$ у нас верно всегда.}
    
    \end{proof}
    
    % вопрос 47
    \subsection*{47. Напишите произведение Валлиса и его значение (формула Валлиса). Вычисление каких
        интегралов приводит к этой формуле?}
    \begin{proposition}
        Произведение Валлиса
        \begin{flalign*}
            & \prod_{n=1}^\infty \frac{4n^2}{4n^2-1} = \frac{\pi}{2} \text{ -- формула Валлиса}
            \\
            &  \text{-- получается из анализа интегралов } \int_{0}^{\frac{\pi}{2}}\sin^4x dx
        \end{flalign*}
    \end{proposition}
        
    % вопрос 48
        
    % вопрос 49
    \subsection*{49. Дайте определения: функциональная последовательность, точка сходимости функциональной последовательности, область (множество) сходимости функциональной последовательности, поточечная сходимость функциональной последовательности на данном множестве.}
    \definition{
        Функциональным рядом (последовательностью) называется такой ряд (последовательность), что его элементами являются не числа, а функции $f_n(x)$.
    }
   
    \definition{
           Пусть $\forall n, n \in \mathbb{N}, f_n: D \rightarrow \mathbb{R}, D \subseteq \mathbb{R}$.
           Говорят, что $a \in D$ - точка сходимости $\{f_n(x)\}$, если последовательность $\{f_n(a)\}$ сходится.
    }
   \definition{
        Множество всех точек сходимости называется множеством сходимости.
    }
   \definition{
        Говорят, что последовательность сходится на $D$ поточечно, если $D$ – множество сходимости.
   }
    % вопрос 50
        
    % вопрос 51
    \subsection*{51. Сформулируйте определения равномерной сходимости функциональной последовательности: в терминах нормы и на языке $\varepsilon - \delta$.}
    \begin{enumerate}
        \item $f_n \overset{D}{\rightrightarrows} f \iff ||f_n - f|| \rightarrow 0$. \\
      \item $\sum f_n(x) \rightrightarrows S(x) \iff \forall \epsilon > 0, \exists N(\epsilon): \forall n \geqslant N(\epsilon), |S_n(x) - S(x)| < \epsilon$. \\
    \end{enumerate}
        
        
    % вопрос 52
        \subsection*{52. Докажите, что из равномерной сходимости следует поточечная сходимость на данном множестве.}
        Определение поточечной сходимости $\forall x \in E\ \forall \varepsilon > 0\ \exists N = N(\varepsilon, x) : \forall n \geqslant N\ |f_n(x) - f(x)| < \varepsilon$

        Определение равномерной сходимости $\forall \varepsilon > 0\ \exists N = N(\varepsilon) : \forall n \geqslant N\ \forall x \in E\ |f_n(x) - f(x)| < \varepsilon$
        \begin{proof}
        Видно, что в определении равномерной сходимости номер $N$ зависит от $\varepsilon$ и не зависит от $x$, а в определении поточечной - и от $\varepsilon$, и от $x$. Если выполняется равномерная сходимость, то $\forall x \in E\ \exists$ нужное $N$, то есть выполняется поточечная сходимость.
        \end{proof}
    
    % вопрос 53
    \subsection*{53. Приведите пример функциональной последовательности, сходящейся поточечно, но не сходящейся равномерно (с обоснованием).}

    Рассмотрим функциональную последовательность $f_n(x) = \dfrac1{1 + nx}, \; D = [0, \, 1]$ (семинарская задача 4.20). 
    При $x = 0 \;\; f_n(x) = 1$, а при $x \ne 0 \;\; f_n(x) \xrightarrow{n \to \infty} 0 \implies f_n$ сходится поточечно на $D$. 
    При этом равномерная сходимость отсутствует, так как $f_n$ непрерывна $\forall \, n$, а $f$ разрывна в нуле.
        
    % вопрос 54
    \subsection*{54. Приведите пример функциональной последовательности $\left\{f_n(x)\right\}$ (с нетривиальной зависимостью от $n$ и $x$), равномерно сходящейся на некотором множестве (с обоснованием).}
    \begin{example}
        \begin{flalign*}
            &f_n(x) = \frac{1}{n + x}, \hspace{1cm} D = \left[ 0; +\infty \right) \\
            &f_n(x) \overset{D}{\rightrightarrows} 0 \\
            & \norm{f_n - 0} = \norm{f_n} = \sup_{x \in D} \left| f_n(x) \right| = \frac{1}{n} \to 0 \\
            & \implies \text{ последовательность сходится равномерно.}
        \end{flalign*}
    \end{example}
        
    % вопрос 55
    \subsection*{55}
        \textbf{ Доказать, что если две функциональные последовательности сходятся равномерно к предельным функциям, то их сумма также сходится равномерно к сумме двух этих предельных функций.} \\[5 pt]
    $\left\{\begin{array}{lll} 
    f_n &\overset{D}{\rightrightarrows}& f, \\[5 pt]
    g_n &\overset{D}{\rightrightarrows}& g
    \end{array}\right. \Rightarrow \; (f_n + g_n) \overset{D}{\rightrightarrows} (f + g)$
    \begin{proof}
    По определению равномерной сходимости: \\[3 pt]
    $\forall \varepsilon > 0 \; \exists N_1(\varepsilon) : |f_n(x) - f(x)| < \dfrac{\varepsilon}2 \;\;\; \forall n \ge N_1(\varepsilon)\;\; \forall x \in D$, \\[3 pt]
    $\forall \varepsilon > 0 \; \exists N_2(\varepsilon) : |g_n(x) - g(x)| < \dfrac{\varepsilon}2 \;\;\; \forall n \ge N_2(\varepsilon)\;\; \forall x \in D$ \\[3 pt]
    Тогда \\[3 pt]
    $\forall x \in D \;\; \forall n \ge max(N_1, N_2) : |(f_n(x) + g_n(x)) - (f(x) + g(x))|  = |(f_n(x) - f(x)) + (g_n(x) - g(x))|  \le \\[3 pt]
    \le |f_n(x) - f(x)| + |g_n(x) - g(x)| < \dfrac{\varepsilon}2 + \dfrac{\varepsilon}2 = \varepsilon$, т.е. \\[3 pt]
    $\forall \varepsilon > 0 \; \exists N = max(N_1, N_2) : |(f_n(x) + g_n(x)) - (f(x) + g(x))| < \varepsilon \;\;\; n \ge N \;\; \forall x \in D$, \\[5 pt]
    т.е. сумма $(f_n + g_n)$ равномерно сходится к $(f + g)$ на $D$.
    \end{proof}    
    % вопрос 56
    
    \subsection*{56. Докажите, что если 2 функциональные последовательности сходятся равномерно к ограниченным предельным функциям, то их произведение также сходится равномерно к произведению этих предельных функций.}
        
    \begin{proof}
    
    Пусть наши последовательности - $\{f_n\}, \, \{g_n\};$ их предельные функции - $f, g$ соотв.
    
    Знаем: $\forall \; \varepsilon_1, \varepsilon_2 \; \exists \; N_1(\varepsilon_1), \, N_2(\varepsilon_2): |f_n(x) - f(x)| < \varepsilon_1; \; |g_m(x) - g(x)| < \varepsilon_2$ при $n \geq N_1(\varepsilon_1), \;m \geq N_2(\varepsilon_2).$
    
     Пусть $|f(x)|$ ограничен ограничен какой-нибудь константой $C_1$.
    
    Так как $|g(x)|$ ограничен, то $|g_n(x)|$ ограничен какой-нибудь константой $C_2$. Следовательно,
    \begin{flalign}
    & |f_n(x)\cdot g_n(x) - f(x)\cdot g(x)| = \\
    & = |f_n(x)\cdot g_n(x) - f(x)\cdot g_n(x) + f(x)\cdot g_n(x) - f(x)\cdot g(x)| \leq \\
    & \leq |f_n(x)\cdot g_n(x) - f(x)\cdot g_n(x)| + |f(x)\cdot g(x) - f(x)\cdot g_n(x)| =\\
    & = |g_n(x)| \cdot |f_n(x) - f(x)| + |f(x)| \cdot |g(x) - g_n(x)| \leq C_2 \cdot \epsilon_1 + C_1 \cdot \epsilon_2 \text{ (начиная с $n = \max(N_1(\varepsilon_1), N_2(\varepsilon_2))$.}
    \end{flalign}
    Теперь возьмем произвольный $\varepsilon > 0$, и положим $\varepsilon_1 = \frac{\varepsilon}{3 \cdot C_2}; \; \varepsilon_2 = \frac{\varepsilon}{3 \cdot C_1}.$  
    
    Начиная с $n = \max(N_1(\varepsilon_1), N_2(\varepsilon_2))$ верно, что $ |f_n(x)\cdot g_n(x) - f(x)\cdot g(x)| \leq \varepsilon/3 + \varepsilon/3 < \varepsilon.$ Мы победили.
    \end{proof}
        
    % вопрос 57
    \subsection*{57. Пусть функциональная последовательность $\{f_n\}$ сходится равномерно на множестве $D$
        к предельной функции $f$, отделённой от нуля (т.е. $\inf_{x \in D} |f(x)| > 0$), то функциональная
        последовательность $\frac{1}{f_n}$ сходится равномерно на $D$ к $\frac{1}{f}$.}

    \begin{proof}
        \begin{flalign*}
            & \norm{\frac{1}{f_n} - \frac{1}{f}} = \norm{\frac{f_n - f}{f_n \cdot f}} =
            \sup_{x \in D}{\left|\frac{f_n - f}{f_n \cdot f}\right|} \circled{\leq} \sup_{x \in D} \frac{\epsilon}{|f_n \cdot f|} \; \text{ при } n \geq N(\epsilon), \;  \text{т.к. } \norm{f_n - f} \leq \epsilon \text{ при } n \geq N(\epsilon).
            \\ \\
            & \inf |f(x)| = m > 0 \implies |f(x)| \geq m \; \forall \, x \in D. \\
            \\
            & |f_n| + |f_n - f| \geq |f_n - (f_n - f)|  = |f| \iff |f_n| \geq |f| - |f_n - f|. \text{ Поэтому если } \epsilon < m/2 \text{, то }\\ \\
            & |f_n| \geq |f| - |f_n - f| > m - \epsilon > m - m/2 = m/2 \; \text{ при } n \geq N(\epsilon),  \; \text{ведь } |f| \geq m, \; |f_n - f| < \epsilon < m/2.\\
            & |f_n| > m/2 \implies \frac{1}{|f_n|} < \frac{2}{m}; ~ |f| \geq m \implies \frac{1}{|f|} \leq \frac{1}{m}. \text{ Поэтому:} \\
            &  \norm{\frac{1}{f_n} - \frac{1}{f}} \leq \sup_{x \in D} \frac{\epsilon}{|f_n \cdot f|} < \frac{\epsilon}{m/2 \cdot m} = \epsilon \cdot \frac{2 }{m^2} \; \; (\forall \, n \geq N(\epsilon))\\
        \end{flalign*}
        Так как $\frac{2}{m^2}$ - фиксированное число, а $\epsilon$ у нас - сколь угодно малое, то это означает, что $ \norm{\frac{1}{f_n} - \frac{1}{f}} \to 0,$ что является по определению равномерной сходимостью $f_n$ к  $f$.
    \end{proof}
        
    % вопрос 58
        
    % вопрос 59
    \subsection*{59. Пусть $\displaystyle \varphi: G \to D$ -- биекция. Докажите, что
    равномерная сходимость функциональной последовательности $\displaystyle \{f_n\}$
    на множество $\displaystyle D$ равносильна равномерное сходимости на функциональной
    последовательности $\displaystyle \{f_n \circ \varphi\}$ на множестве $G$.
    }
    \begin{proof} \ \\
        $\displaystyle X \in D, f_n(x)$\\
        $ t \in G, \varphi(t) \in D$ \\
        $\displaystyle (f_n \circ \varphi)(t) = f_n(\varphi(t))$ \\
        Знаем, что $\displaystyle f_n \overset{D}{\rightrightarrows} f$ \\
        Хотим доказать: $\displaystyle f_n \circ \varphi \overset{G}{\rightrightarrows}
        f \circ \varphi$ \\[9.5pt]
        $\displaystyle ||f_n \circ \varphi - f \circ \varphi|| =
        \underset{t \in G}{sup} |f_n(\varphi(t)) - f(\varphi(t))| = M_n$ \\ 
        Что означает, что супремум равен $\displaystyle M_n$? Это означает, что:
        \begin{itemize}
            \item[1)] $\displaystyle |f_n(\varphi(t)) - f(\varphi(t))| \leqslant M_n, \forall t$
            \item[2)] $\displaystyle \exists \{t_k\}: |f_n(\varphi(t_k)) - f(\varphi(t_k))|
            \xrightarrow[k \to \infty]{} M_n$ 
        \end{itemize}  
        Что получаем?
        \begin{itemize}
            \item[1)] $\displaystyle \Leftrightarrow \forall x \in D \
            |f_n(x) - f(x)| \leqslant M_n$
            \item[2)] $\displaystyle \Leftrightarrow \exists \{x_k\}: 
            |f_n(x_k) - f(x_k)| \xrightarrow[k \to \infty]{} M_n$, где $\displaystyle x_k =
            \varphi(t_k)$
        \end{itemize}
        $\displaystyle \Rightarrow M_n =
        \underset{x \in D}{sup} |f_n(x) - f(x)|$ \\ 
        $\displaystyle \Rightarrow ||f_n \circ \varphi - f \circ \varphi||_G =
        ||f_n - f||_D$\\
        Получается, что если одна норма равна 0, то и вторая норма будет равна 0. А так как везде
        знаки равносильности, то доказали мы сразу в две стороны.
    \end{proof}
    % вопрос 60
        
    % вопрос 61
    \subsection*{61. Докажите, что предел равномерно сходящейся последовательности непрерывных функций является непрерывной функцией.}
    \textit{Доказательство:} Пусть функция $s(x)$ $-$ предел некоторой последовательности непрерывных функций $s_n(x)$. Тогда непрерывность функции $s(x)$, которую нам нужно доказать, по определению будет заключаться в том, что в любой точке $x_0$ для любого $\varepsilon > 0$ можно найти такое $\delta$, что из $|h| < \delta$ следует, что $|s(x_0 + h) - s(x_0)| < \varepsilon$. \\
    Для любых $x_0, h, n$ имеем \\
    \begin{flalign*}
        &|s(x_0 + h) - s(x_0)| = |s(x_0 + h) - s_n(x_0 + h) + s_n(x_0 + h) - s_n(x_0) + s_n(x_0) - s(x_0)|\leq &\\
        &\leq |s(x_0 + h) - s_n(x_0 + h)| + |s_n(x_0 + h) - s_n(x_0)| + |s_n(x_0) - s(x_0)|&
    \end{flalign*}
    По определению равномерной сходимости мы можем взять такое $n$, что для любого $x_0$ будет выполняться неравенство $|s(x_0) - s_n(x_0)| < \frac{\varepsilon}{3}$. Значит справедливы неравенства \\
    \begin{equation*}
        |s(x_0 + h) - s_n(x_0 + h)| < \frac{\varepsilon}{3}
    \end{equation*}
    \begin{equation*}
        |s(x_0) - s_n(x_0)| < \frac{\varepsilon}{3}
    \end{equation*}
    Итак, пусть мы зафиксировали некоторое $n$, тогда, поскольку функция $s_n(x)$ монотонна по условию, найдётся такое $\delta$, что для любого $|h| < \delta$ выполняется неравенство $|s_n(x_0 + h) - s_n(x_0)| < \frac{\varepsilon}{3}$. Таким образом \\
    \begin{equation*}
        |s(x_0 + h) - s(x_0)| \leq |s(x_0 + h) - s_n(x_0 + h)| + |s_n(x_0 + h) - s_n(x_0)| + |s_n(x_0) - s(x_0)| < \frac{\varepsilon}{3} + \frac{\varepsilon}{3} + \frac{\varepsilon}{3} = \varepsilon \; \; \blacksquare
    \end{equation*}

    % вопрос 62
        
    % вопрос 63
    \subsection*{63. Приведите контрпример, показывающий, что в формулировке теоремы Дини о равномерной сходимости 
    нельзя отказаться от условия непрерывности предельной функции (с обоснованием).}

    Возьмем функциональную последовательность $f_n(x) = \dfrac1{1 + nx}, \; D = [0, \, 1]$ (семинарская задача 4.20). 
    Если исключить условие непрерывности предельной функции, то остальные условия выполнятся, что означало бы равномерную сходимость $f_n$ на $D$:

    \begin{itemize}
        \item $D = [0, \, 1]$ --- действительно компакт

        \item $f_n$ монотонно убывает на $D$

        \item $f_n$ непрерывна на $D$
    \end{itemize}

    При этом $f_n(x) \xrightarrow{n \to \infty} f(x) = 
    \begin{cases}
    1, & x = 0 \\
    0, & x \ne 0 \\
    \end{cases} \implies f_n$ сходится неравномерно на $D$. Следовательно, \\[4 pt] 
    непрерывность предельной функции нужно обязательно учитывать при использовании теоремы Дини.
        
    % вопрос 64
    \subsection*{64. Покажите на примере как доказать неравномерность сходимости функциональной последовательности с помощью локализации особенности (с обоснованием).}
    \begin{flalign*}
        &f_n = \frac{1}{x^n}, \hspace{1cm} D = \left(1,\ +\infty\right) \\
        &f_n \overset{D}{\to} f = 0 \\
        &f_n \to \begin{dcases}
            1,\ &x = 1\\
            0,\ &x > 1
        \end{dcases}
    \end{flalign*}
    Получили, что, если добавить точку, непрерывная функция стремится к разрывной $\implies$ локализовали особенность $\implies$ функциональная последовательность $f_n$ сходится неравномерно.
    
    % вопрос 65
    \subsection*{65}
    \textbf{ Сформулировать и доказать теорему о почленном переходе к пределу в функциональной последовательности.} \\[5 pt]
    $-\infty \le a < b \le +\infty$, рассмотрим $D = (a;\,b), \; D = [a;\,b]$ \\[3 pt]
    Пусть $\; f_n \overset{D}{\rightrightarrows} f, \;\; x \in D, \;\; y_n = \lim\limits_{x \to x_0} f_n(x), \;\; \{ y_n \}$ сходится к $y$ \\[3 pt]
    Тогда $\; \lim\limits_{x \to x_0} f(x) = y$, \\[3 pt]
    т.е. $\; \lim\limits_{x \to x_0} \underbrace{\left( \lim\limits_{n \to \infty} f_n(x) \right)}_{f(x)} = 
    \underbrace{\lim\limits_{n \to \infty} \overbrace{\left( \lim\limits_{x \to x_0} f_n(x) \right)}^{y_n}}_{y}$
    \begin{proof}
    По определению предела сходящейся последовательности: \\[3 pt]
    $\forall \varepsilon > 0 \; \exists N : |y - y_n| < \dfrac{\varepsilon}3, \;\; \norm{f_n - f} < \dfrac{\varepsilon}3 \;\; \forall n \ge N$, \\[3 pt]
    $\forall \varepsilon > 0 \; \exists \delta > 0 : |x - x_0| < \delta \;\Rightarrow\; |f_n(x) - y_n| < \dfrac{\varepsilon}3 \;\;\; \forall n$ \\[5 pt]
    Тогда \\[5 pt]
    $|y - f_n(x)| \le |y - y_n| + |y_n - f_n(x)| + |f_n(x) - f(x)| < \varepsilon$, \\[5 pt]
    т.е. $f_n(x) \xrightarrow{x \to x_0} y$, что и требовалось доказать.
    \end{proof}    
    % вопрос 66
    
    \subsection*{66. Сформулируйте теорему о почленном дифференцировании функциональной последовательности.}
    
    $-\infty \leq a < b \leq +\infty, \; \; D = (a, b)$ или $D = [a, b].$
    
    Пусть $f_n$ дифф. на мн-ве $D$, и $f'_n  \stackrel{D}{\rightrightarrows} g, \; \exists \; c \in D : \{f_n(c)\}$ сходится.
    
    Тогда $\exists$ такая предельная функция $f : f_n \stackrel{D}{\to} f$ (причем, если $D$ ограничена, то $f_n \stackrel{D}{\rightrightarrows} f$), что $f$ дифф., и $f' = g.$
    
    Говоря иначе, $\displaystyle \left( \lim_{n \to \infty} f_n(x) \right)' = \lim_{n \to \infty} f'_n(x).$
    
    % вопрос 67
    \subsection*{67. Сформулируйте теорему о почленном интегрировании функциональной последовательности.}
    \begin{proposition}
        \begin{flalign*}
            & -\infty < a < b < \infty, ~ D = [a; b]
            \\
            & \text{Пусть } f_n \text{ непрерывна на } D, f_n \overset{D}{\rightrightarrows} f
            (\implies f \text{ непр. на } D)
            \\
            & \text{Тогда: } \int_a^x f_n(t) dt \overset{D}{\rightrightarrows} \int_a^x f(t) dt,
            \\
            & \text{т.е. } \int_a^x \lim_{n \to \infty} f_n(t) dt = \lim_{n \to \infty} \int_a^x f_n(t) dt
        \end{flalign*}
    \end{proposition}
        
    % вопрос 68
        
    % вопрос 69
    
    \subsection*{69. Дайте определение равномерной сходимости функционального ряда.}
        
        $D \subseteq \RR, \; a_n : D \to \RR.$ 
        Рассмотрим функциональный ряд $\displaystyle \sum_{n = 1}^{\infty} a_n(x),$ и его ч.с. $S_N (x) := \displaystyle \sum_{n = 1}^{N} a_n(x).$
        
        Говорят, что ряд сх-ся равномерно на $D$, если последовательность $\{ S_N \}$ сх-ся равномерно на $D$.
        
        
    % вопрос 70
        
    % вопрос 71
    \subsection*{71. Сформулируйте критерий Коши равномерной сходимости функционального ряда.}
    Функциональный ряд $\sum_{n=1}^{\infty} a_n(x)$ сходится равномерно на $D \iff$ $\forall \varepsilon > 0$ $\exists N(\varepsilon)$, $\forall n \geq N$, $\forall m$: 
    $$||a_n + a_{n + 1} + \dots + a_{n + m}|| < \varepsilon$$
            
    Т.е. $|a_n(x) + a_{n + 1}(x) + \dots + a_{n + m}(x)| < \varepsilon$ $\forall x \in D$.
        
    % вопрос 72
        \subsection*{72. Сформулируйте следствие критерия Коши -- достаточное условие того, что функциональный ряд не является сходящимся равномерно.}
        Отрицание критерия Коши 
            
        $\exists \varepsilon > 0\ \forall N \in \NN\ \exists m,n > N\ \exists x = x(N) \in E: |\sum_{k = m}^n x_k| \geqslant \varepsilon \iff$ ряд $\sum x_n$ сходится на $E$ неравномерно
            
    % вопрос 73
    \subsection*{73. Приведите пример функционального ряда, сходящегося на некотором множестве поточечно, но не равномерно (с обоснованием).}
    \begin{example}
        \begin{flalign*}
            & \sum \frac{1}{1 + \left(x - n\right)^2}, \hspace{1cm} D = \RR \\
            & \text{Очевидно, что } \sum \frac{1}{1 + \left(x - n\right)^2} \text{ сходится как гармонический ряд.} \\
            & \text{Проверим необходимое условие равномерной сходимости. } \\
            & a_n(x) = \frac{1}{1 + \left(x - n\right)^ 2} \to 0 \\
            & \text{Но } \sup_{x \in D} \norm{\frac{1}{1 + \left(x - n\right)^2}} \underset{x_n = n}{\geq} \frac{1}{1} \nrightarrow 0 \\
            & \implies \text{ необходимое условие не выполняется } \implies \text{ функциональный ряд сходится неравномерно.} 
        \end{flalign*}
    \end{example}
        
    % вопрос 74
    \subsection*{74. Сформулируйте мажорантный признак Вейерштрасса абсолютной и равномерной сходимости функционального ряда.}
    \begin{proposition}[Признак Вейерштрасса для функционального ряда.]
        Если $\left|a_n(x)\right| \leq b_n$ при $\forall n \geq n_0,\ \forall x \in D$, а ряд $\sum b_n$ сходится, то $\sum a_n(x)$ сходится на $D$ абсолютно и равномерно.
    \end{proposition}
        
    % вопрос 75
    \subsection*{75}
    \textbf{ Как применяются признаки Даламбера и Коши для исследования сходимости функционального ряда?} \\[5 pt]
    \textbf{ Признак Даламбера} \\[5 pt] 
    Если $\exists q < 1 : \; |a_{n+1}(x)| \le q \cdot |a_n(x)|$ при $\forall n \ge n_0, \; x \in D$, 
    причем $a_{n0}(x)$ ограничена на $D$ (т.е. $\norm{a_{n0}} < \infty$), \\[1 pt]
    то $\; \sum a_n(x)$ сходится на $D$ абсолютно и равномерно. \\[5 pt]
    \textbf{ Радикальный признак Коши} \\[5 pt] 
    Если $\sum u_n$ --- знакоположительный числовой ряд, и существует конечный предел \\[3 pt]
    $l = \lim\limits_{n \to \infty} \sqrt[n]{u_n}$, то \\[-20 pt]
    \begin{enumerate}
    \item $l < 1 \;\; \Rightarrow \; $ ряд сходится
    \item $l > 1 \;\; \Rightarrow \; $ ряд расходится
    \item $l = 1 \;\; \Rightarrow \; $ необходимо дополнительное исследование
    \end{enumerate}
    Заметно, что признаки практически идентичны соответствующим признакам для числовых рядов.    
    % вопрос 76
    
    % вопрос 77
    
    \subsection*{77. Сформулируйте признак Лейбница равномерной сходимости знакочередующегося функционального ряда.}
    
    Рассмотрим знакочередующийся функциональный ряд: $\displaystyle \sum_{n = 1}^{\infty} (-1)^n u_n(x), \; u_n(x) \geq 0$ на $D$.
    
    Если $u_n(x) \downarrow_{(n)}$ и $u_n \stackrel{D}{\rightrightarrows} 0$, то ряд сходится равномерно.
        
    % вопрос 78
        
    % вопрос 79
    
    \subsection*{79. Сформулируйте признак Абеля равномерной сходимости функционального ряда.}
        
        Рассмотрим ряд $\displaystyle \sum_{n = 1}^{\infty} a_n(x) b_n(x) = \circledast$.
        
        Если $a_n(x)$ мотонна по $n$ (при $\forall \, x \in D \subseteq \RR$) и $ \| a_n \| \leq C$ при всех $n$,
        
        а ряд $\sum b_n(x)$ сх-ся равномерно, то $\circledast$ сх-ся равномерно.
        
    % вопрос 80
        
    % вопрос 81
    \subsection*{81. Сформулируйте теорему о почленном дифференцировании функционального ряда.}
    $-\infty \leq a < b \leq +\infty$, $D= (a; b)$, $D = [a; b]$ \\
    Пусть $c_n(x)$ дифференцируемы на $D$ и $\sum_{n=1}^{\infty} c'_n(x)$ сходится равномерно на $D$. \\
    Тогда ряд $\sum_{n=1}^{\infty} c_n(x)$ сходится на $D$ (а если $D$ огр, то сходится равномерно), а его сумма будет дифференцируемой функцией на $D$ и $\left(\sum_{n=1}^{\infty} c_n(x)\right)' = \sum_{n=1}^{\infty} c'_n(x)$ \\
        
    % вопрос 82
        \subsection*{82. Сформулируйте теорему о почленном интегрировании функционального ряда.}
        Пусть $-\infty < a < b < +\infty$, $D= (a; b)$, $D = [a; b]$, $c_n$ равномерно сходится на $D$ и имеет суммой функцию $s(x)$

        $\int_{a}^{x}\left(\sum_{n=1}^{\infty} c_n(t)\right) dt = \sum_{n=1}^{\infty} \int_{a}^{x} c_n(t) dt$ -- сходится равномерно на $D$ и имееет суммой функцию $\int_a^x s_n$.
    
    % вопрос 83
    \subsection*{83. Что такое степенной ряд? Как определяются радиус и интервал сходимости степенного ряда? 
    Что можно утверждать о характере сходимости ряда на интервале сходимости?}

    \begin{itemize}
        \item Степенным рядом называется функциональный ряд вида $\sum \limits_{n = 0}^{\infty} c_n \, (x - x_0)^n$, 
    где $c_n$ --- числовая последовательность и $x_0 =$ const.

        \item Радиусом сходимости степенного ряда $\sum \limits_{n = 0}^{\infty} c_n \, (x - x_0)^n$ называется такое число $R$,
    равное \\ ${\sup \, \bigl\{|x - x_0| : \text{ряд сходится}\bigr\}} = {\inf \, \bigl\{|x - x_0| : \text{ряд расходится}\bigr\}}$
    (если ряд сходится всюду, то ${R = +\infty}$). Также по формуле Коши-Адамара $R = \dfrac1{\overline{\lim} \sqrt[n]{|c_n|}}$.

        \item Интервалом сходимости степенного ряда $\sum c_n \, (x - x_0)^n$ с радиусом сходимости $R$ называется интервал $(x_0 - R, \: x_0 + R)$.

        \item Степенной ряд сходится равномерно на отрезке $[x_0 - r, \; x_0 + r]$, если $0 \leqslant r < R$ 
    (неверно утверждать, что это происходит на интервале $(x_0 - R, \: x_0 + R)$). 
    \end{itemize}
        
    % вопрос 84
    \subsection*{84. Что можно утверждать про равномерную сходимость степенного ряда?}
    Пусть есть степенной ряд $\sum a_n \left(x - x_0\right)^n$, его радиус сходимости равен $R$. Тогда в интервале $\left(x_0 - R,\ x_0 + R\right)$ этот ряд сходится абсолютно и равномерно.
        
    % вопрос 85
    \subsection*{85}
    \textbf{ Сформулировать и доказать теорему Абеля о сходимости степенного ряда.} \\[5 pt]
    \textbf{ Теорема Абеля} \\[-15 pt] 
    \begin{enumerate}
    \item[$1)$] Если степенной ряд $\sum c_n (x - x_0)^n$ сходится в точке $x_1 \ne x_0$, 
    то он сходится при всех $x : |x - x_0| < |x_1 - x_0|$
    \item[$2)$] Если степенной ряд $\sum c_n (x - x_0)^n$ расходится в точке $x_2 \ne x_0$, 
    то он расходится при всех $x : |x - x_0| > |x_2 - x_0|$\\[-30 pt]
    \end{enumerate}
    \begin{proof}
    $\left| \sum\limits_{n = m}^N c_n (x - x_0)^n \right| = 
    \left| \sum\limits_{n = m}^N c_n \cdot (x - x_0)^n \cdot \left( \dfrac{x - x_0}{x_1 - x_0} \right)^n \right| \le  \\[3 pt]
    \sum\limits_{n = m}^N \underbrace{\left| c_n \cdot (x - x_0)^n \right|}_{< \varepsilon \; \forall m \ge n_0} \cdot 
    \underbrace{\left| \dfrac{x - x_0}{x_1 - x_0} \right|^n}_{q^n}  \le
    \varepsilon \cdot (q^m + \ldots + q^N) \le \varepsilon \cdot q^m \cdot \dfrac1{1 - q} \to 0$
    \end{proof}    
    % вопрос 86
    
    \subsection*{86. Докажите, что если степенной ряд $\displaystyle \sum c_n (x - x_0)^n$ расходится в точке $x_1$, то он расходится во всех точках $x$, для которых $|x - x_0| > |x_1 - x_0|.$}
        
        \begin{proof}
        
        Докажем, что если $\displaystyle \sum c_n (x - x_0)^n$ сходится в точке $x_1$, то он сходится во всех точках $x$, для которых $|x - x_0| < |x_1 - x_0| \;  \circledast.$ Из этого будет следовать сформулированное выше утверждение (методом от противного).
        
        Итак, доказываем $\circledast$. (Будем рассматривать нетривиальный случай $x_1 \neq x_0$, иначе очевидно).
        
        $ \Bigg| \displaystyle \sum_{n = m}^{N} c_n (x - x_0)^n \Bigg| =
        \Bigg| \sum_{n = m}^{N} c_n \cdot (x_1 - x_0)^n \cdot \left( \frac{x - x_0}{x_1 - x_0} \right)^n \Bigg| \leq
        \sum_{n = m}^{N}  \big| c_n \cdot (x_1 - x_0)^n \big| \cdot \bigg| \frac{x - x_0}{x_1 - x_0} \bigg|^n = \bigstar$.
        
        Заметим, что $\big| c_n \cdot (x_1 - x_0)^n \big| < \varepsilon $ при $m \geq n_0 (\varepsilon)$ (следствие из необходимого условия сходимости).
        
        Далее, (при наших условиях) $\sum \bigg| \frac{x - x_0}{x_1 - x_0} \bigg|^n$ образуют геом. прогрессию, где $q = \bigg| \frac{x - x_0}{x_1 - x_0} \bigg| < 1.$
        
        Так что $\bigstar \leq \varepsilon \cdot (q^m + \dots + q^n)
        \leq \varepsilon \cdot  q^m \cdot \frac{1}{1 - q} \to 0.$ 
        
        Почему к нулю? При $m \to \infty $ выражение $q^m \cdot \frac{1}{1 - q}$ остается ограниченным одной и той же константой, а $\varepsilon$ - это произвольная сколь угодно малая величина.
        
        Итог: ряд сходится по критерию Коши.
        
        
        \end{proof}
        
    % вопрос 87
    \subsection*{87. Выведите формулу Коши-Адамара для радиуса сходимости степенного ряда.}
    
    Для степенного ряда $\displaystyle \sum_{n = 0}^{\infty} c_n \cdot (x - x_0)^n,$ где $\{ c_n \}$ - числовая посл-ть, $x_0 \in \RR$ фиксирован,  $x \in \RR$ - переменная, радиус сходимости $R$ вычислим по формуле Коши-Адамара:
    
    \fbox{$R =  \frac{1}{\overline{\lim} \sqrt[n]{|c_n|}}$}
    
        \begin{proof} 
        В нашем ряде $a_n(x) = c_n \cdot (x - x_0)^n.$ Применим радикальный признак Коши: 
        
        $\sqrt[n]{|a_n(x)|} = \sqrt[n]{|c_n|} \cdot |x - x_0| \implies
        \overline{\lim} \sqrt[n]{|a_n(x)|} = \overline{\lim} \sqrt[n]{|c_n|} \cdot |x - x_0| =
        |x - x_0| \cdot \overline{\lim} \sqrt[n]{|c_n|} \implies $ 
        
        если $|x - x_0| \cdot \overline{\lim} \sqrt[n]{|c_n|} < 1,$ то ряд сх-ся;
        
        если $|x - x_0| \cdot \overline{\lim} \sqrt[n]{|c_n|} > 1,$ то ряд расх-ся.
        
        Введем $R := \frac{1}{\overline{\lim} \sqrt[n]{|c_n|}}.$
        
        Из полученных результатов ясно, что $|x - x_0| < R \iff |x - x_0| \cdot \overline{\lim} \sqrt[n]{|c_n|} < 1$ и ряд сходится; 
        
        $|x - x_0| > R \iff |x - x_0| \cdot \overline{\lim} \sqrt[n]{|c_n|} > 1$ и ряд расходится. А это определение радиуса сходимости.
        
        \end{proof}
        
    % вопрос 88
        
    % вопрос 89
    \subsection*{89. Пусть $\displaystyle R$ -- радиус сходимости степенного
    ряда. Сформулируйте и докажите теорему о равномерной сходимости степенного ряда
    на $\displaystyle [0; R]$.}
    \theorem{Пусть $\displaystyle \sum c_n R^n$ сходится. Тогда степенной ряд
    $\displaystyle \sum c_n(x - x_0)^n$ сходится равномерно на $\displaystyle
     [x_0, x_0 + R]$}
    \begin{proof} \ \\
        В начале доказательства хочу заметить,
         что доказываю не совсем то, что в вопросе просят, однако это вроде 
         ошибка Маевского, а не моя. Для того, чтобы всё было, как вопросе, возьмите
         $\displaystyle x_0 = 0$. \\ 
         $\displaystyle \sum_{n = 0}^{\infty}c_n(x - x_0)^n =
         \sum_{n = 0}^{\infty}(c_nR^n)\cdot \left(\frac{x - x_0}{R}\right)^n$\\
         $\displaystyle \sum_{n = 0}^{\infty}(c_nR^n)$ -- сходится равномерно 
         (от $\displaystyle x$ не зависит) \\ 
         $\displaystyle \sum_{n = 0}^{\infty}\left(\frac{x - x_0}{R}\right)^n 
         \downarrow_{(n)}$ \ $\displaystyle \forall x \in [x_0, x_0 + R]$ \\ 
         $\displaystyle \Rightarrow \sum_{n = 0}^{\infty}c_n(x - x_0)^n$ -- сходится
         равномерно на $\displaystyle
         [x_0, x_0 + R]$
         по признаку Абеля 
    \end{proof}
    
    % вопрос 90
        
    % вопрос 91
        
    % вопрос 92
        
    % вопрос 93
    \subsection*{93. Что можно утверждать о радиусе сходимости степенного ряда, полученного почленным интегрировании исходного ряда? Обоснуйте ответ.}

    Радиус сходимости при интегрировании степенного ряда не изменяется.

    \begin{proof}
        Пусть дан степенной ряд $\sum c_n \, (x - x_0)^n$ с радиусом сходимости $R$. Утверждается (доказательство в пункте 95), что
        \begin{flalign*}
            \int_{x_0}^x \!\left( \sum_{n = 0}^{\infty} c_n \, (t - x_0)^n \right) \! dt \;=\; 
        \sum_{n = 0}^{\infty} \frac{c_n}{n + 1} \, (x - x_0)^{n + 1} = 
        \sum_{n = 0}^{\infty} c'_n (x - x_0)^{n} \text{, где } c'_n =
            \begin{cases}
            0, & n = 0 \\[2 pt]
            \dfrac{c_{n - 1}}{n}, & n > 0 \\
            \end{cases}
        \end{flalign*}
        По формуле Коши-Адамара радиус $R'$ сходимости нового ряда равен
        \begin{flalign*}
            \frac1{\overline{\lim} \sqrt[n]{|c'_n|}} = 
        \frac{\underline{\lim} \sqrt[n]{n}}{\overline{\lim} \sqrt[n]{|c_n|}} = 
        \frac1{\overline{\lim} \sqrt[n]{|c_n|}} = R
        \end{flalign*}
    \end{proof}
        
    % вопрос 94
    \subsection*{94. Сформулируйте и докажите теорему о почленном дифференцировании степенного ряда.}
    \begin{theorem}[Почленное дифференцирование степенного ряда.]
        $\sum c_n\left(x - x_0\right)^n$, $R > 0$~--- его радиус сходимости.

        При почленном дифференцировании получаем ряд $\sum_{n=0}^\infty c_n \cdot n \cdot \left(x - \right)^{n - 1}$.

        Его радиус сходимости равен радиусу сходимости исходного ряда, то есть он сходится равномерно при $\left|x - x_0\right| \leq r < R$.
    \end{theorem}
    \begin{proof}
        Пусть дан ряд $\sum a_n x^n = f$. Ряд, составленный из производных~--- $\sum a_n n x^{n-1} = f'$.
        Покажем по формуле Коши-Адамара, что радиусы сходимости исходного ряда и ряда, составленного из производных, равны.
        \begin{flalign*}
            &\frac{1}{R_f} = \varlimsup \sqrt[n]{\left|a_n\right|} \\
            &\frac{1}{R_f'} = \varlimsup \sqrt[n]{\left|a_n\right| \cdot n} = \left[\sqrt[n]{n} \to 1\right] = \varlimsup \sqrt[n]{\left| a_n \right|} = \frac{1}{R_f}
        \end{flalign*}
    \end{proof}
    
    % вопрос 95
    \subsection*{95}
    \textbf{ Сформулировать и доказать теорему о почленном интегрировании степенного ряда.} \\[5 pt]
    $\int\limits_{x_0}^x \left( \sum\limits_{n = 0}^{\infty} c_n(t - x_0)^n \right) dt = 
    \sum\limits_{n = 0}^{\infty} \dfrac{c_n}{n + 1}(x - x_0)^{n + 1} = 
    \sum\limits_{n = 1}^{\infty} \dfrac{c_{n - 1}}n (x - x_0)^n$
    \begin{proof}
    $\int\limits_{x_0}^x \left( \sum\limits_{n = 0}^{\infty} c_n(t - x_0)^n \right) dt = 
    \sum\limits_{n = 0}^{\infty} \int\limits_{x_0}^x c_n(t - x_0)^n dt = 
    \sum\limits_{n = 0}^{\infty} \dfrac{c_n}{n + 1}(t - x_0)^{n + 1}\Big|_{x_0}^x = \\[3 pt]
    = \sum\limits_{n = 0}^{\infty} \dfrac{c_n}{n + 1}(x - x_0)^{n + 1} - 
    \sum\limits_{n = 0}^{\infty} \dfrac{c_n}{n + 1}(x_0 - x_0)^{n + 1} = 
    \sum\limits_{n = 0}^{\infty} \dfrac{c_n}{n + 1}(x - x_0)^{n + 1} = \sum\limits_{n = 1}^{\infty} \dfrac{c_{n - 1}}n (x - x_0)^n$
    \end{proof}    
    % вопрос 96
    
    \subsection*{96. Запишите формулу Тейлора для бесконечно дифференцируемой функции с остаточным членом в формах Лагранжа и Коши.}
    
    Если функция $f(x)$ беск. дифф. в точке $x_0$, то $f(x)$ можно сопоставить в соотв. ее ряд Тейлора:
    
    $\displaystyle \sum_{n = 0}^{\infty} \frac{f^{(n)} (x_0)}{n!} (x - x_0)^n.$ При этом $f(x) = \displaystyle \sum_{n = 0}^{N} \frac{f^{(n)} (x_0)}{n!} (x - x_0)^n + r_N (x).$
    
    Фор-ла Лагранжа: $r_N(x) = \frac{ f^{(N + 1)} (x_0 + \Theta(x - x_0))}{(N + 1)!} (x - x_0)^{N + 1}, \; \Theta \in (0, 1). $
    
    Фор-ла Коши: $r_N(x) = \frac{f^{(N + 1)} (x_0 + \Theta(x - x_0))}{N!} (1 - \Theta)^N (x - x_0)^{N + 1}, \; \Theta \in (0, 1).$
        
        
    % вопрос 97
    
    \subsection*{97. Сформулируйте и докажите утверждение о единственности разложения функции в степенной ряд.}
    
    
    Если $f(x) = \displaystyle \sum_{n = 0}^{\infty} c_n \cdot (x - x_0)^n, \; |x - x_0| < \delta$ (говоря иначе, функция представлена степенным рядом в некой окр-ти $x_0$); то этот степенной ряд - ее ряд Тейлора. 
    
        \begin{proof} 
    \begin{flalign}
    & f^{(k)} (x) = 
    \sum_{n = 0}^{\infty} c_n \cdot n \cdot (n - 1) \dots (n - k + 1) \cdot (x - x_0)^{n - k} = \sum_{n = k}^{\infty} c_n \cdot n \cdot (n - 1) \dots (n - k + 1) \cdot (x - x_0)^{n - k} \implies \\
    & f^{(k)} (x_0) = c_k \cdot k! \implies c_k = \frac{f^{(k)}(x_0)}{k!}.
    \end{flalign}
    
    \textit{(Мы заменили в первом переходе нижнюю границу суммирования с нуля на k, так как все предыдущие слагаемые зануляются)}
    
    То есть функция может быть представлена в виде степенного ряда единственным образом - и это будет ее р.Т.\\
    
        \end{proof}
    % вопрос 98
        
    % вопрос 99
    \subsection*{99. Приведите пример бесконечной дифференцируемой функции, 
    не являющейся аналитической.}
    \example{
        $\displaystyle f(x) = \begin{cases}
            e^{-\frac{1}{x^2}}, x \neq 0 \\ 
            0, x = 0
        \end{cases}$ \\
        Такая функция бесконечно дифференцируема, но все её производные
        в нуле равны 0: \\ $\displaystyle f'(0) = f''(0) = f'''(0) = \ldots = 0$ \\
        Получается, что её ряд Тейлора при $\displaystyle x_0 = 0: 
        0 + 0x + 0x^2 + \ldots = 0$\\ 
        То есть, такая функция не является аналитической.
    }    
    % вопрос 100
    \subsection*{100. Запишите разложения в степенной ряд с центром в нуле для функций $e^x$, $\sin x$, $\cos x$. Каково множество сходимости ряда? На каком множестве сумма ряда представляет собой исходную функцию? Обоснуйте ответ.}
    \begin{enumerate}
        \item 
        \begin{flalign*}
            & e^x = \sum_{n=0}^\infty \frac{x^n}{n!}, \hspace{1cm} R = \infty \\
            &\text{Докажем равенство. Оценим остаток по формуле Лагранжа.} \\
            & r_N(x) = \underbrace{e^{\Theta x}}_\text{const} \cdot \underbrace{\frac{x^{N+1}}{\left(N + 1\right)!}}_{\to 0 \text{ при }\forall x}, \hspace{1cm} \Theta \in \left(0;\ 1\right)
        \end{flalign*}
        Множество сходимости~--- $\RR$, на множестве $\RR$ сумма ряда представляет собой исходную функцию.
        \item Для $\cos x$ аналогично:
        \begin{flalign*}
            & \cos x = \sum_{n=0}^\infty \left(-1\right)^n \cdot \frac{x^{2n}}{\left(2n\right)!} \hspace{1cm} x \in \RR, \hspace{0.5cm} R = \infty
        \end{flalign*}
        \item Для $\sin x$ аналогично:
        \begin{flalign*}
            & \sin x = \sum_{n=0}^\infty \left(-1\right)^n \cdot \frac{x^{2n+1}}{\left(2n+1\right)!} \hspace{1cm} x \in \RR, \hspace{0.5cm} R = \infty
        \end{flalign*}
    \end{enumerate}

    
    % вопрос 101
        
    % вопрос 102

\end{document}
