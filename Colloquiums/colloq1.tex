\documentclass[a4paper, fleqn]{article}
\usepackage{header}

\title{Коллоквиум 1}
\author{
        % ОТКОММЕНТИРУЙ СЕБЯ
    Александр Богданов   \\ \href{https://t.me/SphericalPotatoInVacuum}{Telegram} \and
    % Алиса Вернигор       \\ \href{https://t.me/allisyonok}{Telegram} \and
    % Анастасия Григорьева \\ \href{https://t.me/weifoll}{Telegram} \and
    % Василий Шныпко       \\ \href{https://t.me/yourvash}{Telegram} \and
    % Данил Казанцев       \\ \href{https://t.me/vserosbuybuy}{Telegram} \and
    Денис Козлов         \\ \href{https://t.me/DKozl50}{Telegram} \and
    % Елизавета Орешонок   \\ \href{https://t.me/eaoresh}{Telegram} \and
    % Иван Пешехонов       \\ \href{https://t.me/JohanDDC}{Telegram} \and
    % Иван Добросовестнов  \\ \href{https://t.me/ivankot13}{Telegram} \and
    % Настя Городилова     \\ \href{https://t.me/nastygorodi}{Telegram} \and
    % Никита Насонков      \\ \href{https://t.me/nnv_nick}{Telegram} \and
    % Сергей Лоптев        \\ \href{https://t.me/beast_sl}{Telegram}
}

\date{Версия от {\ddmmyyyydate\today} \currenttime}

\begin{document}
\maketitle
% Решение вопроса пишем после комментария 
% комментарии трогать пожалуйста не надо, будет круто

% давайте для постоянства формулировку в сабсекшн кидать
% доказательства советуют оборачивать в \begin{proof} \end{proof}

% вопрос 1

% вопрос 2

% вопрос 3

% вопрос 4

% вопрос 5

% вопрос 6

% вопрос 7

\subsection*{7. Сформулируйте и докажите признак Лобачевского-Коши}

\begin{proposition}{1}
    Пусть $a_n \downarrow$. Рассмотрим ряды:

    \begin{flalign*}
        \sum a_n ~ (1) \text{ и } \sum 2^n \cdot a_{2^n} ~ (2)
    \end{flalign*}

    Тогда ряды (1) и (2) ведут себя одинаково
\end{proposition}

\begin{proof}
    \begin{flalign*}
        &2^m \text{ слаг.: } a_1 + \underbrace{a_2}_{\substack{\leq a_1 \\ \geq a_2}} +
        \underbrace{a_3 + a_4}_{\substack{\leq 2a_2 \\ \geq 2a_4}} +
        \underbrace{a_5 + a_6 + a_7 + a_8}_{\substack{\leq 4a_4 \\ \geq 4a_8}} + \dots +
        \underbrace{a_{2^{m-1}+1} + a_{2^{m-1}+2} + \dots + a_{2^{m-1} +
                    2^{m-1}}}_{\substack{\leq 2^{m-1} \cdot a_{2^{m-1}} \\ \geq \frac{1}{2} \cdot 2^{m} \cdot a_{2^{m}}}}
        \\
        &a_1 + \frac{1}{2} \sum_{n = 1}^{m} 2^na_{2^n} \leq \sum_{n=1}^{2^m}a_n \leq a_1 + \sum_{n=0}^{m-1}2^n \cdot a_{2^n}
        \\
        &\text{Левая часть -- сумма нижних оценок, правая -- сумма верхних}
    \end{flalign*}
\end{proof}

% вопрос 8

% вопрос 9

% вопрос 10

% вопрос 11

% вопрос 12

% вопрос 13

% вопрос 14

% вопрос 15

% вопрос 16

% вопрос 17

\subsection*{17. Приведите пример положительного ряда, вопрос о поведении которого не может быть решён с помощью
    признака Даламбера, но может быть решён с помощью радикального признака Коши (с обоснованием)}
\begin{example}
    \begin{flalign*}
        & 0 < a < 1 < b
        \\
        & 1 + a + ab + a^2b + a^2b^2 + \dots = \sum_{n=1}^\infty a^{\left[\frac{n}{2}\right]} \cdot b^{\left[\frac{n - 1}{2}\right]}
        \\
        & \frac{a_{n+1}}{a_n} = \begin{cases}
            a, \text{ n -- нечёт.} \\
            b, \text{ n -- чёт.}
        \end{cases}
        \implies \overline{\lim}\frac{a_{n+1}}{a_n} = b > 1, \underline{\lim}\frac{a_{n+1}}{a_n} = a < 1
        \\
        & \text{-- признак Даламбера не работает}
        \\
        & \sqrt[n]{a_n} = \begin{cases}
            a^{\frac{1}{2}} \cdot b^{\frac{n-2}{2n}},    & \text{ n -- чёт.}   \\
            a^{\frac{n-1}{2n}} \cdot b^{\frac{n-1}{2n}}, & \text{ n -- нечёт.}
        \end{cases}
        \implies \lim \sqrt[n]{n} = \sqrt{ab}
        \\
        & \text{Если } a \neq b \text{, то радикальный признак работает}
    \end{flalign*}
\end{example}

% вопрос 18

% вопрос 19

% вопрос 20

% вопрос 21

% вопрос 22

% вопрос 23

% вопрос 24

% вопрос 25

% вопрос 26

% вопрос 27

\subsection*{27. Что такое группировка членов ряда? Докажите, что любой ряд, полученный из сходящегося
    группировкой его членов, сходится и имеет ту же сумму.}

\begin{definition}
    Говорят, что ряд $\sum A_k$ получен из ряда $\sum a_n$ группировкой членов, если
    $\exists n_1, n_2, \dots \colon 1 \leq n_1 < n_2 < \dots$ такие, что
    \begin{flalign*}
        & A_1 = a_1 + a_2 + \dots + a_{n_1}
        \\
        & A_2 = a_{n_1 + 1} + a_{n_1 + 2} + \dots + a_{n_2}
    \end{flalign*}
\end{definition}

\begin{proposition}
    Если яд $\sum a_n$ сходится, то ряд $\sum A_k$ тоже сходится, причём к той же сумме.
\end{proposition}

\begin{proof}
    Последовательность частичных сумм $S_k' = A_1 + \dots + A_k$ ряда $\sum A_k$
    явл. подпоследовательностью последовательности частичных сумм $S_n = a_1 + \dots + a_n$ ряда $\sum a_n$
\end{proof}

% вопрос 28

% вопрос 29

% вопрос 30

% вопрос 31

% вопрос 32

% вопрос 33

% вопрос 34

% вопрос 35

% вопрос 36

% вопрос 37

\subsection*{37. Сформулируйте свойство абсолютно сходящегося ряда, связанное с перестановкой членов.}
\begin{proposition}
    Сумма абс. сходящегося ряда не меняется при любой перестановке его членов
\end{proposition}

% вопрос 38

% вопрос 39

% вопрос 40

% вопрос 41

% вопрос 42

% вопрос 43

% вопрос 44

% вопрос 45

% вопрос 46

% вопрос 47

\subsection*{47. Напишите произведение Валлиса и его значение (формула Валлиса). Вычисление каких
    интегралов приводит к этой формуле?}
\begin{proposition}
    Произведение Валлиса
    \begin{flalign*}
        & \prod_{n=1}^\infty \frac{4n^2}{4n^2-1} = \frac{\pi}{2} \text{ -- формула Валлиса}
        \\
        &  \text{-- получается из анализа интегралов } \int_{0}^{\frac{\pi}{2}}\sin^4x dx
    \end{flalign*}
\end{proposition}

% вопрос 48

% вопрос 49

% вопрос 50

% вопрос 51

% вопрос 52

% вопрос 53

% вопрос 54

% вопрос 55

% вопрос 56

% вопрос 57

\subsection*{57. Пусть функциональная последовательность $\{f_n\}$ сходится равномерно на множестве $D$
    к предельной функции $f$, отделённой от нуля (т.е. $\inf_{x \in D} |f(x)| > 0$), то функциональная
    последовательность $\frac{1}{f_n}$ сходится равномерно на $D$ к $\frac{1}{f}$.}

\begin{proof}
    \begin{flalign*}
        & \norm{\frac{1}{f_n} - \frac{1}{n}} = \norm{\frac{f_n - f}{f_n \cdot f}} =
        \sup_{x \in D}{\left|\frac{f_n - f}{f_n \cdot f}\right|} \circled{\leq} \sup_{x \in D} \frac{\epsilon}{|f_n \cdot f|}
        \\
        & \text{т.к. } \norm{f_n - f} \le \epsilon \text{ при } n \geq N(\epsilon)
        \\
        & \inf |f(x)| = m > 0 \implies |f(x)| \geq m  \forall x \in D
        \\
        & \text{Если } \epsilon < m/2 \text{, то } |f_n| \geq |f| - |f_n - f| \geq m - \epsilon \geq m/2
        \\
        & \frac{1}{|f_n|} \leq \frac{1}{m - \epsilon}; ~ \frac{1}{|f|} \leq \frac{1}{m}
        \\
        & \circled{\leq} \frac{\epsilon}{(m - \epsilon)m} \leq \frac{\epsilon}{m/2 \cdot m}
    \end{flalign*}
\end{proof}

% вопрос 58

% вопрос 59

% вопрос 60

% вопрос 61

% вопрос 62

% вопрос 63

% вопрос 64

% вопрос 65

% вопрос 66

% вопрос 67

\subsection*{67. Сформулируйте теорему о почленном интегрировании функциональной последовательности.}
\begin{proposition}
    \begin{flalign*}
        & -\infty < a < b < \infty, ~ D = [a; b]
        \\
        & \text{Пусть } f_n \text{ непрерывна на } D, f_n \overset{D}{\rightrightarrows} f
        (\implies f \text{ непр. на } D)
        \\
        & \text{Тогда: } \int_a^x f_n(t) dt \overset{D}{\rightrightarrows} \int_a^x f(t) dt,
        \\
        & \text{т.е. } \int_a^x \lim_{n \to \infty} f_n(t) dt = \lim_{n \to \infty} \int_a^x f_n(t) dt
    \end{flalign*}
\end{proposition}

% вопрос 68

% вопрос 69

% вопрос 70

% вопрос 71

% вопрос 72

% вопрос 73

% вопрос 74

% вопрос 75

% вопрос 76

% вопрос 77

% вопрос 78

% вопрос 79

% вопрос 80

% вопрос 81

% вопрос 82

% вопрос 83

% вопрос 84

% вопрос 85

% вопрос 86

% вопрос 87

% вопрос 88

% вопрос 89

% вопрос 90

% вопрос 91

% вопрос 92

% вопрос 93

% вопрос 94

% вопрос 95

% вопрос 96

% вопрос 97

% вопрос 98

% вопрос 99

% вопрос 100

% вопрос 101

% вопрос 102

\end{document}
