\documentclass[a4paper, fleqn]{article}

\usepackage{header}

\title{Коллоквиум 1}
\author{
        % ОТКОММЕНТИРУЙ СЕБЯ
    % Александр Богданов   \\ \href{https://t.me/SphericalPotatoInVacuum}{Telegram} \and
    % Алиса Вернигор       \\ \href{https://t.me/allisyonok}{Telegram} \and
    % Анастасия Григорьева \\ \href{https://t.me/weifoll}{Telegram} \and
    % Василий Шныпко       \\ \href{https://t.me/yourvash}{Telegram} \and
    % Данил Казанцев       \\ \href{https://t.me/vserosbuybuy}{Telegram} \and
    Денис Козлов         \\ \href{https://t.me/DKozl50}{Telegram} \and
    % Елизавета Орешонок   \\ \href{https://t.me/eaoresh}{Telegram} \and
    % Иван Пешехонов       \\ \href{https://t.me/JohanDDC}{Telegram} \and
    % Иван Добросовестнов  \\ \href{https://t.me/ivankot13}{Telegram} \and
    % Настя Городилова     \\ \href{https://t.me/nastygorodi}{Telegram} \and
    % Никита Насонков      \\ \href{https://t.me/nnv_nick}{Telegram} \and
    % Сергей Лоптев        \\ \href{https://t.me/beast_sl}{Telegram}
}

\date{Версия от {\ddmmyyyydate\today} \currenttime}

\begin{document}
    \maketitle
    % Решение вопроса пишем после комментария 
    % комментарии трогать пожалуйста не надо, будет круто
    
    % давайте для постоянства формулировку в сабсекшн кидать
    % доказательства советуют оборачивать в \begin{proof} \end{proof}
    
    % вопрос 1

    % вопрос 2

    % вопрос 3
        
    % вопрос 4
        
    % вопрос 5
        
    % вопрос 6
        
    % вопрос 7
        
    % вопрос 8
        
    % вопрос 9
        
    % вопрос 10
        
    % вопрос 11
        
    % вопрос 12
        
    % вопрос 13
        
    % вопрос 14
        
    % вопрос 15
        
    % вопрос 16
        
    % вопрос 17
        
    % вопрос 18
        
    % вопрос 19
        
    % вопрос 20
        
    % вопрос 21
        
    % вопрос 22
        
    % вопрос 23
        
    % вопрос 24
        
    % вопрос 25
        
    % вопрос 26
        
    % вопрос 27
        
    % вопрос 28
        
    % вопрос 29
        
    % вопрос 30
        
    % вопрос 31
        
    % вопрос 32
        
    % вопрос 33
        
    % вопрос 34
        
    % вопрос 35
        
    % вопрос 36
        
    % вопрос 37
        
    % вопрос 38
        
    % вопрос 39
        
    % вопрос 40
        
    % вопрос 41
        
    % вопрос 42
        
    % вопрос 43
        
    % вопрос 44
        
    % вопрос 45
        
    % вопрос 46
        
    % вопрос 47
        
    % вопрос 48
        
    % вопрос 49
        
    % вопрос 50
        
    % вопрос 51
        
    % вопрос 52
        
    % вопрос 53
        
    % вопрос 54
        
    % вопрос 55
        
    % вопрос 56
        
    % вопрос 57
        
    % вопрос 58
        
    % вопрос 59
        
    % вопрос 60
        
    % вопрос 61
        
    % вопрос 62
        
    % вопрос 63
        
    % вопрос 64
        
    % вопрос 65
        
    % вопрос 66
        
    % вопрос 67
        
    % вопрос 68
        
    % вопрос 69
        
    % вопрос 70
        
    % вопрос 71
        
    % вопрос 72
        
    % вопрос 73
        
    % вопрос 74
        
    % вопрос 75
        
    % вопрос 76
        
    % вопрос 77
    
    \subsection{Сформулируйте признак Лейбница равномерной сходимости знакочередующегося функционального ряда.}
    
    Рассмотрим знакочередующийся функциональный ряд: $\displaystyle \sum_{n = 1}^{\infty} (-1)^n u_n(x), \; u_n(x) \geq 0$ на $D$.
    
    Если $u_n(x) \downarrow_{(n)}$ и $u_n \stackrel{D}{\rightrightarrows} 0$, то ряд сходится равномерно.
        
    % вопрос 78
        
    % вопрос 79
        
    % вопрос 80
        
    % вопрос 81
        
    % вопрос 82
        
    % вопрос 83
        
    % вопрос 84
        
    % вопрос 85
        
    % вопрос 86
        
    % вопрос 87
    
    \subsection{Выведите формулу Коши-Адамара для радиуса сходимости степенного ряда.}
    
    Для степенного ряда $\displaystyle \sum_{n = 0}^{\infty} c_n \cdot (x - x_0)^n,$ где $\{ c_n \}$ - числовая посл-ть, $x_0 \in \RR$ фиксирован,  $x \in \RR$ - переменная, радиус сходимости $R$ вычислим по формуле Коши-Адамара:
    
    \fbox{$R =  \frac{1}{\overline{\lim} \sqrt[n]{|c_n|}}$}
    
        \begin{proof} 
        В нашем ряде $a_n(x) = c_n \cdot (x - x_0)^n.$ Применим радикальный признак Коши: 
        
        $\sqrt[n]{|a_n(x)|} = \sqrt[n]{|c_n|} \cdot |x - x_0| \implies
        \overline{\lim} \sqrt[n]{|a_n(x)|} = \overline{\lim} \sqrt[n]{|c_n|} \cdot |x - x_0| =
        |x - x_0| \cdot \overline{\lim} \sqrt[n]{|c_n|} \implies $ 
        
        если $|x - x_0| \cdot \overline{\lim} \sqrt[n]{|c_n|} < 1,$ то ряд сх-ся;
        
        если $|x - x_0| \cdot \overline{\lim} \sqrt[n]{|c_n|} > 1,$ то ряд расх-ся.
        
        Введем $R := \frac{1}{\overline{\lim} \sqrt[n]{|c_n|}}.$
        
        Из полученных результатов ясно, что $|x - x_0| < R \iff |x - x_0| \cdot \overline{\lim} \sqrt[n]{|c_n|} < 1$ и ряд сходится; 
        
        $|x - x_0| > R \iff |x - x_0| \cdot \overline{\lim} \sqrt[n]{|c_n|} > 1$ и ряд расходится. А это определение радиуса сходимости.
        
        \end{proof}
        
    % вопрос 88
        
    % вопрос 89
        
    % вопрос 90
        
    % вопрос 91
        
    % вопрос 92
        
    % вопрос 93
        
    % вопрос 94
        
    % вопрос 95
        
    % вопрос 96
        
    % вопрос 97
    
    \subsection{Сформулируйте и докажите утверждение о единственности разложения функции в степенной ряд.}
    
    
    Если $f(x) = \displaystyle \sum_{n = 0}^{\infty} c_n \cdot (x - x_0)^n, \; |x - x_0| < \delta$ (говоря иначе, функция представлена степенным рядом в некой окр-ти $x_0$); то этот степенной ряд - ее ряд Тейлора. 
    
        \begin{proof} 
    \begin{flalign}
    & f^{(k)} (x) = 
    \sum_{n = 0}^{\infty} c_n \cdot n \cdot (n - 1) \dots (n - k + 1) \cdot (x - x_0)^{n - k} = \sum_{n = k}^{\infty} c_n \cdot n \cdot (n - 1) \dots (n - k + 1) \cdot (x - x_0)^{n - k} \implies \\
    & f^{(k)} (x_0) = c_k \cdot k! \implies c_k = \frac{f^{(k)}(x_0)}{k!}.
    \end{flalign}
    
    \textit{(Мы заменили в первом переходе нижнюю границу суммирования с нуля на k, так как все предыдущие слагаемые зануляются)}
    
    То есть функция может быть представлена в виде степенного ряда единственным образом - и это будет ее р.Т.\\
    
        \end{proof}

    % вопрос 98
        
    % вопрос 99
        
    % вопрос 100
        
    % вопрос 101
        
    % вопрос 102

\end{document}
